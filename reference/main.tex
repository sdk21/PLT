\documentclass[11pt]{article}

\def\withcolors{0}
\def\withnotes{0}
\def\withindex{1}
\def\withbiblioconversion{1} % long names for the journals
%%%%%%%%%%%%%%%%%%%%%%%%%%%%%%%%%%%%%%%%%%%%%%%%%%%%%%%%%%%%%%%%%%%%%%%%%%%%%%%%%%%%
\usepackage[T1]{fontenc}
\usepackage[utf8]{inputenc}

%% Eye-candy
\usepackage{lmodern}
\usepackage{xspace}                                     % Smart spacing with \xspace
\usepackage[protrusion=true,expansion=true]{microtype}  % Improve font rendering

% Striking out text
\usepackage[normalem]{ulem}

%% Math
\usepackage{amsfonts,amsmath,amssymb, amsthm, mathtools}
\usepackage{dsfont} % For the indicator symbol

% Algorithm environment
\usepackage{algorithmicx,algpseudocode,algorithm}

% Colors (with names)
\usepackage[usenames,dvipsnames,table]{xcolor}

% Bibliography
%\usepackage[numbers]{natbib}

% Required for the table of results
\usepackage{multirow}
\usepackage{chngpage} % allows for temporary adjustment of side margins

% Graphics
\usepackage{tikz}
\usetikzlibrary{arrows}
\usetikzlibrary{calc,decorations.pathmorphing,patterns}

% References and links
\usepackage[backref,colorlinks,citecolor=blue,bookmarks=true]{hyperref}
\usepackage{aliascnt}
\usepackage[numbered]{bookmark}

% Full pages
\usepackage{fullpage}

% Compressed lists
\usepackage[shortlabels]{enumitem}
  \setitemize{noitemsep,topsep=3pt,parsep=2pt,partopsep=2pt} % Uncomment for compact item lists
  \setenumerate{itemsep=1pt,topsep=2pt,parsep=2pt,partopsep=2pt}
  \setdescription{itemsep=1pt}
  
% Package for todo notes.
\ifnum\withnotes=1
  \usepackage[colorinlistoftodos,textsize=scriptsize]{todonotes}
\fi

\input{preamble}
\usepackage{multicol}

\newcommand{\complex}{\texttt{com}\xspace}
\newcommand{\integ}{\texttt{int}\xspace}
\newcommand{\float}{\texttt{float}\xspace}
\newcommand{\mat}{\texttt{mat}\xspace}
%\newcommand{\cvect}{\texttt{cvect}\xspace}
%\newcommand{\rvect}{\texttt{rvect}\xspace}
\newcommand{\qubit}{\texttt{qub}\xspace}

\newcommand{\ket}[1]{|#1\rangle}
\newcommand{\bra}[1]{\langle #1|}
\def\dotproduct[#1,#2]{
\langle #1 | #2 \rangle
}

\lstset{ %
  backgroundcolor=\color{white},   % choose the background color; you must add \usepackage{color} or \usepackage{xcolor}
  basicstyle=\small,        % the size of the fonts that are used for the code
  breakatwhitespace=false,         % sets if automatic breaks should only happen at whitespace
  breaklines=true,                 % sets automatic line breaking
  captionpos=b,                    % sets the caption-position to bottom
  commentstyle=\color{OliveGreen},    % comment style
  deletekeywords={...},            % if you want to delete keywords from the given language
  escapeinside={\%*}{*)},          % if you want to add LaTeX within your code
  extendedchars=true,              % lets you use non-ASCII characters; for 8-bits encodings only, does not work with UTF-8
%  frame=single,                    % adds a frame around the code
  keepspaces=true,                 % keeps spaces in text, useful for keeping indentation of code (possibly needs columns=flexible)
  keywordstyle=\color{blue},       % keyword style
  language=sh,                 % the language of the code
  morekeywords={pi,e,int, float, comp, rvect, cvect, mat, qub, true, false,if, elif, else,def, for, from, to, by, while, break,or, and, xor,not, re, im, norm, isunit, trans, det, adj, conj, sin, cos, tan, exp},                     % if you want to add more keywords to the set
  numbers=left,                    % where to put the line-numbers; possible values are (none, left, right)
  numbersep=5pt,                   % how far the line-numbers are from the code
  numberstyle=\tiny\color{Gray}, % the style that is used for the line-numbers
  rulecolor=\color{black},         % if not set, the frame-color may be changed on line-breaks within not-black text (e.g. comments (green here))
  showspaces=false,                % show spaces everywhere adding particular underscores; it overrides 'showstringspaces'
  showstringspaces=false,          % underline spaces within strings only
  showtabs=false,                  % show tabs within strings adding particular underscores
  stepnumber=2,                    % the step between two line-numbers. If it's 1, each line will be numbered
  stringstyle=\color{RedViolet},     % string literal style
  tabsize=2,                       % sets default tabsize to 2 spaces
  title=\lstname                   % show the filename of files included with \lstinputlisting; also try caption instead of title
}

%\def\twobytwomatrix[#1,#2,#3,#4]{
%\begin{bmatrix}
%	#1 & #2 \\
%	#3 & #4
%\end{bmatrix}
%}

\newcommand{\QL}{\textsf{QLang}\xspace}
\title{\QL: Qubit Language\\ \Large(Reference Manual)}
\author{
  Christopher Campbell
  \and Cl\'ement Canonne
  \and Sankalpa Khadka
  \and Winnie Narang
  \and Jonathan Wong
}

\begin{document}

\maketitle
\tableofcontents
\clearpage

\section{Introduction}
\section{Lexical conventions}
There are five kinds of tokens: identifiers, keywords, constants, expression operators, and other separators. There are six kinds of tokens: identifiers, keywords, constants, strings, expression operators, and other separators. If the input stream has been parsed into tokens up to a given character, the next token is taken to include the longest string of characters which could possibly constitute a token.\cmargin{Rephrase: that's plagiarism}

\subsection{Character set}
\QL supports a subset of ASCII; that is, allowed characters are
\fbox{\texttt{a-zA-Z0-9@\#,-\_;:()[]\{\}<>=+/|*}}, as well as tabulations \texttt{\textbackslash{}t}, spaces, and line returns \texttt{\textbackslash{}n} and {\textbackslash{}r}.
\subsection{Comments}
Comments start with a \# sign, which then extends until the next carriage return. Multiline comments are not supported.

\subsection{Identifier (names)}
An identifier is an arbitrarily long sequence of alphabetic and numeric characters, where \texttt{\_} is included as ``alphabetic''. It must start with a lowercase or uppercase letter, i.e. one of $\texttt{a-zA-Z}$.

\noindent The language is case-sensitive: \texttt{hullabaloo} and \texttt{hullABaLoo} are considered as different.

\subsection{Keywords}
The following identifiers as reserved for keywords, and no one shall use them because it's forbidden and uncool.
\begin{multicols}{2}
\verbatiminput{keywords.txt}
\end{multicols}

\subsection{Constants}
There are three sorts of constants in the language, namely \emph{integer}, \emph{complex} and \emph{identifier} constants. The first are comprised of any sequence of integers of the form \texttt{0|([1-9][0-9]*)} (recall that integers are non-negative), and have type \integ. The second are of type \complex and have the form 
\texttt{$R$|$R$+$R$i|$R$i}
where $R$ consists of a \textsf{(i)} sign, \textsf{(ii)} an integer part followed by \textsf{(iii)} a point, \textsf{(iv)} a decimal part, then  \textsf{(v)} either a \texttt{e} or a \texttt{E} followed by an exponent part, possibly signed. \textsf{(i)} and \textsf{(v)} are optional, and either \textsf{(ii)} or \textsf{(iv)} can be missing as well. In more detail, $R$ 
is defined as \texttt{[+-]\{0,1\}((($A$.$B$*|.$B$+)([eE][+-]?B+)?)|$A$[eE][+-]?B+)} and $A=$\texttt{0|([1-9]$B$*)}, $B=$\texttt{0|[1-9]} (that is, $R$ matches a real number such as \texttt{2.78e5}, \texttt{1.5E-1} or \texttt{10.25}).\todo{check this paragraph.}

\noindent Finally, the identifier constants are a subset of the reserved keywords, and include:
\begin{description}
  \item[\texttt{e}] the base of natural logarithm $e=\sum_{k=0}^\infty \frac{1}{k!}$. Equivalent to \texttt{exp(1)}; has type \complex.
  \item[\texttt{Pi}] the constant $\pi$. Has type \complex.
  \item[\texttt{true}] represents the Boolean value \textsf{true}. Stored internally  as \integ 1.
  \item[\texttt{false}] represents the Boolean value \textsf{false}. Stored internally  as \integ 0.
\end{description}

%\section{Syntax notation}
An operation, or language elementary unit, starts from the end of the previous one, and ends whenever a semicolon (that is not part of a matrix declaration) is encountered.

%\section{What's in a Name?}
\section{Objects and types}
\subsection{Objects and lvalues}
As in C, ``an object is a manipulatable region of storage; an lvalue is an expression referring to an object.''

\subsection{Valid types}

The language features 4 elementary types, namely \integ, \float, \complex, \mat. Is also valid, any type that inductively can be built from a valid type as follows:
\begin{itemize}
  \item elementary types are valid;
  \item an \emph{matrix} of a valid type is valid. Matrices have fixed size (that must be declared at compilation time), and are comprised of any elements of any type (that is, a matrix can have elements of non-necessarily identical types);
  \item a \emph{function} taking as input a fixed number of elements from (non-necessarily identical) valid types, and returning a valid type.
\end{itemize}

\section{Conversions}
Applying unary or binary operators to some values may cause an implicit conversion of their operands. In this section, we list the possible conversions, and their expected result -- any conversion not listed here is impossible, and attempting to force it would generate a compilation error.

\begin{itemize}
  \item $\complex\to\float$: the imaginary part of the complex number is dropped (will generate a warning).
  \item $\float\to\integ$: the floating number is rounded towards zero.
  \item $\complex\to\integ$: equivalent to $\complex\to\float\to\integ$.
\end{itemize}


\section{Expressions}
\subsection{Operator Precedence}
\begin{flushleft}
\begin{tabular}{|l|l|l|}
	\hline    
	                   
	Operator Type & 
	Operator & 
	Associativity\\
	
	\hline
	Primary Expressions &
	\textsf{() [] <| |>} &
	Left\\
	Unary & 
	\textsf{re im norm unit trans det adj conj sin cos tan - $|$ not} & 
	Right\\
	Binary & 
	\textsf{\string^ $|$ * / \% $|$ + - $|$ lt gt leq geq $|$ eq neq $|$ and $|$ or xor } &
	Left (except \string^ which is Right)\\
	Assignment &
	\textsf{=} 
	& Right\\
	
	\hline  
\end{tabular}
\end{flushleft}

\subsection{Literals}
Literals are integers, floats, complex numbers, qubits, and matrices, as well as the built-in constants of the language (e.g. \textsf{pi}). Integers are of type \integ, floats are of type \float, complex numbers are of type \complex, qubits and matrices are of type \mat. The built-in constants have pre-determined types described above (e.g. \textsf{pi} is of type \float).

The remaining major subsections of this section describe the groups of \textit{expression} operators, while the minor subsections describe the individual operators within a group.
\subsection{Primary Expressions}
\subsubsection{identifier}
Identifiers are primary expression. All identifiers have an associated type that is given to them upon declaration (e.g. \float \textit{ident} declares an identifier named ident that is of type \float).
\subsubsection{literals}
Literals are primary expression. They are described above.

\subsubsection{(\textit{expression})}
Parenthesized expressions are primary expressions. The type and value of a parenthesized expression is the same as the type and value of the expression without parenthesis. Parentheses allow expressions to be evaluated in a desired precedence. Parenthesized expressions are evaluated relative to each other starting with the expression that is nested the most deeply and ending with the expression that is nested the least deeply (i.e. the shallowest).

\subsubsection{\textit{primary-expression}(\textit{expression-list})}
Primary expressions followed by a parenthesized expression list are primary expressions. Such primary expressions can be used in the declaration of functions or function calls. The expression list must consist of one or more expressions separated by commas. If being used in function declarations, they must be preceded by the correct function declaration syntax and each \textit{expression} in the expression list must evaluate to a type followed by an identifier. If being used in function calls each \textit{expression} in the expression list must evaluate to an identifier.

\subsubsection{\textit{primary-expression}[\textit{expression-list}]}
Primary expressions followed by a bracketed expression list are primary expressions. Such primary expressions can be used in the declaration of matrices and arrays or to access an element of a matrix or array. The expression list must consist of one (for matrices and arrays) or two (for matrices) expressions separated by commas, and must evaluate to \integ.

\subsubsection{[\textit{expression}-elementlist]}
Expression element lists in brackets are primary expressions. Such primary expressions are used to define matrices and therefore are of type \mat. The expression element list must consist of one or more expressions separated by commas or parenthsized. Commas separate expressions into matrix columns and parentheses group expressions into matrix rows. The expressions must evaluate to the same type and can be of type \integ, \float, \complex, or \mat. Additionally, the number of expressions in each row of the matrix must be the same. An example matrix is shown below.

\begin{lstlisting}
int a = 3;
int b = 12;
mat my_matrix = [ (0+1, 2, a)( 5-1, 2*3-1, 12/2)];
\end{lstlisting}

\subsubsection{<\textit{expression}|}
Expressions with a less than sign on the left and a bar on the right are primary expression. Such expressions are used to define qubits and therefore are of type \mat. The notation is meant to mimic the "bra-" of "bra-ket" notation and can therefore be thought of as a row vector representation of the given qubit. Following "bra-ket" notation, the expression must evaluate  to an integer literal of only 0's and 1's, which represents the state of the qubit. An example "bra-" qubit is shown below.

\begin{lstlisting}
mat b_qubit = <0100|;
\end{lstlisting}

\subsubsection{|\textit{expression}>}
Expressions with a bar on the left and a greater than sign on the right are primary expression. All of the considerations are the same as for <\textit{expression}|, except that this notation mimics the "ket" of "bra-ket" notation and can therefore be though of as a column vector representation of the given qubit. An example "ket-" qubit is shown below.

\begin{lstlisting}
int a = 001;
mat k_qubit = |a>;
\end{lstlisting}

\subsection{Unary Operators}
\subsubsection{not \textit{expression}}
The result is a Boolean indicating the logical \textsf{not} of the \textit{expression}. The type of the expression must be \integ or \float. In the \textit{expressions}, 0 is considered false and all other values are considered true.
\subsubsection{re \textit{expression}}
The result is the real component of the \textit{expression}. The type of the expression must be  \complex. The result has the same type as the expression (it is a complex number with  0 imaginary component).
\subsubsection{im \textit{expression}}
The result is the imaginary component of the \textit{expression}. The type of the expression must be  \complex. The result has the same type as the expression (it is a complex number with  0 real component).
\subsubsection{norm \textit{expression}}
The result is the norm of the \textit{expression}. The type of the expression must be \mat, \complex, or \float. The result has type \float, and corresponds to the $2$-norm; in the case of \complex or \float, this coincides with respectively the module and absolute value.
% if the expression is an integer matrix or \float matrix and type \complex if the expression is a complex number matrix.
\subsubsection{unit \textit{expression}}
The result is a Boolean indicating whether the expression is a unit matrix. The type of the expression must be \mat.
\subsubsection{trans \textit{expression}}
The result is the transpose of the \textit{expression}. The type of the expression must be \mat. The result has the same type as the \textit{expression}.
\subsubsection{det \textit{expression}}
The result is the determinant of the \textit{expression}. The type of the expression must be \mat. The result has type \float if the expression is an integer matrix or \float matrix and type \complex if the expression is a complex number matrix.
\subsubsection{adj \textit{expression}}
The result is the adjoint of the \textit{expression}. The type of the expression must be \mat. The result has the same type as the \textit{expression}.
\subsubsection{conj \textit{expression}}
The result is the complex conjugate of the \textit{expression}. The type of the expression must be \complex or \mat. The result has the same type as the \textit{expression}.
\subsubsection{sin \textit{expression}}
The result is the evaluation of the trigonometric function sine on the \textit{expression}. The type of the expression must be \integ, \float, or  \complex. The result has type \float if the expression is of type \integ or \float and type \complex if the expression is of type  \complex.
\subsubsection{cos \textit{expression}}
The result is the evaluation of the trigonometric function cosine on the \textit{expression}. The type of the expression must be \integ, \float, or  \complex. The result has type \float if the expression is of type \integ or \float and type \complex if the expression is of type  \complex.
\subsubsection{tan \textit{expression}}
The result is the evaluation of the trigonometric function tangent on the \textit{expression}. The type of the expression must be \integ, \float, or  \complex. The result has type \float if the expression is of type \integ or \float and type \complex if the expression is of type  \complex. \new{(If an error occured because of a division by zero, a runtime exception is raised.)}
\subsection{Binary Operators}
\subsubsection{\textit{expression} $\hat{}$ \textit{expression}}
The result is the exponentiation of the first \textit{expression} by the second \textit{expression}. The types of the expression must be of type \integ, \float, or  \complex. If the expressions are of the same type, the result has the same type as the \textit{expressions}. Otherwise, if at least one \textit{expression} is a \complex, the result is of type \complex; if neither expressions are comp, but at least one is \float, the result is of type \float.
\subsubsection{\textit{expression} * \textit{expression}}
The result is the product of the \textit{expressions}. The type considerations are the same as they are for \textit{expression} $\hat{}$ \textit{expression} except that it also allows for matrices.
\subsubsection{\textit{expression} / \textit{expression}}
The result is the quotient of the \textit{expressions}, where the first \textit{expression} is the dividend and the second is the divisor. The type considerations are the same as they are for \textit{expression} $\hat{}$ \textit{expression}. Integer division is rounded towards 0 and truncated. \new{(If an error occured because of a division by zero, a runtime exception is raised.)}
\subsubsection{\textit{expression} \% \textit{expression}}
The result is the remainder of the division of the \textit{expressions}, where the first \textit{expression} is the dividend and the second is the divisor. The sign of the dividend and the divisor are ignored, so the result returned is always the remainder of the absolute value (or module) of the dividend divided by the absolute value of the divisor. The type considerations are the same as they are for \textit{expression} $\hat{}$ \textit{expression}.
\subsubsection{\textit{expression} + \textit{expression}}
The result is the sum of the \textit{expressions}. The types of the expressions must be of type \integ, \float, \complex, or \mat. If at least one \textit{expression} is a \complex, the result is of type \complex; if neither expressions are comp, but at least one is \float, the result is of type \float. Qubits and matrices are special and can only be summed with within operands of the same type (and, in the case of matrices, dimensions).
\subsubsection{\textit{expression} - \textit{expression}}
The result is the difference of the first and second \textit{expression}. The type considerations are the same as they are for \textit{expression} + \textit{expression}.
\subsubsection{\textit{expression} @ \textit{expression}}
The result is the tensor product of the first and second \textit{expressions}. The expressions must be of type of \mat. The result has the same type as the \textit{expression}.
\subsubsection{\textit{expression} eq \textit{expression}}
The result is a Boolean indicating if it is true or false that the two \textit{expression} are equivalent. The type of the expressions must either be the same, or one of the two should be implicitly convertible to the other type (e.g., $0.2 \textsf{ eq } 1$, where the right-hand side is an \integ that can be cast into a \float).
\subsubsection{\textit{expression} lt \textit{expression}}
The result is a Boolean indicating if it is true or false that the first \textit{expression} is less than the second. The type of the expressions must be \integ or \float.% and must be the same.  
\subsubsection{\textit{expression} gt \textit{expression}}
The result is a Boolean indicating if it is true or false that the first \textit{expression} is greater than the second. The type of the expressions must be \integ or \float.% and must be the same.  
\subsubsection{\textit{expression} leq \textit{expression}}
The result is a Boolean indicating if it is true or false that the first \textit{expression} is less than  or equal to the second. The type of the expressions must be \integ or \float.% and must be the same.  
\subsubsection{\textit{expression} geq \textit{expression}}
The result is a Boolean indicating if it is true or false that the first \textit{expression} is greater than or equal to the second. The type of the expressions must be \integ or \float.% and must be the same.  
\subsubsection{\textit{expression} or \textit{expression}}
The result is a Boolean indicating the logical \textit{or} of the \textit{expressions}. The type of the expressions must be \integ or \float and must be the same. In the \textit{expressions}, 0 is considered \textsf{false} and all other values are considered \textsf{true}.
\subsubsection{\textit{expression} and \textit{expression}}
The result is a Boolean indicating the logical \textit{and} of the \textit{expressions}. The type considerations are the same as they are for \textit{expression} or \textit{expression}.
\subsubsection{\textit{expression} xor \textit{expression}}
The result is a Boolean indicating the logical \textit{xor} of the \textit{expressions}. The type considerations are the same as they are for \textit{expression} or \textit{expression}.
\subsection{Assignment Operators}
Assignment operators have left associativity
\subsubsection{lvalue $=$ \textit{expression}}
The result is the assignment of the expression to the lvalue. The lvalue must have been previously declared. The type of the expression must be of the same that the lvalue was declared as. Recall, lvalues can be declared as \integ, \float, \complex, and \mat.

\section{Declarations}
Declarations are used within functions to specify how to interpret each identifier. Declarations have the form\\

	\textit{ declaration: }
		\\*\indent\indent\textit{ type-specifier declarator-list}

\subsection{Type Specifiers}

There are four main type specifiers\\

	\textit{type-specifier: }
		\\*\indent\indent\textit{int}
		\\*\indent\indent\textit{float}
		\\*\indent\indent\textit{com}
		\\*\indent\indent\textit{mat}

\subsection{Declarators}
The declarator-list consist of either a single declarator, or a series of declarators separated by commas.\\

	\textit{declarator-list:}
		\\*\indent\indent\textit{declarator}
		\\*\indent\indent\textit{declarator , declarator-list}\\

A declarator refers to an object with a type determined by the type-specifier in the overall declaration. Declarators can have the following form\\

	\textit{declarator:}
		\\*\indent\indent\textit{identifier}
		\\*\indent\indent\textit{declarator ( )}
		\\*\indent\indent\textit{declarator [ constant-expression ]}
		\\*\indent\indent\textit{( declarator )}\\

\subsection{ Meaning of Declarators }
Each declarator that appears in an expression is a call to create an object of the specified type. Each declarator has one identifier, and it is this identifier that is now associated with the created object. 

If declarator D has the form\\
		\\*\indent\indent\textit{D ( )}\\\\
then the contained identifier has the type "function" that is returning an object. This object has the type which the identifier would have had if the declarator had just been D.\\\\
If a declarator has the form\\
		\\*\indent\indent\textit{D[constant-expression]}\\
or
		\\*\indent\indent\textit{D[ ]}\\\\
then it is a declarator whose identifier is of type "array". In the first case, the constant- expression is an expression whose value can be defined at compile time.  The type of that constant-expression is int. In the second case, the constant expression 1 is used. \\

An array can be constructed from one of the basic types, or from another array.\\

Parentheses in declarators do not change the the type of contained identifier, but can affect the relations between the individual components of the declarator.\\

Not all possible combinations of the above syntax are permitted. There are certain restrictions such as how array of functions cannot be declared.

\section{Statements}


\subsection{Expression statements}

Expression statements are the building blocks of an executable program. As the name suggests, expression statements are nothing but expressions, delimited by semicolons. 
Expressions can actually be declarations, assignments, operations or even function calls.
For example,
\begin{verbatim}
x = a + 3;
\end{verbatim}
is a valid expression statement, and so is 
\begin{verbatim}
print(a);
\end{verbatim}

\subsection{The if-elif-else statement}
The \texttt{if-elif-else} statement is used for selectively executing statements based on some condition.Essentially, if the condition following the \texttt{if} keyword is satisfied, the specified statements get executed.To specify what happens if the condition does not evaluate to true, we have the \texttt{else} keyword.
In case we want to evaluate more than one condition at a time, we also have the \texttt{elif} keyword. So an if can be followed by any number of \texttt{elif}s, and at most one else block which is the end of the construct.The statements following the \texttt{else} are executed only if neither of the conditions specified before that evaluate to true.


\begin{verbatim}


	if ( condition) {
	} elif (condition) {
	} else {
	}


Example:
if ( x==5) {
       print("x is 5");
	} elif (x==3) {
	   print("x is 3");
	} else {
	   print("x is neither 5 nor 3")
	}
\end{verbatim}

\subsection{The for loop}

The for statement is used for executing a set of statements a specified number of times.	The statements within the for loop are executed as long as the value of the variable is within the specified range. 
As soon as the value goes out of range, control comes out of the for loop. To ensure termination, each iteration of the for loop increments/decrements the value of the variable, bringing it one step closer to the final value that is to be achieved.

By default, increment or decrement is by 1. However, if the desired increment is something other than one, the by keyword lets you specify that explicitly.

An example of for loop, increment by 2 is as follows: 

\begin{verbatim}

	for k from 1 to 10 by 2 {
	}
\end{verbatim}




  
\section{Scope rules}
Name bindings have a block scope. That is to say, the scope of a name binding
is limited to a section of code that is grouped together. That name can only be
used to refer to associated entity in that block of code. Blocks of code in
QLang are deliminated by the opening curly brace ('\{') at the start of the
block, and the closing curly brace ('\}') at the end of the block. 

Within a program, variables may be declared and/or defined  in various places. The scope of each variable is different, depending on where it is declared.There are three primary scope rules.

If a variable is defined at the outset/outer block of a program, it is visible everywhere in the program.

If a variable is defined as a parameter to a function, or inside a function/block of code, it is visible only within that function.

Declarations made after a specific declaration are not visible to it, or to any declarations before it.

For instance, consider the following snippet.

\begin{lstlisting}

int x = 5;

int y = x + 10;  # this works

int z = a + 100;  # this does not

int a = 200; 
\end{lstlisting}

\section{Constant expressions}
In order to facilitate efficiency in writing expression, the language introduces various mathematical constants such as $\pi$ , $\mathrm{e}$ and matrices such  \emph{Pauli} matrices and \emph{Hadamard} matrices which are frequently used in quantum computation. The keywords \emph{I, X, Y, Z, and  H} are reserved for this expressions.

\[
I =
\begin{bmatrix}
 1 & 0  \\
 0 & 1  
\end{bmatrix}
\qquad
X =
\begin{bmatrix}
 0 & 1  \\
 1 & 0  
\end{bmatrix}
\qquad
Z =
\begin{bmatrix}
 1 & 0  \\
 0 & -1  
\end{bmatrix}
\qquad
Y =
\begin{bmatrix}
 0 & -i  \\
 i & 0  
\end{bmatrix}.
\]

The \emph{Hadamard gate} is defined by the matrix:
\[
H= \frac{1}{\sqrt{2}}\begin{bmatrix}
 1 & 1  \\
 1 & -1
\end{bmatrix}.
\]

\section{Examples}
We present some examples that illustrates the use of Qlang in solving quantum computing problems.

\subsection { Solving Quantum Computation Problem}
\subsubsection{Problem1}
Evaluate the following expressions: a. $(H \otimes X) \ket{00}$ b. $\dotproduct[101,000]$ c. $\bra{01} H \otimes H \ket{01} $
\begin{lstlisting}
	
def pseudo = evaluate (){
		
	# a quit type declaration follows dirac notation
	qub mat0 = |00>;
		
	# Both X and H are constant with type mat and
	# @ corresponds to tensor product.
	mat HX = H @ X;
		
	pseudo = HX * mat0;
}
	
\end{lstlisting}

\subsubsection{Problem 2}
Find the matrix corresponding to the quantum circuit:
\begin{figure}[h!]
\begin{center}
\includegraphics{ref/circuit1}
\end{center}
\caption{ Quantum Circuit implementing series of control gates\label{cir1}}
\end{figure}

\begin{lstlisting}
def circuitMat = findMatrix (){
	
	# all basis qubit in 2 dimension
	qub mat0=|00>;
	qub mat1=|01>;
	qub mat2=|10>;
	qub mat3=|11>;
	
	# controlled not matrix	
	mat CNOT = [1,0,0,0:0,1,0,0:0,0,0,1:0,0,1,0]
	
	#controlled hadmard matrix
	mat HNOT = [1/sqrt(2),0,0,1/sqrt(2):0,1,0,0:1/sqrt(2),0,1,-1/sqrt(2):0,0,0,0]
	#composition of control gates
	mat allGates = CNOT * HNOT * CNOT
	
	# Matrix corresponding to the circuit	
	circuitMat =[allGates*mat0:allGates*mat1:allGates*mat2:allGates*mat3]
		
}
\end{lstlisting}
\subsubsection{Problem 3}
Consider the circuit and show the probabilities of outcome 0 where $\ket{\Psi_{in}} = \ket{1}$
\begin{figure}[h!]
\begin{center}
\includegraphics{ref/circuit2}
\end{center}
\caption{ Quantum Circuit\label{cir1}}
\end{figure}

\begin{lstlisting}
def probability = outcomeZero(){
	
	# top and bottom qubits
	qub top = |0>;
	qub bottom = |1>;
	
	# Applying H on top qubit 
	mat output = (H @ I) * (top @ bottom);
	
	# Controlled Not operator
	mat CNOT = [I, [0,0:0,0]: [0,0:0,0], X];
	
	# Controlled Y operator
	mat CY = [Y,[0,0:0,0]:[0,0:0,0], I];
	
	# Applying Control Operators
	output = (CY)*(CNOT)*output
	
	# Applying measurement operator on top qubit |0> <0|
	mat M = (|0>*<0| @ I)
	
	# state after applying measurement operator on top qubit
	outcome = M * output;
	
	#probability of outcome
	probability = norm(outcome);
	
}
\end{lstlisting}

\subsection{ Simulation of Quantum Algorithm}

\subsubsection{Deutsch Jozsa Algorithm}
\begin{lstlisting}

def outcome = deutschJozsa(qub top, mat U){
		
	# in corresponds to the qubit in top register
	# input is the tensor product of top register and bottom register
	mat input=  top @ |1>;
		
	# application of Hadamard gate on both top and bottom inputs
	input = (H @ H)*input;
		
	# application of U gate on the above result
	input = U * input;
		
	# application of Hadamard gate on the top register
	input = (H @ I)*input;
		
	# application of measurement operator on the top register
	# top * Adj (top) corresponds to the Measurement operator
		
	input=(top*Adj(top)@ I)*input;
		
	#after the measurement is applied, check if the input is 0 or not
	if (input == 0){
		#probability of outcome 0 is 0
		outcome = 0;
	}
	else{
		# probability of outcome 0 is 1
		outcome = 1;
	}
}
	
\end{lstlisting}

\subsubsection {Grover's Search Algorithm}

\begin{figure}[h!]
\begin{center}
\includegraphics{ref/grover}
\end{center}
\caption{ Grover Algorithm Circuit 
\label{fig:grover}}
\end{figure}

\begin{lstlisting}

def result = grover (quit top, int x0){
	# returns the probability to find x0 for a function f such that f(x0)=1
	# x0 can be x0=0,1,?,2^n-1
	# this is a special case where n=1
	
	# qubit in the bottom register
	qub bottom = |1>;
	
	# tensor product of top and bottom qubit
	mat input = top @ bottom;
	
	#application of Hadamard
	input = (H @ H) * input;
	
	#define S
	mat S = [1,0:0-1]
	
	# k : number of time grover operator is applied
	# for n > 1 k=ceil((pi*2^(n/2))/4);
	int k = 1;
	
	# define O operator  such that O|x>|q>=|x>|q mod f(x)> or O|x>=(-1)f(x)|x> 
	# for n > 1 O = I(2^(n1+1));
	mat O = I;
	O(x0+1, x0+1) = -1;
	
	# Grover iteration matrix
	mat GO = (G*O)^k;
	
	# After application of Grover iteration matrix
	mat output = GO * input;
	
	result = (H @ H)* output;
	
}

\end{lstlisting}

\end{document}
