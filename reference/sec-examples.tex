We present some examples that illustrates the use of Qlang in solving quantum computing problems.

\subsection{Deutsch Jozsa Algorithm}
\begin{lstlisting}

	def outcome = deutschjozsa(qubit in, mat U){
		
		qubitInput = in @ |1>;
		input = (H @ H)*input;
		input = U * input;
		input = (H @ I)*input;
		input=(in*Adj(in)@ I)*input;
		
		if (M0 == 0){
			outcome = 0;
		}
		else{
			outcome = 1;
		}
	}
\end{lstlisting}