

\subsection{Expression statements}

Expression statements are the building blocks of an executable program. As the name suggests, expression statements are nothing but expressions, delimited by semicolons. 
Expressions can actually be declarations, assignments, operations or even function calls.
For example,
\begin{verbatim}
x = a + 3;
\end{verbatim}
is a valid expression statement, and so is 
\begin{verbatim}
print(a);
\end{verbatim}

\subsection{The if-elif-else statement}
The \texttt{if-elif-else} statement is used for selectively executing statements based on some condition.Essentially, if the condition following the \texttt{if} keyword is satisfied, the specified statements get executed.To specify what happens if the condition does not evaluate to true, we have the \texttt{else} keyword.
In case we want to evaluate more than one condition at a time, we also have the \texttt{elif} keyword. So an if can be followed by any number of \texttt{elif}s, and at most one else block which is the end of the construct.The statements following the \texttt{else} are executed only if neither of the conditions specified before that evaluate to true.


\begin{verbatim}


	if ( condition) {
	} elif (condition) {
	} else {
	}


Example:
if ( x==5) {
       print("x is 5");
	} elif (x==3) {
	   print("x is 3");
	} else {
	   print("x is neither 5 nor 3")
	}
\end{verbatim}

\subsection{The for loop}

The for statement is used for executing a set of statements a specified number of times.	The statements within the for loop are executed as long as the value of the variable is within the specified range. 
As soon as the value goes out of range, control comes out of the for loop. To ensure termination, each iteration of the for loop increments/decrements the value of the variable, bringing it one step closer to the final value that is to be achieved.

By default, increment or decrement is by 1. However, if the desired increment is something other than one, the by keyword lets you specify that explicitly.

An example of for loop, increment by 2 is as follows: 

\begin{verbatim}

	for k from 1 to 10 by 2 {
	}
\end{verbatim}




  