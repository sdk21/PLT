The \QL language is a scientific tool that enables easy and simple simulation of quantum computing on classical computers. 
Featuring a clear and intuitive syntax, \QL makes it possible to take any quantum algorithm and implement it seamlessly, while 
conserving both the overall structure and syntactical features of the original pseudocode. The \QL code is then compiled to 
C++, allowing for an eventual high-performance execution -- a process made simple and transparent to the user, who can focus on
the algorithmic aspects of the quantum simulation.
\section{Background: Quantum Computing}

In classical computing, data are stored in the form of binary digits or bits. A \emph{bit} is the basic unit of information stored and manipulated in a computer, which in one of two possible distinct states (for instance: two distinct voltages, on and off state of electric switch, two directions of magnetization, etc.). 
The two possible values/states of a system are represented as binary digits, $0$ and $1$. In a quantum computer, however, data are stored in the form of \emph{qubits}, or quantum bits. A quantum system of $n$ qubits is a Hilbert space of dimension $2^n$; fixing any orthonormal basis, any \emph{quantum state} can thus be uniquely written as a linear combination of $2^n$ orthogonal vectors $\{\ket{i}\}_i$ where $i$ is an $n$-bit binary number.

\begin{example}A $3$ qubit system has a canonical basis of 8 orthonormal states denoted 
$\ket{000}$, $\ket{001}$, $\ket{010}$, $\ket{011}$, $\ket{100}$, $\ket{101}$, $\ket{110}$, $\ket{111}$.
\end{example}
To put it briefly, while a classical bit has only two states (either $0$ or $1$), a qubit can have  states $\ket{0}$ and $\ket{1}$, or any linear combination of states also known as a \emph{superposition}: %. Hence, a qubit can have an infinitely many states,
\[
	\ket{\phi}=\alpha\ket{0}+\beta\ket{1}
\]
where $\alpha,\beta\in\C$ are any complex numbers such that $\abs{\alpha}^2+\abs{\beta}^2=1$.\medskip

Similarly, one may recall that logical operations, also known as \emph{logical gates}, are the basis of computation in classical computers. Computers are built with circuit that is made up of logical gates. The examples of logical gates are \textsf{AND}, \textsf{OR}, \textsf{NOT}, \textsf{NOR}, \textsf{XOR}, etc. The analogue for quantum computers, \emph{quantum gates}, are operations which are a \emph{unitary transformation} on qubits. The quantum gates are represented by matrices, and a gate acts on $n$ qubits is represented by $2^n \times 2^n$ unitary matrix\footnote{That is, a matrix $U\in\C^{2^n\times 2^n}$ such that $\adjoint{U}U=I_{2^n}$, where $\adjoint{\cdot}$ denotes the Hermitian conjugate.}. Analogous to the classical computer which is built from an electrical circuit containing wires and logic gates, quantum computers are built from  quantum circuits containing ``wires'' and quantum gates to carry out the computation.\medskip

\noindent More on this, as well as the definition of the usual quantum gates, can be found in \autoref{app:quantum:more}.

\subsection {Dirac notation for quantum computation}
In quantum computing, \emph{Dirac notation} is generally used to represent qubits. This notation provides concise and intuitive representation of complex matrix operations.

More precisely, a column vector $\left[ \begin {array} {c} c_1\\ c_2\\ \vdots\\ c_n \end{array} \right]$ is represented as $\ket{\psi}$, also read as ``ket psi''.  In particular, the computational basis states, also know as \emph{pure states} are represented as $\ket{i}$  where  $i$ is a $n$-bit binary number. For example,
\[
\ket{000}=\begin{bmatrix} 1\\ 0\\ 0\\ 0\\ 0\\ 0\\ 0\\ 0 \end{bmatrix},
\ket{001}=\begin{bmatrix} 0\\ 1\\ 0\\ 0\\ 0\\ 0\\ 0\\ 0 \end{bmatrix},
\ket{010}=\begin{bmatrix} 0\\ 0\\ 1\\ 0\\ 0\\ 0\\ 0\\ 0 \end{bmatrix},
\dots,
\ket{101}=\begin{bmatrix} 0\\ 0\\ 0\\ 0\\ 0\\ 1\\ 0\\ 0 \end{bmatrix},
\ket{110}=\begin{bmatrix} 0\\ 0\\ 0\\ 0\\ 0\\ 0\\ 1\\ 0 \end{bmatrix},
\ket{111}=\begin{bmatrix} 0\\ 0\\ 0\\ 0\\ 0\\ 0\\ 0\\ 1 \end{bmatrix}
\]
Similarly, the row vector $\begin{bmatrix} c^\ast_1 & c^\ast_2  & \dots & c^\ast_n & \end{bmatrix}$, which is also complex conjugate transpose of $\ket{\psi}$, is represented as $\bra{\psi}$, also read as ``bra psi''.\medskip

The inner product of vectors $\ket{\varphi}$ and $\ket{\psi}$ is written $\dotprod{ \varphi }{ \psi }$.
The tensor product of vectors $\ket{\varphi}$ and $\ket{\psi}$ is written $\ket{\varphi} \otimes \ket{\psi}$  and more commonly $\ket{\varphi}\ket{\psi}$.
We list below a few other mathematical notions that are relevant in quantum computing:

\begin{itemize}[-]
\item $z^\ast$ (complex conjugate of elements)\\  if $z=a+ib$, then $z^\ast = a - ib$.
\item $A^\ast$ (complex conjugate of matrices)\\ if $A = \twobytwomatrix[1,6i,3i,2+4i]$ then $A^\ast = \twobytwomatrix[1,-6i,-3i,2-4i]$.
%%%%%%%%%%%%%%%%%%%%%% Transpose of A
\item $\transpose{A}$  (transpose of matrix $A$)\\ if $A = \twobytwomatrix[1,6i,3i,2+4i]$ then $\transpose{A} = \twobytwomatrix[1,3i,6i,2+4i]$.
%%%%%%%%%%%%%%%%%%%%%% Adjoint of A
\item $\adjoint{A}$  (Hermitian conjugate (adjoint) of matrix $A$)\\
Defined as $\adjoint{A} = \left(\transpose{A}\right)^\ast$; if $A = \twobytwomatrix[1,6i,3i,2+4i]$ then $A^{\dag} = \twobytwomatrix[1,-3i,-6i,2-4i]$
%%%%%%%%%%%%%%%%%%%%%% Norm
\item $\norm{\ket{\psi}}$ ($\lp[2]$ norm of vector $\ket{\psi}$)\\
$\norm{\ket{\psi}} = \sqrt{\dotproduct[{\psi}, {\psi}]}$. 
 (This is often used to normalize $\ket{\psi}$ into a unit vector $\frac{\ket{\psi}}{\norm{\ket{\psi}}}$.)
%%%%%%%%%%%%%%%%%%%%%% Other

\item $\bra{\varphi}A\ket{\psi}$ (inner product of $\ket{\varphi}$ and $A\ket{\psi}$). \\
Equivalently\footnote{Recall that we work in a complex Hilbert space: the inner product is a sesquilinear form.}, inner product of $A^{\dag}\ket{\varphi}$ and $\ket{\psi}$

\end{itemize}

\subsection{Quantum Algorithms}
A quantum algorithm is an algorithm that, in addition to operations on bits, can apply quantum gates to qubits and measure the outcome, in order to perform a computation or solve a search problem. Inherently, the outcome of such algorithms will be probabilistic: for instance, a quantum algorithm is said to \emph{compute a function $f$ on input $x$} if, for all $x$, the value $f(x)$ it outputs is correct with high probability.
  The representation of a quantum computation process requires an input register, output register and unitary transformation that takes a computational basis states into linear combination of computational basis states. If $x$ represents an $n$ qubit input register and $y$ represents an $m$ qubit output register, then the effect of a unitary transformation $U_f$ on the computational basis $\ket{x}_n\ket{y}_m$ is represented as follows:
	\begin{equation}
	U_f(\ket{x}_n\ket{y}_m)=\ket{x}_n\ket{y\oplus f(x)}_m,
	\end{equation}
	where $f$ is a function that takes an $n$ qubit input register and returns an $m$ qubit output and $\oplus$ represents mod-$2$ bitwise addition.


\section{Related Work}

\todo{Fill the related work here. Other languages, etc.}
\section{Goal and objectives}

\QL has been designed with a handful of key characteristics in mind:
\begin{description}
  \item[Intuitive.] Any student or researcher familiar with quantum computing should be able to transpose and implement their algorithms easily and quickly, without wasting time struggling to udnerstand idiosyncrasies of the language. 
  \item[Specific.] The language has one purpose -- implementing quantum algorithms. Anything that is not related to nor useful for this purpose should not be -- and is not -- part of \QL (e.g., the language does not support strings).
  \item[Simple.] Matrices, vector operations are pervasive in quantum computing -- thus, they must be easy to use and understand. All predefined structures and functions are straightforward to use, and have no puzzling nor counter-intuitive behavior.
\end{description}
In a nutshell, \QL is simple, includes everything it should -- and nothing it should not.
