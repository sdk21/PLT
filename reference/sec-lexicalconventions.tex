There are five kinds of tokens: identifiers, keywords, constants, expression operators, and other separators. There are six kinds of tokens: identifiers, keywords, constants, strings, expression operators, and other separators. If the input stream has been parsed into tokens up to a given character, the next token is taken to include the longest string of characters which could possibly constitute a token.\cmargin{Rephrase: that's plagiarism}

\subsection{Character set}
\QL supports a subset of ASCII; that is, allowed characters are
\fbox{\texttt{a-zA-Z0-9@\#,-\_;:()[]\{\}<>=+/|*}}, as well as tabulations \texttt{\textbackslash{}t}, spaces, and line returns \texttt{\textbackslash{}n} and {\textbackslash{}r}.
\subsection{Comments}
Comments start with a \# sign, which then extends until the next carriage return. Multiline comments are not supported.

\subsection{Identifier (names)}
An identifier is an arbitrarily long sequence of alphabetic and numeric characters, where \texttt{\_} is included as ``alphabetic''. It must start with a lowercase or uppercase letter, i.e. one of $\texttt{a-zA-Z}$.

\noindent The language is case-sensitive: \texttt{hullabaloo} and \texttt{hullABaLoo} are considered as different.

\subsection{Keywords}
The following identifiers as reserved for keywords, and no one shall use them because it's forbidden and uncool.
\begin{verbatim}
pi e
int float comp rvect cvect mat
true false
if elif else
def for from to by while break
or and xor
not re im norm isunit trans det adj conj sin cos tan
\end{verbatim}
\subsection{Constants}
There are three sorts of constants in the language, namely \emph{integer}, \emph{complex} and \emph{identifier} constants. The first are comprised of any sequence of integers of the form \texttt{0|([1-9][0-9]*)} (recall that integers are non-negative), and have type \integ. The second are of type \complex and have the form 
\texttt{$R$|$R$+$R$i|$R$i}
where $R$ consists of a \textsf{(i)} sign, \textsf{(ii)} an integer part followed by \textsf{(iii)} a point, \textsf{(iv)} a decimal part, then  \textsf{(v)} either a \texttt{e} or a \texttt{E} followed by an exponent part, possibly signed. \textsf{(i)} and \textsf{(v)} are optional, and either \textsf{(ii)} or \textsf{(iv)} can be missing as well. In more detail, $R$ 
is defined as \texttt{[+-]\{0,1\}((($A$.$B$*|.$B$+)([eE][+-]?B+)?)|$A$[eE][+-]?B+)} and $A=$\texttt{0|([1-9]$B$*)}, $B=$\texttt{0|[1-9]} (that is, $R$ matches a real number such as \texttt{2.78e5}, \texttt{1.5E-1} or \texttt{10.25}).\todo{check this paragraph.}

\noindent Finally, the identifier constants are a subset of the reserved keywords, and include:
\begin{description}
  \item[\texttt{e}] the base of natural logarithm $e=\sum_{k=0}^\infty \frac{1}{k!}$. Equivalent to \texttt{exp(1)}; has type \complex.
  \item[\texttt{Pi}] the constant $\pi$. Has type \complex.
  \item[\texttt{true}] represents the Boolean value \textsf{true}. Stored internally  as \integ 1.
  \item[\texttt{false}] represents the Boolean value \textsf{false}. Stored internally  as \integ 0.
\end{description}
