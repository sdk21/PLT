\section {Common quantum gates}

\subsubsection*{Pauli Operators}
	The \emph{Pauli operators} are the special single qubit gates which are represented by the Pauli matrices $\{I, X, Y, Z\}$ as follows
	\[
I =
\begin{bmatrix}
 1 & 0  \\
 0 & 1  
\end{bmatrix}
\qquad
X =
\begin{bmatrix}
 0 & 1  \\
 1 & 0  
\end{bmatrix}
\qquad
Z =
\begin{bmatrix}
 1 & 0  \\
 0 & -1  
\end{bmatrix}
\qquad
Y =
\begin{bmatrix}
 0 & -i  \\
 i & 0  
\end{bmatrix}.
\]
For example, the application of $X$ causes bit-flip in following ways:
\[
X\ket{0}=\begin{bmatrix}
 0 & 1  \\
 1 & 0  
\end{bmatrix}
\begin{bmatrix}
 1   \\
 0   
\end{bmatrix}=\begin{bmatrix}
 0   \\
 1   
\end{bmatrix}= \ket{1}
\]
\[
X\ket{1}=\begin{bmatrix}
 0 & 1  \\
 1 & 0  
\end{bmatrix}
\begin{bmatrix}
 0   \\
 1   
\end{bmatrix}=\begin{bmatrix}
 1   \\
 0   
\end{bmatrix}= \ket{0}.
\]
\subsubsection*{Hadamard Gate}
The \emph{Hadamard gate} is defined by the matrix:
\[
H= \frac{1}{\sqrt{2}}\begin{bmatrix}
 1 & 1  \\
 1 & -1
\end{bmatrix}.
\]
The Hadamard gate maps the computational basis states into superposition of states. The Hadamard gate is significant since it produces maximally entangled states from basis states in the following ways:
\[
H\ket{0}=\frac{1}{\sqrt{2}}(\ket{0}+\ket{1})
\qquad
H\ket{1}=\frac{1}{\sqrt{2}}(\ket{0}-\ket{1}).
\]

\subsubsection*{Controlled-U Gates}
	A \emph{controlled-U gate} is the quantum gate in which the $U$ operator acts on the \new{$n$\textsuperscript{th} $n$-qubit} only if the value of the preceeding qubit is $1$.\\ For example: In a Controlled-\textsf{NOT} gate, the \textsf{NOT} operator flips the second qubit if the first qubit is $1$.
	\[
	\textsf{CNOT} = \begin{bmatrix}
 1&0&0&0\\
0&1&0&0\\
0&0&0&1 \\
0&0&1&0\end{bmatrix}
	\]
	
	\[
	\textsf{CNOT}\ket{00}=\ket{00}
	\]
	\[
	\textsf{CNOT}\ket{01}=\ket{01}
	\]
	\[
	\textsf{CNOT}\ket{10}=\ket{11}
	\]
	\[
	\textsf{CNOT}\ket{11}=\ket{10}.
	\]
	
	
\section {Tensor product and its properties}

Let $A=(a_{i,j})$ be a matrix with respect to the ordered basis $\mathcal{A}=(u_1,\dots,u_n)$ and $B=(b_{i,j})$ be a matrix with respect to the ordered basis $\mathcal{B}=(v_1,\dots,v_m)$. Consider the ordered basis $\mathcal{C}=(u_i \otimes v_j)$ ordered by lexicographic order, that is $u_i \otimes v_j \leq u_l \otimes v_k$ if if $i<l$ or $i=l$ and $j<k$. The matrix of $A \otimes B$ with respect to $\mathcal{C}$ is : 
\[
	A \otimes B= 
	\begin{bmatrix}
 	a_{1,1}B & a_{1,2}B & \dots & a_{1,n}B\\
	a_{2,1}B & a_{2,2}B & \dots & a_{2,n}B\\
	\vdots & \vdots & \ddots & \vdots \\
	a_{n,1}B & a_{n,2}B & \dots & a_{n,n}B
	\end{bmatrix} 
\]
		This matrix is called the tensor product of the matrix $A$ with the matrix $B$.
\begin{itemize}
\item $A \otimes B \otimes C =  (A \otimes B ) \otimes C = A \otimes (B \otimes C)$
\item $ a ( \ket{x} \otimes \ket{y}) = a \ket{x} \otimes \ket{y} = \ket{x} \otimes a\ket{y}$
\item $ ( A \otimes B) \cdot (\ket{y}\ket{z}) = A\ket{y} \otimes B\ket{z}$
\item $ ( A \otimes B) \cdot ( C \otimes D) = AC \otimes BD$
\item $ (A \otimes B) ^{H} = A^{H} \otimes B^{H}$
\item If $ A$ and $B$ unitary, $A \otimes B$ is unitary.
\item If $\ket{x}=\ket{x_1} \ket{x_2}$ and $\ket{y}=\ket{y_1}\ket{y_2}$ then $\dotproduct[x,y]=\dotproduct[x_1,y_1] \dotproduct[x_2, y_2]$ 
\end{itemize}
