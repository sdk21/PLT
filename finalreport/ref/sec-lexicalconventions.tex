There are five kinds of tokens in the language, namely
\textsf{(1)} literals, \textsf{(2)} constants, \textsf{(3)} identifiers, \textsf{(4)} keywords, \textsf{(5)} expression operators, and \textsf{(6)} other separators. At a given point in the parsing, the next token is chosen as to include the longest possible string of characters forming a token.
\subsection{Character set}
\QL supports a subset of ASCII; that is, allowed characters are
\fbox{\texttt{a-zA-Z0-9@\#,-\_;:()[]\{\}<>=+/|*}}, as well as tabulations \texttt{\textbackslash{}t}, spaces, and line returns \texttt{\textbackslash{}n} and \texttt{\textbackslash{}r}.
\subsection{Literals}
There are three sorts of literals in the language, namely \emph{integer}, \emph{float}, and \emph{complex}. All three can be negative or positive (negation is achieved by applying the unary negative operator to them). Integers are given by the regular expression \texttt{['0'-'9']+}, floats are given by \texttt{['0'-'9']+ '.' ['0'-'9']*}, and complex are given by \texttt{C($F$)|C($F$+$F$I)|C($F$I)}, where F is any floating point number.
\subsection{Constants}
\noindent There are several built-in numerical constants that can be treated as literals, they include:
\begin{description}
  \item[\texttt{e}] the base of natural logarithm $e=\sum_{k=0}^\infty \frac{1}{k!}$. Equivalent to \texttt{exp(1)}; has type \complex.
\item[\texttt{pi}] the constant $\pi$. Has type \float.\\
\end{description}
\subsection{Identifier (names)}
An identifier is an arbitrarily long sequence of alphabetic and numeric characters, where \texttt{\_} is included as ``alphabetic''. It must start with a lowercase or uppercase letter, i.e. one of $\texttt{a-zA-Z}$.
\noindent The language is case-sensitive: \texttt{hullabaloo} and \texttt{hullABaLoo} are considered as different.
\subsection{Keywords}
The following identifiers are reserved for keywords, using them as function of variable name will result in an error at compilation time.
\verbatiminput{ref/keywords.txt}
\subsection{Expression Operators}
Expression operators are discussed in detail in section 3.4, Expressions.
\subsection{Seperators}
Commas are used to separator lists of actual and formal parameters, colons are used to separate the rows of matrices, semi-colons are used to terminate statements, and the hash-symbol (\#) is used to begin a comment. Comments extends until the next carriage return. Multi-line comments are not supported.
