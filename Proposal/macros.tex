%%% Macros

\def\>{\rangle}
\def\be{\begin{equation}}
\def\ee{\end{equation}}
\newcommand{\ket}[1]{|#1\rangle}
\newcommand{\bra}[1]{\langle #1|}
\newcommand{\inset}{\hspace*{0.5in}}

\newcommand{\COS}[1]{{\cos\left(#1\right)}}
\newcommand{\SIN}[1]{{\sin\left(#1\right)}}
\newcommand{\EXP}[1]{{e^{#1}}}

\newcommand{\Real}{{\sf R\hspace*{-0.9ex}%
    \rule{0.15ex}{1.5ex}\hspace*{0.9ex}}}
\newcommand{\Natural}{{\sf N\hspace*{-1.0ex}%
    \rule{0.15ex}{1.3ex}\hspace*{1.0ex}}}
\newcommand{\Q}{{\sf Q\hspace*{-1.1ex}%
    \rule{0.15ex}{1.5ex}\hspace*{1.1ex}}}
\newcommand{\Complex}{{\sf C\hspace*{-0.9ex}%
    \rule{0.15ex}{1.3ex}\hspace*{0.9ex}}}

\newcommand{\zpo}{\left(\ket{0} + \ket{1}\right)}
\newcommand{\zmo}{\left(\ket{0} - \ket{1}\right)}

\newcommand{\norm}[1]{\parallel #1 \parallel}

%%%% other macros
\def\dotproduct[#1,#2]{
\langle #1 | #2 \rangle
}

%%%% Matrix macros
\def\twoket[#1,#2]{
\left[
\begin{array}{r}
	#1 \\
	#2
\end{array}
\right]
}

\def\twobra[#1,#2]{
\left[ #1, #2 \right]
}

\def\fourket[#1,#2,#3,#4]{
\left[
\begin{array}{r}
	#1 \\
	#2 \\
	#3 \\
	#4
\end{array}
\right]
}

\def\fourbra[#1,#2,#3,#4]{
\left[ #1, #2, #3, #4 \right]
}

\def\eightket[#1,#2,#3,#4,#5,#6,#7,#8]{
\left[
\begin{array}{r}
	#1 \\
	#2 \\
	#3 \\
	#4 \\
	#5 \\
	#6 \\
	#7 \\
	#8
\end{array}
\right]
}


\def\twobytwomatrix[#1,#2,#3,#4]{
\left[
\begin{array}{rr}
	#1 & #2 \\
	#3 & #4
\end{array}
\right]
}


\newcommand{\Xgatematrix}{
\twobytwomatrix[0,1,1,0]
}

\newcommand{\Igatematrix}{
\twobytwomatrix[1,0,0,1]
}

\newcommand{\Zgatematrix}{
\twobytwomatrix[1,0,0,-1]
}

\newcommand{\Ygatematrix}{
\twobytwomatrix[0,-i,i,0]
}

\newcommand{\Tgatematrix}{
\twobytwomatrix[1,0,0,\EXP{i\pi/4}]
}


\newcommand{\itwo}{
{1\over\sqrt{2}}
}

\newcommand{\onehalf}{
{1\over{2}}
}

\newcommand{\eprpair}{
\itwo\left(\ket{00} + \ket{11}\right)
}

\newcommand{\Hgatematrix}{
\itwo \twobytwomatrix[1,1,1,-1]
}

\newcommand{\CNotmatrix}{
\left[
\begin{array}{rrrr}
	1 & 0 & 0 & 0 \\
	0 & 1 & 0 & 0 \\
	0 & 0 & 0 & 1 \\
	0 & 0 & 1 & 0
\end{array}
\right]
}

\newcommand{\horizbar}{\rule{\linewidth}{.5mm}}
\newcommand{\app}[1]{{\sc #1}}
 
\renewcommand{\em}{\it}
 
\newcommand{\BigO}[1]{${\cal O}(#1)$}
\newcommand{\BigOmega}[1]{$\Omega(#1)$}
\newcommand{\BigTheta}[1]{$\Theta(#1)$}
 
\newcommand{\ceiling}[1]{\left\lceil #1 \right\rceil}
\newcommand{\faM}{\lfloor \alpha M \rfloor}
\newcommand{\C}[2]{{#1 \choose #2}}
 
\newcommand{\comment}[1]{}


\newcommand{\ignore}[1]{}
\input{remark}
\remarktrue
%\newcommand{\remark}[1]{#1}


%%%%% SINGLE FIGURE
\def\cfigure[#1,#2,#3]{
\begin{figure}
\vspace*{0.35in}
\begin{center}

\epsfxsize=2.5in\
\epsfbox{#1}\
 
\vspace*{-3mm}\caption[]{#2
} \label{#3}
 
\vspace*{-5mm}
\end{center}
%\horizbar
\vspace*{4mm}
\end{figure}}

%%%%% Single programmable size figure
\def\wpfigure[#1,#2,#3,#4]{
\begin{figure}[ht]
\vspace*{4mm}
\begin{center}

\epsfxsize=#4\
\epsfbox{#1}\

\vspace*{-3mm}\caption[]{#2
} \label{#3}

\vspace*{-5mm}
\end{center}
%\horizbar
\end{figure}}

%%%%% SINGLE WIDE FIGURE
\def\wcfigure[#1,#2,#3]{
\begin{figure*}
\vspace*{4mm}
\begin{center}

\epsfxsize=5in\
\epsfbox{#1}\

\vspace*{-3mm}\caption[]{#2
} \label{#3}

\vspace*{-5mm}
\end{center}
%\horizbar
\end{figure*}}

%%%%% SINGLE WIDE BUT Y-CONSTRAINED FIGURE
\def\wycfigure[#1,#2,#3]{
\begin{figure*}
\vspace*{4mm}
\begin{center}

\epsfysize=3.0in\
\epsfbox{#1}\

\vspace*{-3mm}\caption[]{#2
} \label{#3}

\vspace*{5mm}
\end{center}
%\horizbar
\end{figure*}}

%%%%%% DOUBLE FIGURE
\def\dcfigure[#1,#2,#3,#4,#5,#6]{
{
\begin{figure*}
\vspace*{0.2in}\
\begin{center}
\begin{minipage}{3in}{
\epsfxsize=3in\
\epsfbox{#1}
\vspace*{-3mm}\caption[]{#2} \label{#3} \
}\end{minipage}\hspace*{0.5in}\
\begin{minipage}{3in}{
\epsfxsize=3in\
\epsfbox{#4}
\vspace*{-3mm}\caption[]{#5}\label{#6} \
}\end{minipage}
\end{center}
\vspace*{-0.4in}\
\end{figure*}
}
}

\def\key#1{
\dimen0=\columnwidth
\advance\dimen0 by -1in
\hspace*{0.5in}
\fbox{\parbox[b]{\dimen0}{{\bf Key: \\} \protect #1}}
}
