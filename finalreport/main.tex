\documentclass[11pt]{report}

\def\withcolors{1}
\def\withnotes{1}
\def\withindex{1}
\def\withbiblioconversion{1} % long names for the journals
%%%%%%%%%%%%%%%%%%%%%%%%%%%%%%%%%%%%%%%%%%%%%%%%%%%%%%%%%%%%%%%%%%%%%%%%%%%%%%%%%%%%
\usepackage[T1]{fontenc}
\usepackage[utf8]{inputenc}

%% Eye-candy
\usepackage{lmodern}
\usepackage{xspace}                                     % Smart spacing with \xspace
\usepackage[protrusion=true,expansion=true]{microtype}  % Improve font rendering

% Striking out text
\usepackage[normalem]{ulem}

%% Math
\usepackage{amsfonts,amsmath,amssymb, amsthm, mathtools}
\usepackage{dsfont} % For the indicator symbol

% Algorithm environment
\usepackage{algorithmicx,algpseudocode,algorithm}

% Colors (with names)
\usepackage[usenames,dvipsnames,table]{xcolor}

% Bibliography
%\usepackage[numbers]{natbib}

% Required for the table of results
\usepackage{multirow}
\usepackage{chngpage} % allows for temporary adjustment of side margins

% Graphics
\usepackage{tikz}
\usetikzlibrary{arrows}
\usetikzlibrary{calc,decorations.pathmorphing,patterns}

% References and links
\usepackage[backref,colorlinks,citecolor=blue,bookmarks=true]{hyperref}
\usepackage{aliascnt}
\usepackage[numbered]{bookmark}

% Full pages
\usepackage{fullpage}

% Compressed lists
\usepackage[shortlabels]{enumitem}
  \setitemize{noitemsep,topsep=3pt,parsep=2pt,partopsep=2pt} % Uncomment for compact item lists
  \setenumerate{itemsep=1pt,topsep=2pt,parsep=2pt,partopsep=2pt}
  \setdescription{itemsep=1pt}
  
% Package for todo notes.
\ifnum\withnotes=1
  \usepackage[colorinlistoftodos,textsize=scriptsize]{todonotes}
\fi

\input{preamble}
\input{package-qcircuit} % To draw quantum circuits in LaTeX

\newcommand{\complex}{\textsf{com}\xspace}
\newcommand{\integ}{\textsf{int}\xspace}
\newcommand{\float}{\textsf{float}\xspace}
\newcommand{\mat}{\textsf{mat}\xspace}

\newcommand{\QL}{\textsf{QLang}\xspace}
\title{\QL: Qubit Language\\ \Large(Final Report)}
\author{
  Christopher Campbell
  \and Cl\'ement Canonne
  \and Sankalpa Khadka
  \and Winnie Narang
  \and Jonathan Wong
}

\begin{document}

\maketitle
\tableofcontents

\ifnum\withnotes=1
  \listoftodos
\fi

\chapter{An Introduction to the Language}
  The \QL language is a scientific tool that enables easy and simple simulation of quantum computing on classical computers. 
Featuring a clear and intuitive syntax, \QL makes it possible to take any quantum algorithm and implement it seamlessly, while 
conserving both the overall structure and syntactical features of the original pseudocode. The \QL code is then compiled to 
C++, allowing for an eventual high-performance execution -- a process made simple and transparent to the user, who can focus on
the algorithmic aspects of the quantum simulation.
\section{Background: Quantum Computing}

In classical computing, data are stored in the form of binary digits or bits. A \emph{bit} is the basic unit of information stored and manipulated in a computer, which in one of two possible distinct states (for instance: two distinct voltages, on and off state of electric switch, two directions of magnetization, etc.). 
The two possible values/states of a system are represented as binary digits, $0$ and $1$. In a quantum computer, however, data are stored in the form of \emph{qubits}, or quantum bits. A quantum system of $n$ qubits is a Hilbert space of dimension $2^n$; fixing any orthonormal basis, any \emph{quantum state} can thus be uniquely written as a linear combination of $2^n$ orthogonal vectors $\{\ket{i}\}_i$ where $i$ is an $n$-bit binary number.

\begin{example}A $3$ qubit system has a canonical basis of 8 orthonormal states denoted 
$\ket{000}$, $\ket{001}$, $\ket{010}$, $\ket{011}$, $\ket{100}$, $\ket{101}$, $\ket{110}$, $\ket{111}$.
\end{example}
To put it briefly, while a classical bit has only two states (either $0$ or $1$), a qubit can have  states $\ket{0}$ and $\ket{1}$, or any linear combination of states also known as a \emph{superposition}: %. Hence, a qubit can have an infinitely many states,
\[
	\ket{\phi}=\alpha\ket{0}+\beta\ket{1}
\]
where $\alpha,\beta\in\C$ are any complex numbers such that $\abs{\alpha}^2+\abs{\beta}^2=1$.\medskip

Similarly, one may recall that logical operations, also known as \emph{logical gates}, are the basis of computation in classical computers. Computers are built with circuit that is made up of logical gates. The examples of logical gates are \textsf{AND}, \textsf{OR}, \textsf{NOT}, \textsf{NOR}, \textsf{XOR}, etc. The analogue for quantum computers, \emph{quantum gates}, are operations which are a \emph{unitary transformation} on qubits. The quantum gates are represented by matrices, and a gate acts on $n$ qubits is represented by $2^n \times 2^n$ unitary matrix\footnote{That is, a matrix $U\in\C^{2^n\times 2^n}$ such that $\adjoint{U}U=I_{2^n}$, where $\adjoint{\cdot}$ denotes the Hermitian conjugate.}. Analogous to the classical computer which is built from an electrical circuit containing wires and logic gates, quantum computers are built from  quantum circuits containing ``wires'' and quantum gates to carry out the computation.\medskip

\noindent More on this, as well as the definition of the usual quantum gates, can be found in \autoref{app:quantum:more}.

\subsection {Dirac notation for quantum computation}
In quantum computing, \emph{Dirac notation} is generally used to represent qubits. This notation provides concise and intuitive representation of complex matrix operations.

More precisely, a column vector $\left[ \begin {array} {c} c_1\\ c_2\\ \vdots\\ c_n \end{array} \right]$ is represented as $\ket{\psi}$, also read as ``ket psi''.  In particular, the computational basis states, also know as \emph{pure states} are represented as $\ket{i}$  where  $i$ is a $n$-bit binary number. For example,
\[
\ket{000}=\begin{bmatrix} 1\\ 0\\ 0\\ 0\\ 0\\ 0\\ 0\\ 0 \end{bmatrix},
\ket{001}=\begin{bmatrix} 0\\ 1\\ 0\\ 0\\ 0\\ 0\\ 0\\ 0 \end{bmatrix},
\ket{010}=\begin{bmatrix} 0\\ 0\\ 1\\ 0\\ 0\\ 0\\ 0\\ 0 \end{bmatrix},
\dots,
\ket{101}=\begin{bmatrix} 0\\ 0\\ 0\\ 0\\ 0\\ 1\\ 0\\ 0 \end{bmatrix},
\ket{110}=\begin{bmatrix} 0\\ 0\\ 0\\ 0\\ 0\\ 0\\ 1\\ 0 \end{bmatrix},
\ket{111}=\begin{bmatrix} 0\\ 0\\ 0\\ 0\\ 0\\ 0\\ 0\\ 1 \end{bmatrix}
\]
Similarly, the row vector $\begin{bmatrix} c^\ast_1 & c^\ast_2  & \dots & c^\ast_n & \end{bmatrix}$, which is also complex conjugate transpose of $\ket{\psi}$, is represented as $\bra{\psi}$, also read as ``bra psi''.\medskip

The inner product of vectors $\ket{\varphi}$ and $\ket{\psi}$ is written $\dotprod{ \varphi }{ \psi }$.
The tensor product of vectors $\ket{\varphi}$ and $\ket{\psi}$ is written $\ket{\varphi} \otimes \ket{\psi}$  and more commonly $\ket{\varphi}\ket{\psi}$.
We list below a few other mathematical notions that are relevant in quantum computing:

\begin{itemize}[-]
\item $z^\ast$ (complex conjugate of elements)\\  if $z=a+ib$, then $z^\ast = a - ib$.
\item $A^\ast$ (complex conjugate of matrices)\\ if $A = \twobytwomatrix[1,6i,3i,2+4i]$ then $A^\ast = \twobytwomatrix[1,-6i,-3i,2-4i]$.
%%%%%%%%%%%%%%%%%%%%%% Transpose of A
\item $\transpose{A}$  (transpose of matrix $A$)\\ if $A = \twobytwomatrix[1,6i,3i,2+4i]$ then $\transpose{A} = \twobytwomatrix[1,3i,6i,2+4i]$.
%%%%%%%%%%%%%%%%%%%%%% Adjoint of A
\item $\adjoint{A}$  (Hermitian conjugate (adjoint) of matrix $A$)\\
Defined as $\adjoint{A} = \left(\transpose{A}\right)^\ast$; if $A = \twobytwomatrix[1,6i,3i,2+4i]$ then $A^{\dag} = \twobytwomatrix[1,-3i,-6i,2-4i]$
%%%%%%%%%%%%%%%%%%%%%% Norm
\item $\norm{\ket{\psi}}$ ($\lp[2]$ norm of vector $\ket{\psi}$)\\
$\norm{\ket{\psi}} = \sqrt{\dotproduct[{\psi}, {\psi}]}$. 
 (This is often used to normalize $\ket{\psi}$ into a unit vector $\frac{\ket{\psi}}{\norm{\ket{\psi}}}$.)
%%%%%%%%%%%%%%%%%%%%%% Other

\item $\bra{\varphi}A\ket{\psi}$ (inner product of $\ket{\varphi}$ and $A\ket{\psi}$). \\
Equivalently\footnote{Recall that we work in a complex Hilbert space: the inner product is a sesquilinear form.}, inner product of $A^{\dag}\ket{\varphi}$ and $\ket{\psi}$

\end{itemize}

\subsection{Quantum Algorithms}
A quantum algorithm is an algorithm that, in addition to operations on bits, can apply quantum gates to qubits and measure the outcome, in order to perform a computation or solve a search problem. Inherently, the outcome of such algorithms will be probabilistic: for instance, a quantum algorithm is said to \emph{compute a function $f$ on input $x$} if, for all $x$, the value $f(x)$ it outputs is correct with high probability.
  The representation of a quantum computation process requires an input register, output register and unitary transformation that takes a computational basis states into linear combination of computational basis states. If $x$ represents an $n$ qubit input register and $y$ represents an $m$ qubit output register, then the effect of a unitary transformation $U_f$ on the computational basis $\ket{x}_n\ket{y}_m$ is represented as follows:
	\begin{equation}
	U_f(\ket{x}_n\ket{y}_m)=\ket{x}_n\ket{y\oplus f(x)}_m,
	\end{equation}
	where $f$ is a function that takes an $n$ qubit input register and returns an $m$ qubit output and $\oplus$ represents mod-$2$ bitwise addition.


\section{Related Work}

\todo{Fill the related work here. Other languages, etc.}
\section{Goal and objectives}

\QL has been designed with a handful of key characteristics in mind:
\begin{description}
  \item[Intuitive.] Any student or researcher familiar with quantum computing should be able to transpose and implement their algorithms easily and quickly, without wasting time struggling to udnerstand idiosyncrasies of the language. 
  \item[Specific.] The language has one purpose -- implementing quantum algorithms. Anything that is not related to nor useful for this purpose should not be -- and is not -- part of \QL (e.g., the language does not support strings).
  \item[Simple.] Matrices, vector operations are pervasive in quantum computing -- thus, they must be easy to use and understand. All predefined structures and functions are straightforward to use, and have no puzzling nor counter-intuitive behavior.
\end{description}
In a nutshell, \QL is simple, includes everything it should -- and nothing it should not.

\chapter{\QL in practice: a Tutorial}
  To illustrate and describe the process of writing in \QL, this section will walk the reader through the implementation of one of the most emblematic quantum examples, namely \emph{Deutsch-Jozsa Algorithm}. The goal of this algorithm is to answer the following question: given query access to an unknown fucntion $f\colon\{0,1\}^n \to \{0,1\}$, promised to be either constant or balanced\footnote{$f$ is said to be balanced if $f(x)=0$ for exactly half of the inputs $x\in\{0,1\}^n$; or, equivalently, if $\shortexpect_{x}[f(x)] = \frac{1}{2}$.}, which of the two holds?  Classically, it is easy to see that this requires (in the worst case) querying just over half the solution space, that is $2^{n-1} + 1$ queries.  Quantumly, the Deutsch-Jozsa algorithm enables us to answer this question with just \emph{one} query!

\begin{align*}
 \Qcircuit @C=1em @R=.7em {
  \lstick{\ket{0}} & /^n \qw & \gate{H^{\otimes n}} & \multigate{1}{U_f} & \gate{H^{\otimes n}}	& \meter & \cw \\
  \lstick{\ket{1}} & \qw     & \gate{H}             & \ghost{U_f}        & \qw
 }
\end{align*}

\chapter{Reference Manual}\label{sec:reference}
  
\section{Lexical conventions}
There are five kinds of tokens: identifiers, keywords, constants, expression operators, and other separators. There are six kinds of tokens: identifiers, keywords, constants, strings, expression operators, and other separators. If the input stream has been parsed into tokens up to a given character, the next token is taken to include the longest string of characters which could possibly constitute a token.\cmargin{Rephrase: that's plagiarism}

\subsection{Character set}
\QL supports a subset of ASCII; that is, allowed characters are
\fbox{\texttt{a-zA-Z0-9@\#,-\_;:()[]\{\}<>=+/|*}}, as well as tabulations \texttt{\textbackslash{}t}, spaces, and line returns \texttt{\textbackslash{}n} and {\textbackslash{}r}.
\subsection{Comments}
Comments start with a \# sign, which then extends until the next carriage return. Multiline comments are not supported.

\subsection{Identifier (names)}
An identifier is an arbitrarily long sequence of alphabetic and numeric characters, where \texttt{\_} is included as ``alphabetic''. It must start with a lowercase or uppercase letter, i.e. one of $\texttt{a-zA-Z}$.

\noindent The language is case-sensitive: \texttt{hullabaloo} and \texttt{hullABaLoo} are considered as different.

\subsection{Keywords}
The following identifiers as reserved for keywords, and no one shall use them because it's forbidden and uncool.
\begin{multicols}{2}
\verbatiminput{keywords.txt}
\end{multicols}

\subsection{Constants}
There are three sorts of constants in the language, namely \emph{integer}, \emph{complex} and \emph{identifier} constants. The first are comprised of any sequence of integers of the form \texttt{0|([1-9][0-9]*)} (recall that integers are non-negative), and have type \integ. The second are of type \complex and have the form 
\texttt{$R$|$R$+$R$i|$R$i}
where $R$ consists of a \textsf{(i)} sign, \textsf{(ii)} an integer part followed by \textsf{(iii)} a point, \textsf{(iv)} a decimal part, then  \textsf{(v)} either a \texttt{e} or a \texttt{E} followed by an exponent part, possibly signed. \textsf{(i)} and \textsf{(v)} are optional, and either \textsf{(ii)} or \textsf{(iv)} can be missing as well. In more detail, $R$ 
is defined as \texttt{[+-]\{0,1\}((($A$.$B$*|.$B$+)([eE][+-]?B+)?)|$A$[eE][+-]?B+)} and $A=$\texttt{0|([1-9]$B$*)}, $B=$\texttt{0|[1-9]} (that is, $R$ matches a real number such as \texttt{2.78e5}, \texttt{1.5E-1} or \texttt{10.25}).\todo{check this paragraph.}

\noindent Finally, the identifier constants are a subset of the reserved keywords, and include:
\begin{description}
  \item[\texttt{e}] the base of natural logarithm $e=\sum_{k=0}^\infty \frac{1}{k!}$. Equivalent to \texttt{exp(1)}; has type \complex.
  \item[\texttt{Pi}] the constant $\pi$. Has type \complex.
  \item[\texttt{true}] represents the Boolean value \textsf{true}. Stored internally  as \integ 1.
  \item[\texttt{false}] represents the Boolean value \textsf{false}. Stored internally  as \integ 0.
\end{description}

An operation, or language elementary unit, starts from the end of the previous one, and ends whenever a semicolon (that is not part of a matrix declaration) is encountered.

\section{Objects and types}
\subsection{Objects and lvalues}
As in C, ``an object is a manipulatable region of storage; an lvalue is an expression referring to an object.''

\subsection{Valid types}

The language features 4 elementary types, namely \integ, \float, \complex, \mat. Is also valid, any type that inductively can be built from a valid type as follows:
\begin{itemize}
  \item elementary types are valid;
  \item an \emph{matrix} of a valid type is valid. Matrices have fixed size (that must be declared at compilation time), and are comprised of any elements of any type (that is, a matrix can have elements of non-necessarily identical types);
  \item a \emph{function} taking as input a fixed number of elements from (non-necessarily identical) valid types, and returning a valid type.
\end{itemize}

\section{Conversions}
Applying unary or binary operators to some values may cause an implicit conversion of their operands. In this section, we list the possible conversions, and their expected result -- any conversion not listed here is impossible, and attempting to force it would generate a compilation error.

\begin{itemize}
  \item $\complex\to\float$: the imaginary part of the complex number is dropped (will generate a warning).
  \item $\float\to\integ$: the floating number is rounded towards zero.
  \item $\complex\to\integ$: equivalent to $\complex\to\float\to\integ$.
\end{itemize}


\section{Expressions}
\subsection{Operator Precedence}
\begin{flushleft}
\begin{tabular}{|l|l|l|}
	\hline    
	                   
	Operator Type & 
	Operator & 
	Associativity\\
	
	\hline
	Primary Expressions &
	\textsf{() [] <| |>} &
	Left\\
	Unary & 
	\textsf{re im norm unit trans det adj conj sin cos tan - $|$ not} & 
	Right\\
	Binary & 
	\textsf{\string^ $|$ * / \% $|$ + - $|$ lt gt leq geq $|$ eq neq $|$ and $|$ or xor } &
	Left (except \string^ which is Right)\\
	Assignment &
	\textsf{=} 
	& Right\\
	
	\hline  
\end{tabular}
\end{flushleft}

\subsection{Literals}
Literals are integers, floats, complex numbers, qubits, and matrices, as well as the built-in constants of the language (e.g. \textsf{pi}). Integers are of type \integ, floats are of type \float, complex numbers are of type \complex, qubits and matrices are of type \mat. The built-in constants have pre-determined types described above (e.g. \textsf{pi} is of type \float).

The remaining major subsections of this section describe the groups of \textit{expression} operators, while the minor subsections describe the individual operators within a group.
\subsection{Primary Expressions}
\subsubsection{identifier}
Identifiers are primary expression. All identifiers have an associated type that is given to them upon declaration (e.g. \float \textit{ident} declares an identifier named ident that is of type \float).
\subsubsection{literals}
Literals are primary expression. They are described above.

\subsubsection{(\textit{expression})}
Parenthesized expressions are primary expressions. The type and value of a parenthesized expression is the same as the type and value of the expression without parenthesis. Parentheses allow expressions to be evaluated in a desired precedence. Parenthesized expressions are evaluated relative to each other starting with the expression that is nested the most deeply and ending with the expression that is nested the least deeply (i.e. the shallowest).

\subsubsection{\textit{primary-expression}(\textit{expression-list})}
Primary expressions followed by a parenthesized expression list are primary expressions. Such primary expressions can be used in the declaration of functions or function calls. The expression list must consist of one or more expressions separated by commas. If being used in function declarations, they must be preceded by the correct function declaration syntax and each \textit{expression} in the expression list must evaluate to a type followed by an identifier. If being used in function calls each \textit{expression} in the expression list must evaluate to an identifier.

\subsubsection{\textit{primary-expression}[\textit{expression-list}]}
Primary expressions followed by a bracketed expression list are primary expressions. Such primary expressions can be used in the declaration of matrices and arrays or to access an element of a matrix or array. The expression list must consist of one (for matrices and arrays) or two (for matrices) expressions separated by commas, and must evaluate to \integ.

\subsubsection{[\textit{expression}-elementlist]}
Expression element lists in brackets are primary expressions. Such primary expressions are used to define matrices and therefore are of type \mat. The expression element list must consist of one or more expressions separated by commas or parenthsized. Commas separate expressions into matrix columns and parentheses group expressions into matrix rows. The expressions must evaluate to the same type and can be of type \integ, \float, \complex, or \mat. Additionally, the number of expressions in each row of the matrix must be the same. An example matrix is shown below.

\begin{lstlisting}
int a = 3;
int b = 12;
mat my_matrix = [ (0+1, 2, a)( 5-1, 2*3-1, 12/2)];
\end{lstlisting}

\subsubsection{<\textit{expression}|}
Expressions with a less than sign on the left and a bar on the right are primary expression. Such expressions are used to define qubits and therefore are of type \mat. The notation is meant to mimic the "bra-" of "bra-ket" notation and can therefore be thought of as a row vector representation of the given qubit. Following "bra-ket" notation, the expression must evaluate  to an integer literal of only 0's and 1's, which represents the state of the qubit. An example "bra-" qubit is shown below.

\begin{lstlisting}
mat b_qubit = <0100|;
\end{lstlisting}

\subsubsection{|\textit{expression}>}
Expressions with a bar on the left and a greater than sign on the right are primary expression. All of the considerations are the same as for <\textit{expression}|, except that this notation mimics the "ket" of "bra-ket" notation and can therefore be though of as a column vector representation of the given qubit. An example "ket-" qubit is shown below.

\begin{lstlisting}
int a = 001;
mat k_qubit = |a>;
\end{lstlisting}

\subsection{Unary Operators}
\subsubsection{not \textit{expression}}
The result is a Boolean indicating the logical \textsf{not} of the \textit{expression}. The type of the expression must be \integ or \float. In the \textit{expressions}, 0 is considered false and all other values are considered true.
\subsubsection{re \textit{expression}}
The result is the real component of the \textit{expression}. The type of the expression must be  \complex. The result has the same type as the expression (it is a complex number with  0 imaginary component).
\subsubsection{im \textit{expression}}
The result is the imaginary component of the \textit{expression}. The type of the expression must be  \complex. The result has the same type as the expression (it is a complex number with  0 real component).
\subsubsection{norm \textit{expression}}
The result is the norm of the \textit{expression}. The type of the expression must be \mat, \complex, or \float. The result has type \float, and corresponds to the $2$-norm; in the case of \complex or \float, this coincides with respectively the module and absolute value.
% if the expression is an integer matrix or \float matrix and type \complex if the expression is a complex number matrix.
\subsubsection{unit \textit{expression}}
The result is a Boolean indicating whether the expression is a unit matrix. The type of the expression must be \mat.
\subsubsection{trans \textit{expression}}
The result is the transpose of the \textit{expression}. The type of the expression must be \mat. The result has the same type as the \textit{expression}.
\subsubsection{det \textit{expression}}
The result is the determinant of the \textit{expression}. The type of the expression must be \mat. The result has type \float if the expression is an integer matrix or \float matrix and type \complex if the expression is a complex number matrix.
\subsubsection{adj \textit{expression}}
The result is the adjoint of the \textit{expression}. The type of the expression must be \mat. The result has the same type as the \textit{expression}.
\subsubsection{conj \textit{expression}}
The result is the complex conjugate of the \textit{expression}. The type of the expression must be \complex or \mat. The result has the same type as the \textit{expression}.
\subsubsection{sin \textit{expression}}
The result is the evaluation of the trigonometric function sine on the \textit{expression}. The type of the expression must be \integ, \float, or  \complex. The result has type \float if the expression is of type \integ or \float and type \complex if the expression is of type  \complex.
\subsubsection{cos \textit{expression}}
The result is the evaluation of the trigonometric function cosine on the \textit{expression}. The type of the expression must be \integ, \float, or  \complex. The result has type \float if the expression is of type \integ or \float and type \complex if the expression is of type  \complex.
\subsubsection{tan \textit{expression}}
The result is the evaluation of the trigonometric function tangent on the \textit{expression}. The type of the expression must be \integ, \float, or  \complex. The result has type \float if the expression is of type \integ or \float and type \complex if the expression is of type  \complex. \new{(If an error occured because of a division by zero, a runtime exception is raised.)}
\subsection{Binary Operators}
\subsubsection{\textit{expression} $\hat{}$ \textit{expression}}
The result is the exponentiation of the first \textit{expression} by the second \textit{expression}. The types of the expression must be of type \integ, \float, or  \complex. If the expressions are of the same type, the result has the same type as the \textit{expressions}. Otherwise, if at least one \textit{expression} is a \complex, the result is of type \complex; if neither expressions are comp, but at least one is \float, the result is of type \float.
\subsubsection{\textit{expression} * \textit{expression}}
The result is the product of the \textit{expressions}. The type considerations are the same as they are for \textit{expression} $\hat{}$ \textit{expression} except that it also allows for matrices.
\subsubsection{\textit{expression} / \textit{expression}}
The result is the quotient of the \textit{expressions}, where the first \textit{expression} is the dividend and the second is the divisor. The type considerations are the same as they are for \textit{expression} $\hat{}$ \textit{expression}. Integer division is rounded towards 0 and truncated. \new{(If an error occured because of a division by zero, a runtime exception is raised.)}
\subsubsection{\textit{expression} \% \textit{expression}}
The result is the remainder of the division of the \textit{expressions}, where the first \textit{expression} is the dividend and the second is the divisor. The sign of the dividend and the divisor are ignored, so the result returned is always the remainder of the absolute value (or module) of the dividend divided by the absolute value of the divisor. The type considerations are the same as they are for \textit{expression} $\hat{}$ \textit{expression}.
\subsubsection{\textit{expression} + \textit{expression}}
The result is the sum of the \textit{expressions}. The types of the expressions must be of type \integ, \float, \complex, or \mat. If at least one \textit{expression} is a \complex, the result is of type \complex; if neither expressions are comp, but at least one is \float, the result is of type \float. Qubits and matrices are special and can only be summed with within operands of the same type (and, in the case of matrices, dimensions).
\subsubsection{\textit{expression} - \textit{expression}}
The result is the difference of the first and second \textit{expression}. The type considerations are the same as they are for \textit{expression} + \textit{expression}.
\subsubsection{\textit{expression} @ \textit{expression}}
The result is the tensor product of the first and second \textit{expressions}. The expressions must be of type of \mat. The result has the same type as the \textit{expression}.
\subsubsection{\textit{expression} eq \textit{expression}}
The result is a Boolean indicating if it is true or false that the two \textit{expression} are equivalent. The type of the expressions must either be the same, or one of the two should be implicitly convertible to the other type (e.g., $0.2 \textsf{ eq } 1$, where the right-hand side is an \integ that can be cast into a \float).
\subsubsection{\textit{expression} lt \textit{expression}}
The result is a Boolean indicating if it is true or false that the first \textit{expression} is less than the second. The type of the expressions must be \integ or \float.% and must be the same.  
\subsubsection{\textit{expression} gt \textit{expression}}
The result is a Boolean indicating if it is true or false that the first \textit{expression} is greater than the second. The type of the expressions must be \integ or \float.% and must be the same.  
\subsubsection{\textit{expression} leq \textit{expression}}
The result is a Boolean indicating if it is true or false that the first \textit{expression} is less than  or equal to the second. The type of the expressions must be \integ or \float.% and must be the same.  
\subsubsection{\textit{expression} geq \textit{expression}}
The result is a Boolean indicating if it is true or false that the first \textit{expression} is greater than or equal to the second. The type of the expressions must be \integ or \float.% and must be the same.  
\subsubsection{\textit{expression} or \textit{expression}}
The result is a Boolean indicating the logical \textit{or} of the \textit{expressions}. The type of the expressions must be \integ or \float and must be the same. In the \textit{expressions}, 0 is considered \textsf{false} and all other values are considered \textsf{true}.
\subsubsection{\textit{expression} and \textit{expression}}
The result is a Boolean indicating the logical \textit{and} of the \textit{expressions}. The type considerations are the same as they are for \textit{expression} or \textit{expression}.
\subsubsection{\textit{expression} xor \textit{expression}}
The result is a Boolean indicating the logical \textit{xor} of the \textit{expressions}. The type considerations are the same as they are for \textit{expression} or \textit{expression}.
\subsection{Assignment Operators}
Assignment operators have left associativity
\subsubsection{lvalue $=$ \textit{expression}}
The result is the assignment of the expression to the lvalue. The lvalue must have been previously declared. The type of the expression must be of the same that the lvalue was declared as. Recall, lvalues can be declared as \integ, \float, \complex, and \mat.

\section{Declarations}
Declarations are used within functions to specify how to interpret each identifier. Declarations have the form\\

	\textit{ declaration: }
		\\*\indent\indent\textit{ type-specifier declarator-list}

\subsection{Type Specifiers}

There are four main type specifiers\\

	\textit{type-specifier: }
		\\*\indent\indent\textit{int}
		\\*\indent\indent\textit{float}
		\\*\indent\indent\textit{com}
		\\*\indent\indent\textit{mat}

\subsection{Declarators}
The declarator-list consist of either a single declarator, or a series of declarators separated by commas.\\

	\textit{declarator-list:}
		\\*\indent\indent\textit{declarator}
		\\*\indent\indent\textit{declarator , declarator-list}\\

A declarator refers to an object with a type determined by the type-specifier in the overall declaration. Declarators can have the following form\\

	\textit{declarator:}
		\\*\indent\indent\textit{identifier}
		\\*\indent\indent\textit{declarator ( )}
		\\*\indent\indent\textit{declarator [ constant-expression ]}
		\\*\indent\indent\textit{( declarator )}\\

\subsection{ Meaning of Declarators }
Each declarator that appears in an expression is a call to create an object of the specified type. Each declarator has one identifier, and it is this identifier that is now associated with the created object. 

If declarator D has the form\\
		\\*\indent\indent\textit{D ( )}\\\\
then the contained identifier has the type "function" that is returning an object. This object has the type which the identifier would have had if the declarator had just been D.\\\\
If a declarator has the form\\
		\\*\indent\indent\textit{D[constant-expression]}\\
or
		\\*\indent\indent\textit{D[ ]}\\\\
then it is a declarator whose identifier is of type "array". In the first case, the constant- expression is an expression whose value can be defined at compile time.  The type of that constant-expression is int. In the second case, the constant expression 1 is used. \\

An array can be constructed from one of the basic types, or from another array.\\

Parentheses in declarators do not change the the type of contained identifier, but can affect the relations between the individual components of the declarator.\\

Not all possible combinations of the above syntax are permitted. There are certain restrictions such as how array of functions cannot be declared.

\section{Statements}


\subsection{Expression statements}

Expression statements are the building blocks of an executable program. As the name suggests, expression statements are nothing but expressions, delimited by semicolons. 
Expressions can actually be declarations, assignments, operations or even function calls.
For example,
\begin{lstlisting}

x = a + 3;
\end{lstlisting}
is a valid expression statement, and so is 
\begin{lstlisting}

print(a);
\end{lstlisting}

\subsection{The if-else statement}
The \texttt{if-else} statement is used for selectively executing statements based on some condition.Essentially, if the condition following the \texttt{if} keyword is satisfied, the specified statements get executed.To specify what happens if the condition does not evaluate to true, we have the \texttt{else} keyword.
In case we want to evaluate more than one condition at a time, \texttt{if-else} can be nested.% So an if can be followed by any number of \texttt{elif}s, and at most one else block which is the end of the construct.The statements following the \texttt{else} are executed only if neither of the conditions specified before that evaluate to true.


\begin{lstlisting}


	if( condition ){
	}
	else{
	}


Example:
 if ( x eq 5) {
    print(5);
 } else if (x eq 3) {
 	  print(3);
 } else {
	  print(0);
 }
\end{lstlisting}

\subsection{The for loop}

The \textsf{for} statement is used for executing a set of statements a specified number of times.	The statements within the for loop are executed as long as the value of the variable is within the specified range. 
As soon as the value goes out of range, control comes out of the \textsf{for} loop. To ensure termination, each iteration of the \textsf{for} loop increments/decrements the value of the variable, bringing it one step closer to the final value that is to be achieved.

By default, increment or decrement is by 1. However, if the desired increment is something other than one, the optional keyword \textsf{by} lets you specify that explicitly.

An example of \textsf{for} loop, increment by 2 is as follows: 
\begin{lstlisting}
  int k;
	for(  k from 1 to 10 by 2 ) {
	}
\end{lstlisting}
The two keywords \textsf{break} and \textsf{continue} can be used inside the body of the loop to respectively exit it prematurely, or skip to the next iteration.


\subsection{The while loop}

The  \textsf{while} statement is used for executing a set of statements as long as a predicate (condition) is true.	As soon as the predicate is no longer satisfied, control comes out of the \textsf{while} loop.
An example of \textsf{while} loop is given below:
\begin{lstlisting}
	while(  k leq 100 ) {
	  k = k^2;
	}
\end{lstlisting}
The two keywords \textsf{break} and \textsf{continue} can be used inside the body of the loop to respectively exit it prematurely, or skip to the next iteration.



  

\section{Scope rules}
\input{ref/sec-scope}
\section{Constant expressions}
In order to facilitate efficiency in writing expression, the language introduces various mathematical constants such as $\pi$ , $\mathrm{e}$ and matrices such  \emph{Pauli} matrices and \emph{Hadamard} matrices which are frequently used in quantum computation. The keywords \emph{I, X, Y, Z, and  H} are reserved for this expressions.

\[
I =
\begin{bmatrix}
 1 & 0  \\
 0 & 1  
\end{bmatrix}
\qquad
X =
\begin{bmatrix}
 0 & 1  \\
 1 & 0  
\end{bmatrix}
\qquad
Z =
\begin{bmatrix}
 1 & 0  \\
 0 & -1  
\end{bmatrix}
\qquad
Y =
\begin{bmatrix}
 0 & -i  \\
 i & 0  
\end{bmatrix}.
\]

The \emph{Hadamard gate} is defined by the matrix:
\[
H= \frac{1}{\sqrt{2}}\begin{bmatrix}
 1 & 1  \\
 1 & -1
\end{bmatrix}.
\]

\section{Examples}
We present some examples that illustrates the use of Qlang in solving quantum computing problems.

\subsection { Solving Quantum Computation Problem}
\subsubsection{Problem1}
Evaluate the following expressions: a. $(H \otimes X) \ket{00}$ b. $\dotproduct[101,000]$ c. $\bra{01} H \otimes H \ket{01} $
\begin{lstlisting}
	
def pseudo = evaluate (){
		
	# a quit type declaration follows dirac notation
	qub mat0 = |00>;
		
	# Both X and H are constant with type mat and
	# @ corresponds to tensor product.
	mat HX = H @ X;
		
	pseudo = HX * mat0;
}
	
\end{lstlisting}

\subsubsection{Problem 2}
Find the matrix corresponding to the quantum circuit:
\begin{figure}[h!]
\begin{center}
\includegraphics{ref/circuit1}
\end{center}
\caption{ Quantum Circuit implementing series of control gates\label{cir1}}
\end{figure}

\begin{lstlisting}
def circuitMat = findMatrix (){
	
	# all basis qubit in 2 dimension
	qub mat0=|00>;
	qub mat1=|01>;
	qub mat2=|10>;
	qub mat3=|11>;
	
	# controlled not matrix	
	mat CNOT = [1,0,0,0:0,1,0,0:0,0,0,1:0,0,1,0]
	
	#controlled hadmard matrix
	mat HNOT = [1/sqrt(2),0,0,1/sqrt(2):0,1,0,0:1/sqrt(2),0,1,-1/sqrt(2):0,0,0,0]
	#composition of control gates
	mat allGates = CNOT * HNOT * CNOT
	
	# Matrix corresponding to the circuit	
	circuitMat =[allGates*mat0:allGates*mat1:allGates*mat2:allGates*mat3]
		
}
\end{lstlisting}
\subsubsection{Problem 3}
Consider the circuit and show the probabilities of outcome 0 where $\ket{\Psi_{in}} = \ket{1}$
\begin{figure}[h!]
\begin{center}
\includegraphics{ref/circuit2}
\end{center}
\caption{ Quantum Circuit\label{cir1}}
\end{figure}

\begin{lstlisting}
def probability = outcomeZero(){
	
	# top and bottom qubits
	qub top = |0>;
	qub bottom = |1>;
	
	# Applying H on top qubit 
	mat output = (H @ I) * (top @ bottom);
	
	# Controlled Not operator
	mat CNOT = [I, [0,0:0,0]: [0,0:0,0], X];
	
	# Controlled Y operator
	mat CY = [Y,[0,0:0,0]:[0,0:0,0], I];
	
	# Applying Control Operators
	output = (CY)*(CNOT)*output
	
	# Applying measurement operator on top qubit |0> <0|
	mat M = (|0>*<0| @ I)
	
	# state after applying measurement operator on top qubit
	outcome = M * output;
	
	#probability of outcome
	probability = norm(outcome);
	
}
\end{lstlisting}

\subsection{ Simulation of Quantum Algorithm}

\subsubsection{Deutsch Jozsa Algorithm}
\begin{lstlisting}

def outcome = deutschJozsa(qub top, mat U){
		
	# in corresponds to the qubit in top register
	# input is the tensor product of top register and bottom register
	mat input=  top @ |1>;
		
	# application of Hadamard gate on both top and bottom inputs
	input = (H @ H)*input;
		
	# application of U gate on the above result
	input = U * input;
		
	# application of Hadamard gate on the top register
	input = (H @ I)*input;
		
	# application of measurement operator on the top register
	# top * Adj (top) corresponds to the Measurement operator
		
	input=(top*Adj(top)@ I)*input;
		
	#after the measurement is applied, check if the input is 0 or not
	if (input == 0){
		#probability of outcome 0 is 0
		outcome = 0;
	}
	else{
		# probability of outcome 0 is 1
		outcome = 1;
	}
}
	
\end{lstlisting}

\subsubsection {Grover's Search Algorithm}

\begin{figure}[h!]
\begin{center}
\includegraphics{ref/grover}
\end{center}
\caption{ Grover Algorithm Circuit 
\label{fig:grover}}
\end{figure}

\begin{lstlisting}

def result = grover (quit top, int x0){
	# returns the probability to find x0 for a function f such that f(x0)=1
	# x0 can be x0=0,1,?,2^n-1
	# this is a special case where n=1
	
	# qubit in the bottom register
	qub bottom = |1>;
	
	# tensor product of top and bottom qubit
	mat input = top @ bottom;
	
	#application of Hadamard
	input = (H @ H) * input;
	
	#define S
	mat S = [1,0:0-1]
	
	# k : number of time grover operator is applied
	# for n > 1 k=ceil((pi*2^(n/2))/4);
	int k = 1;
	
	# define O operator  such that O|x>|q>=|x>|q mod f(x)> or O|x>=(-1)f(x)|x> 
	# for n > 1 O = I(2^(n1+1));
	mat O = I;
	O(x0+1, x0+1) = -1;
	
	# Grover iteration matrix
	mat GO = (G*O)^k;
	
	# After application of Grover iteration matrix
	mat output = GO * input;
	
	result = (H @ H)* output;
	
}

\end{lstlisting}


\chapter{Project Plan and Organization}
\chapter{Architectural Design}
\chapter{Test Plan}
\section{Testing Phases}

\subsection{Unit Testing}
Unit testing was done at very frequent intervals, where each unit developed was tested rigorously using
multiple cases. The scanner,parse and the ast were tested in phase 1 and in the later phase the semantic
checker and the code generator were tested.


\subsection{Integration Testing}
In this phase,the various modules were put together and tested incrementally again. Initially the syntax of the
code was tested using multiple test cases and it was also ensured that only the correct syntax is
accepted by running multiple fail test cases. Later on, the semantic checker was integrated and the
semantics of the code were checked to be accurate without violating any of the rules of the language.
In the final phase the code generator was added to the system and was tested to generate the
corresponding C code. 

\subsection{System Testing}
System testing entailed end to end testing of our entire language framework. The input program written in QLang is fed to the compiler and it gives out the final output of the program, having passed through the parsing, scanning, compiling, code generation and execution phases. The final results were piped to an output file where we could see all the outputs.


\section{Automation and Implementation}
A shell script was written in order to automate the test cases at each level, syntax, semantic, code
generation and accurate execution. 
Our file is called runTests.sh, located in the 'test' folder. It takes a folder having QLang program files, and the operation to be done on them as arguments. The outputs of the respective operation can be seen in the corresponding output file. 


The operation options available are :

a : Parsing, scanning and AST generation.
s : SAST generation.
g : Code generation.
c : Generated code is compiled.
e : Generated executable is run, to generate the program's outputs. 

The operations mentioned above are each inclusive of the operations mentioned above them. That means, if you enter the 'g' option, runTests.sh will perform the tasks under 'a','s' and then the operations specific to 'g' as well.

The second argument is the folder that has the input program files. We have acronyms for two folder that are standard to our implementation, the SemanticSuccess and the SemanticFailures. So to run the sast generation on the files in SemanticSuccess folder, we would write :

sh runTests.sh s ss.

The entire code of this script can be seen in the appendix.


\section{Sample test programs}

The effort has been to exhaustively test every kind of execution scenario, in what can be a typical user program. We have created many test files to showcase varied kinds of programs that can be written in QLang, as can be seen in the contents of the SemanticSuccess and SemanticFailures folders.

The rationale is to make sure that syntactically or semantically incorrect programs are not compiled and echo corresponding meaningful error messages to the user, and that correct programs are accepted and executed correctly.

Hence, we have separate test programs to test all kinds of unary and binary operations on all datatypes that our language supports, and also for all kinds of statements and possible combinations of expressions. 
Though the test suite is too large to be included in this section, here are a few sample success and failure cases that showcase different applications of our language :


For instance, break\_continue.ql is a QLang program as follows :

\begin{verbatim}

def func_test(int a) : int ret_name { 
        
        int i;

        for(i from 0 to 2 by 1)
        a=a+5;

        for(i from 2 to 0 by -1)
        {
            a=a*10;
            print(a);
            break;
        }

        for(i from 1 to 5)
        {
            print(a);
            continue;
            a=a*10;

        }

    ret_name = a;
}

def compute(): int trial {

   trial = func_test(20);
} 
\end{verbatim}

It generates break\_continue.cpp as below upon passing it through the code generation code
\begin{verbatim}

#include <iostream>
#include <complex>
#include <cmath>
#include <Eigen/Dense>
#include <qlang>
using namespace Eigen;
using namespace std;
        
int func_test (int a )
{
	int i;
	int ret_name;
 

    for (int i = 0; i < 2; i = i + 1){
        	a = a + 5;

        }
    for (int i = 2; i < 0; i = i +   -1){
        
	{
	a = a * 10;

	cout << a << endl;

break;
	}

        }
    for (int i = 1; i < 5; i = i + 1){
        
	{
	cout << a << endl;

continue;
	a = a * 10;

	}

        }	ret_name = a;

	return ret_name;
}
int main ()
{
	int trial;
 
	trial = func_test(20);

	std::cout << trial << endl;

	return 0;
}
\end{verbatim}


and the generated output of this is :

\begin{verbatim} 
30
30
30
30
30
\end{verbatim}


Another example we consider is  mat\_qubit.ql


\begin{verbatim}
def func_test(mat a, mat b) : mat ret_name { 
        
        ret_name = a*b;

}


def compute(int a):mat trial {
	
	 mat zero;
	 mat one;

	 zero = |0>;
	 one  = |1>;

     trial = func_test(H,zero);
     printq(trial);

     trial = func_test(H,one);
     printq(trial);

}
\end{verbatim}

It generates mat\_qubit.cpp as below :


\begin{verbatim}
#include <iostream>
#include <complex>
#include <cmath>
#include <Eigen/Dense>
#include <qlang>
using namespace Eigen;
using namespace std;
        
MatrixXcf func_test (MatrixXcf a,MatrixXcf b )
{
	MatrixXcf ret_name;
 
	ret_name = a * b;

	return ret_name;
}
int main ()
{
	MatrixXcf zero;
	MatrixXcf one;
	MatrixXcf trial;
 
	zero = genQubit("0",0);
	one = genQubit("1",0);
	trial = func_test(H,zero);
	cout << vectorToBraket(trial) << endl;
	trial = func_test(H,one);
	cout << vectorToBraket(trial) << endl;

	std::cout << trial << endl;

	return 0;
}
\end{verbatim}

and it generates the qubits in the output as well, like :

\begin{verbatim}
(0.707107)|0> + (0.707107)|1>
(0.707107)|0> + (-0.707107)|1>
(0.707107,0)
(-0.707107,0)
\end{verbatim}

One more program we can show here is a demonstration of the capacity of the semantic analyzer to catch incorrect programs. For instance,

The program:

\begin{verbatim}
def func_test1(int z) : int ret_name { 
        int a;
        int b;
        int d;
        a = z;
        ret_name = z;

}
def func_test1(int z) : int ret_name2 { 

        ret_name2 = z;

}
def compute( int a):int trial {
      
      trial = func_test1(4);
}

\end{verbatim}


gives the error :
\begin{verbatim}
Fatal error: exception Analyzer.Except("Invalid function declaration: func_test1 was already declared")
\end{verbatim}


whereas the sample program

\begin{verbatim}
def func_test(float z) : float ret_name { 
        
        float a; 
        a = 5.8;
       
        ret_name = z;  
}
\end{verbatim}


would give the error :
\begin{verbatim}
Fatal error: exception Analyzer.Except("Missing 'compute' function")
\end{verbatim}

More such pass and fail test cases can be found in the appendix and in our project folder.






\chapter{Lesson Learned}
\section{Sankalpa Khadka}
I learned the lesson that

\appendix
\chapter{More on Quantum Computing}\label{app:quantum:more}
  \section {Common quantum gates}

\subsubsection*{Pauli Operators}
	The \emph{Pauli operators} are the special single qubit gates which are represented by the Pauli matrices $\{I, X, Y, Z\}$ as follows
	\[
I =
\begin{bmatrix}
 1 & 0  \\
 0 & 1  
\end{bmatrix}
\qquad
X =
\begin{bmatrix}
 0 & 1  \\
 1 & 0  
\end{bmatrix}
\qquad
Z =
\begin{bmatrix}
 1 & 0  \\
 0 & -1  
\end{bmatrix}
\qquad
Y =
\begin{bmatrix}
 0 & -i  \\
 i & 0  
\end{bmatrix}.
\]
For example, the application of $X$ causes bit-flip in following ways:
\[
X\ket{0}=\begin{bmatrix}
 0 & 1  \\
 1 & 0  
\end{bmatrix}
\begin{bmatrix}
 1   \\
 0   
\end{bmatrix}=\begin{bmatrix}
 0   \\
 1   
\end{bmatrix}= \ket{1}
\]
\[
X\ket{1}=\begin{bmatrix}
 0 & 1  \\
 1 & 0  
\end{bmatrix}
\begin{bmatrix}
 0   \\
 1   
\end{bmatrix}=\begin{bmatrix}
 1   \\
 0   
\end{bmatrix}= \ket{0}.
\]
\subsubsection*{Hadamard Gate}
The \emph{Hadamard gate} is defined by the matrix:
\[
H= \frac{1}{\sqrt{2}}\begin{bmatrix}
 1 & 1  \\
 1 & -1
\end{bmatrix}.
\]
The Hadamard gate maps the computational basis states into superposition of states. The Hadamard gate is significant since it produces maximally entangled states from basis states in the following ways:
\[
H\ket{0}=\frac{1}{\sqrt{2}}(\ket{0}+\ket{1})
\qquad
H\ket{1}=\frac{1}{\sqrt{2}}(\ket{0}-\ket{1}).
\]

\subsubsection*{Controlled-U Gates}
	A \emph{controlled-U gate} is the quantum gate in which the $U$ operator acts on the \new{$n$\textsuperscript{th} $n$-qubit} only if the value of the preceeding qubit is $1$.\\ For example: In a Controlled-\textsf{NOT} gate, the \textsf{NOT} operator flips the second qubit if the first qubit is $1$.
	\[
	\textsf{CNOT} = \begin{bmatrix}
 1&0&0&0\\
0&1&0&0\\
0&0&0&1 \\
0&0&1&0\end{bmatrix}
	\]
	
	\[
	\textsf{CNOT}\ket{00}=\ket{00}
	\]
	\[
	\textsf{CNOT}\ket{01}=\ket{01}
	\]
	\[
	\textsf{CNOT}\ket{10}=\ket{11}
	\]
	\[
	\textsf{CNOT}\ket{11}=\ket{10}.
	\]
	
	
\section {Tensor product and its properties}

Let $A=(a_{i,j})$ be a matrix with respect to the ordered basis $\mathcal{A}=(u_1,\dots,u_n)$ and $B=(b_{i,j})$ be a matrix with respect to the ordered basis $\mathcal{B}=(v_1,\dots,v_m)$. Consider the ordered basis $\mathcal{C}=(u_i \otimes v_j)$ ordered by lexicographic order, that is $u_i \otimes v_j \leq u_l \otimes v_k$ if if $i<l$ or $i=l$ and $j<k$. The matrix of $A \otimes B$ with respect to $\mathcal{C}$ is : 
\[
	A \otimes B= 
	\begin{bmatrix}
 	a_{1,1}B & a_{1,2}B & \dots & a_{1,n}B\\
	a_{2,1}B & a_{2,2}B & \dots & a_{2,n}B\\
	\vdots & \vdots & \ddots & \vdots \\
	a_{n,1}B & a_{n,2}B & \dots & a_{n,n}B
	\end{bmatrix} 
\]
		This matrix is called the tensor product of the matrix $A$ with the matrix $B$.
\begin{itemize}
\item $A \otimes B \otimes C =  (A \otimes B ) \otimes C = A \otimes (B \otimes C)$
\item $ a ( \ket{x} \otimes \ket{y}) = a \ket{x} \otimes \ket{y} = \ket{x} \otimes a\ket{y}$
\item $ ( A \otimes B) \cdot (\ket{y}\ket{z}) = A\ket{y} \otimes B\ket{z}$
\item $ ( A \otimes B) \cdot ( C \otimes D) = AC \otimes BD$
\item $ (A \otimes B) ^{H} = A^{H} \otimes B^{H}$
\item If $ A$ and $B$ unitary, $A \otimes B$ is unitary.
\item If $\ket{x}=\ket{x_1} \ket{x_2}$ and $\ket{y}=\ket{y_1}\ket{y_2}$ then $\dotproduct[x,y]=\dotproduct[x_1,y_1] \dotproduct[x_2, y_2]$ 
\end{itemize}

\chapter{Source Code}

\end{document}
