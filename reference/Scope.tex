

Within a program, variables may be declared and/or defined  in various places. The scope of each variable is different, depending on where it is declared.There are three primary scope rules.

If a variable is defined at the outset/outer block of a program, it is visible everywhere in the program.

If a variable is defined as a parameter to a function, or inside a function/block of code, it is visible only within that function.

Declarations made after a specific declaration are not visible to it, or to any declarations before it.

For instance, consider the following snippet.

\begin{lstlisting}

int x = 5;

int y = x + 10;  # this works

int z = a + 100;  # this does not

int a = 200; 
\end{lstlisting}