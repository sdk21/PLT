\documentclass[11pt]{article}

\def\withcolors{0}
\def\withnotes{0}
\def\withindex{1}
\def\withbiblioconversion{1} % long names for the journals
%%%%%%%%%%%%%%%%%%%%%%%%%%%%%%%%%%%%%%%%%%%%%%%%%%%%%%%%%%%%%%%%%%%%%%%%%%%%%%%%%%%%
\usepackage[T1]{fontenc}
\usepackage[utf8]{inputenc}

%% Eye-candy
\usepackage{lmodern}
\usepackage{xspace}                                     % Smart spacing with \xspace
\usepackage[protrusion=true,expansion=true]{microtype}  % Improve font rendering

% Striking out text
\usepackage[normalem]{ulem}

%% Math
\usepackage{amsfonts,amsmath,amssymb, amsthm, mathtools}
\usepackage{thm-restate}
\usepackage{dsfont} % For the indicator symbol

% Algorithm environment
\usepackage{algorithmicx,algpseudocode,algorithm}

% Colors (with names)
\usepackage[usenames,dvipsnames,table]{xcolor}

% Quotes: \blockquote command
\usepackage{csquotes}

% Relative sizes for text
\usepackage{relsize}

% Bibliography
%\usepackage[numbers]{natbib}

% Required for the table of results
\usepackage{multirow}
\usepackage{chngpage} % allows for temporary adjustment of side margins

% For the commands such as \capitalisewords
\usepackage{mfirstuc}

% Graphics
\usepackage{tikz}
\usetikzlibrary{arrows}
\usetikzlibrary{calc,decorations.pathmorphing,patterns}

% For indexing
\ifnum\withindex=1
  \usepackage{makeidx}
  \usepackage{ifthen}
  \newcommand\indexed[2][]{\ifthenelse{\equal{#1}{}}{#2\index{#2}}{#2\index{#1}}}
  \makeindex %%%% Enable indexing
\fi
%%\usepackage{showidx} % To debug; does not play well with hyperref

% References and links
\usepackage[backref,colorlinks,citecolor=blue,bookmarks=true]{hyperref}
\usepackage{aliascnt}
\usepackage[numbered]{bookmark}

% Full pages
\usepackage{fullpage}

% Compressed lists
\usepackage[shortlabels]{enumitem}
  \setitemize{noitemsep,topsep=3pt,parsep=2pt,partopsep=2pt} % Uncomment for compact item lists
  \setenumerate{itemsep=1pt,topsep=2pt,parsep=2pt,partopsep=2pt}
  \setdescription{itemsep=1pt}
  
% Package for todo notes.
\ifnum\withnotes=1
  \usepackage[colorinlistoftodos,textsize=scriptsize]{todonotes}
\fi

% For testing and draft: to remove afterwards
\usepackage[english]{babel}
\usepackage{blindtext}

\makeatletter
\@ifundefined{theorem}{%
  % Theorems (each with its own style, all same counter). Cf. http://ftp.math.purdue.edu/mirrors/ctan.org/macros/latex/contrib/hyperref/doc/manual.pdf, p.17
  \theoremstyle{plain} %% Style
  	\newtheorem{theorem}{Theorem}[section]
  	\newaliascnt{coro}{theorem}
  	  \newtheorem{corollary}[coro]{Corollary}
  	\aliascntresetthe{coro}
  	\newaliascnt{lem}{theorem}
  		\newtheorem{lemma}[lem]{Lemma}
  	\aliascntresetthe{lem}
  	\newaliascnt{clm}{theorem}
  		\newtheorem{claim}[clm]{Claim}
	\aliascntresetthe{clm}
	\newaliascnt{fact}{theorem}
 	 	\newtheorem{fact}[theorem]{Fact}
	\aliascntresetthe{fact}
  	\newtheorem*{unnumberedfact}{Fact}
  \newaliascnt{prop}{theorem}
  		\newtheorem{proposition}[prop]{Proposition}
	\aliascntresetthe{prop}
	\newaliascnt{conj}{theorem}
  		\newtheorem{conjecture}[conj]{Conjecture}
	\aliascntresetthe{conj}
  \theoremstyle{remark} %% Style
  	\newtheorem{remark}[theorem]{Remark}
  	\newtheorem{question}[theorem]{Question}
  	\newtheorem*{notation}{Notation}
 	 \newtheorem{example}[theorem]{Example}
  \theoremstyle{definition} %% Style
  	\newaliascnt{defn}{theorem}
 		 \newtheorem{definition}[defn]{Definition}
 	 \aliascntresetthe{defn}
}{}
\makeatother
\providecommand*{\lemautorefname}{Lemma} % For \autoref{} to know the name of lemmas
\providecommand*{\clmautorefname}{Claim}
\providecommand*{\propautorefname}{Proposition}
\providecommand*{\coroautorefname}{Corollary}
\providecommand*{\defnautorefname}{Definition}
\newenvironment{proofof}[1]{\begin{proof}[Proof of {#1}]}{\end{proof}}

%% \email{} command
\providecommand{\email}[1]{\href{mailto:#1}{\nolinkurl{#1}\xspace}}

%% Remarks and notes
\ifnum\withcolors=1
  \newcommand{\new}[1]{{\color{red} {#1}}} % new
  \newcommand{\newer}[1]{{\color{blue} {#1}}} % even newer
  \newcommand{\newest}[1]{{\color{orange} {#1}}} % even even newer
  \newcommand{\newerest}[1]{{\color{blue!10!black!40!green} {#1}}} % you get the idea.
  \newcommand{\ccolor}[1]{{\color{RubineRed}#1}} % Clement
\else
  \newcommand{\new}[1]{{{#1}}}
  \newcommand{\newer}[1]{{{#1}}}
  \newcommand{\newest}[1]{{{#1}}}
  \newcommand{\newerest}[1]{{{#1}}}
  \newcommand{\ccolor}[1]{{#1}}
\fi

\ifnum\withnotes=1
  \newcommand{\cnote}[1]{\par\ccolor{\textbf{C: }\sf #1}} % Clement
  \newcommand{\todonote}[2][]{\todo[size=\scriptsize,color=red!40,#1]{#2}}  
	\newcommand{\questionnote}[2][]{\todo[size=\scriptsize,color=blue!30]{#2}}
	\newcommand{\todonotedone}[2][]{\todo[size=\scriptsize,color=green!40]{$\checkmark$ #2}}
	\newcommand{\todonoteinline}[2][]{\todo[inline,size=\scriptsize,color=orange!40,#1]{#2}}  
  \newcommand{\marginnote}[1]{\todo[color=white,linecolor=black]{{#1}}}
\else
  \newcommand{\cnote}[1]{{{#1}}}
  \newcommand{\todonote}[2][]{\ignore{#2}}
	\newcommand{\questionnote}[2][]{\ignore{#2}}
	\newcommand{\todonotedone}[2][]{\ignore{#2}}
	\newcommand{\todonoteinline}[2][]{\ignore{#2}}
  \newcommand{\marginnote}[1]{\ignore{#1}}
\fi
\newcommand{\ignore}[1]{\leavevmode\unskip} % eat unnecessary spaces before
\newcommand{\cmargin}[1]{\marginnote{\ccolor{#1}}} % Clement

% Shortcuts
\newcommand{\eps}{\ensuremath{\varepsilon}\xspace}
\newcommand{\Algo}{\ensuremath{\mathcal{A}}\xspace} % Algorithm A
\newcommand{\Tester}{\ensuremath{\mathcal{T}}\xspace} % Testing algorithm T
\newcommand{\Learner}{\ensuremath{\mathcal{L}}\xspace} % Learning algorithm L
\newcommand{\property}{\ensuremath{\mathcal{P}}\xspace} % Property P
\newcommand{\class}{\ensuremath{\mathcal{C}}\xspace} % Class C
\newcommand{\eqdef}{\stackrel{\rm def}{=}}
\newcommand{\eqlaw}{\stackrel{\mathcal{L}}{=}}
\newcommand{\accept}{\textsf{ACCEPT}\xspace}
\newcommand{\fail}{\textsf{FAIL}\xspace}
\newcommand{\reject}{\textsf{REJECT}\xspace}
\newcommand{\opt}{{\textsc{opt}}\xspace}
\newcommand{\half}{\frac{1}{2}}
\newcommand{\domain}{\ensuremath{\Omega}\xspace} % Domain of a distribution (default notation)
\newcommand{\distribs}[1]{\Delta\!\left(#1\right)} % Domain of a distribution (default notation)
\newcommand{\yes}{{\sf{}yes}\xspace}
\newcommand{\no}{{\sf{}no}\xspace}
\newcommand{\dyes}{{\cal Y}}
\newcommand{\dno}{{\cal N}}

% Complexity
\newcommand{\littleO}[1]{{o\mleft( #1 \mright)}}
\newcommand{\bigO}[1]{{O\mleft( #1 \mright)}}
\newcommand{\bigTheta}[1]{{\Theta\mleft( #1 \mright)}}
\newcommand{\bigOmega}[1]{{\Omega\mleft( #1 \mright)}}
\newcommand{\tildeO}[1]{\tilde{O}\mleft( #1 \mright)}
\newcommand{\tildeTheta}[1]{\operatorname{\tilde{\Theta}}\mleft( #1 \mright)}
\newcommand{\tildeOmega}[1]{\operatorname{\tilde{\Omega}}\mleft( #1 \mright)}
\providecommand{\poly}{\operatorname*{poly}}

% Influence
\newcommand{\totinf}[1][f]{{\mathbf{Inf}[#1]}}
\newcommand{\infl}[2][f]{{\mathbf{Inf}_{#1}(#2)}}
\newcommand{\infldeg}[3][f]{{\mathbf{Inf}_{#1}^{#2}(#3)}}

% Sets and indicators
\newcommand{\setOfSuchThat}[2]{ \left\{\; #1 \;\colon\; #2\; \right\} } 			% sets such as "{ elems | condition }"
\newcommand{\indicSet}[1]{\mathds{1}_{#1}}                                              % indicator function
\newcommand{\indic}[1]{\indicSet{\left\{#1\right\}}}                                             % indicator function
\newcommand{\disjunion}{\amalg}%\coprod, \dotcup...

% Distance
\newcommand{\dtv}{\operatorname{d_{\rm TV}}}
\newcommand{\totalvardist}[2]{{\dtv\!\left({#1, #2}\right)}}
\newcommand{\hellinger}[2]{{\operatorname{d_{\rm{}H}}\!\left({#1, #2}\right)}}
\newcommand{\kolmogorov}[2]{{\operatorname{d_{\rm{}K}}\!\left({#1, #2}\right)}}
\newcommand{\emd}[2]{{\operatorname{d_{\rm{}EMD}}\!\left({#1, #2}\right)}}
\newcommand{\dist}[2]{{\operatorname{dist}\!\left({#1, #2}\right)}}

% Restriction (functions, sequences, etc.)
\newcommand\restr[2]{{% we make the whole thing an ordinary symbol
  \left.\kern-\nulldelimiterspace % automatically resize the bar with \right
  #1 % the function
  \vphantom{\big|} % pretend it's a little taller at normal size
  \right|_{#2} % this is the delimiter
  }}

% Probability
\newcommand{\proba}{\Pr}
\newcommand{\probaOf}[1]{\proba\!\left[\, #1\, \right]}
\newcommand{\probaCond}[2]{\proba\!\left[\, #1 \;\middle\vert\; #2\, \right]}
\newcommand{\probaDistrOf}[2]{\proba_{#1}\left[\, #2\, \right]}

% Support of a distribution/function
\newcommand{\supp}[1]{\operatorname{supp}\!\left(#1\right)}

% Expectation & variance
\newcommand{\expect}[1]{\mathbb{E}\!\left[#1\right]}
\newcommand{\expectCond}[2]{\mathbb{E}\!\left[\, #1 \;\middle\vert\; #2\, \right]}
\newcommand{\shortexpect}{\mathbb{E}}
\newcommand{\var}{\operatorname{Var}}

% Distributions
\newcommand{\uniform}{\ensuremath{\mathcal{U}}}
\newcommand{\uniformOn}[1]{\ensuremath{\uniform\!\left( #1 \right) }}
\newcommand{\geom}[1]{\ensuremath{\operatorname{Geom}\!\left( #1 \right)}}
\newcommand{\bernoulli}[1]{\ensuremath{\operatorname{Bern}\!\left( #1 \right)}}
\newcommand{\bern}[2]{\ensuremath{\operatorname{Bern}^{#1}\!\left( #2 \right)}}
\newcommand{\binomial}[2]{\ensuremath{\operatorname{Bin}\!\left( #1, #2 \right)}}
\newcommand{\poisson}[1]{\ensuremath{\operatorname{Poisson}\!\left( #1 \right) }}
\newcommand{\gaussian}[2]{\ensuremath{ \mathcal{N}\!\left(#1,#2\right) }}
\newcommand{\gaussianpdf}[2]{\ensuremath{ g_{#1,#2}}}
\newcommand{\betadistr}[2]{\ensuremath{ \operatorname{Beta}\!\left( #1, #2 \right) }}

% Norms
\newcommand{\norm}[1]{\lVert#1{\rVert}}
\newcommand{\normone}[1]{{\norm{#1}}_1}
\newcommand{\normtwo}[1]{{\norm{#1}}_2}
\newcommand{\norminf}[1]{{\norm{#1}}_\infty}
\newcommand{\abs}[1]{\left\lvert #1 \right\rvert}
\newcommand{\dabs}[1]{\lvert #1 \rvert}
\newcommand{\dotprod}[2]{ \left\langle #1,\xspace #2 \right\rangle } 			% <a,b>
\newcommand{\ip}[2]{\dotprod{#1}{#2}} 			% shortcut

\newcommand{\vect}[1]{\mathbf{#1}} 			% shortcut

% Ceiling and floor
\newcommand{\clg}[1]{\left\lceil #1 \right\rceil}
\newcommand{\flr}[1]{\left\lfloor #1 \right\rfloor}

% Common sets
\newcommand{\R}{\ensuremath{\mathbb{R}}\xspace}
\newcommand{\C}{\ensuremath{\mathbb{C}}\xspace}
\newcommand{\Q}{\ensuremath{\mathbb{Q}}\xspace}
\newcommand{\Z}{\ensuremath{\mathbb{Z}}\xspace}
\newcommand{\N}{\ensuremath{\mathbb{N}}\xspace}
\newcommand{\cont}[1]{\ensuremath{\mathcal{C}^{#1}}}

% Oracles and variants
\newcommand{\ICOND}{{\sf INTCOND}\xspace}
\newcommand{\EVAL}{{\sf EVAL}\xspace}
\newcommand{\CDFEVAL}{{\sf CEVAL}\xspace}
\newcommand{\STAT}{{\sf STAT}\xspace}
\newcommand{\SAMP}{{\sf SAMP}\xspace}
\newcommand{\COND}{{\sf COND}\xspace}
\newcommand{\PCOND}{{\sf PAIRCOND}\xspace}
\newcommand{\ORACLE}{{\sf ORACLE}\xspace}

%% Terminology
\newcommand{\pdfsamp}{dual\xspace}
\newcommand{\cdfsamp}{cumulative dual\xspace}
\newcommand{\Pdfsamp}{\expandafter\capitalisewords\expandafter{\pdfsamp}}
\newcommand{\Cdfsamp}{\expandafter\capitalisewords\expandafter{\cdfsamp}}

% L_p norms
\newcommand{\lp}[1][1]{\ell_{#1}}

% Convolution
\DeclareMathOperator{\convolution}{\ast}

%% Terminology
\newcommand{\D}{\ensuremath{D}}
\newcommand{\distrD}{\ensuremath{\mathcal{D}}}
\newcommand{\birge}[2][\D]{\Phi_{#2}(#1)}
\newcommand{\iid}{i.i.d.\xspace}

% Sign
\DeclareMathOperator{\sign}{sgn}

%% Roman numerals
\makeatletter
\newcommand{\rom}[1]{\romannumeral #1}
\newcommand{\Rom}[1]{\expandafter\@slowromancap\romannumeral #1@}
\newcommand{\century}[2][th]{\Rom{#2}\textsuperscript{#1}}
\makeatother

% Hyperref and \autoref{} -- names
\renewcommand{\sectionautorefname}{Section} % To have "Section 5" instead of "section 5" with \autoref{}
\renewcommand{\chapterautorefname}{Chapter} % To have "Chapter 5" instead of "chapter 5" with \autoref{}
\renewcommand{\subsectionautorefname}{Section} % To have "Section 5" instead of "subsection 5" with \autoref{}
\renewcommand{\subsubsectionautorefname}{Section} % To have "Section 5" instead of "subsubsection 5" with \autoref{}
\def\algorithmautorefname{Algorithm}

\newcommand{\ket}[1]{|#1\rangle}
\newcommand{\bra}[1]{\langle #1|}
\def\twobytwomatrix[#1,#2,#3,#4]{
\begin{bmatrix}
	#1 & #2 \\
	#3 & #4
\end{bmatrix}
}

\def\dotproduct[#1,#2]{
\langle #1 | #2 \rangle
}
\newcommand{\transpose}[1]{{#1}^{\rm T}} 			% shortcut
\newcommand{\adjoint}[1]{{#1}^{\dagger}} 			% shortcut

\usepackage{multicol}

\newcommand{\complex}{\texttt{com}\xspace}
\newcommand{\integ}{\texttt{int}\xspace}
\newcommand{\float}{\texttt{float}\xspace}
\newcommand{\mat}{\texttt{mat}\xspace}
%\newcommand{\cvect}{\texttt{cvect}\xspace}
%\newcommand{\rvect}{\texttt{rvect}\xspace}
\newcommand{\qubit}{\texttt{qub}\xspace}

\newcommand{\ket}[1]{|#1\rangle}
\newcommand{\bra}[1]{\langle #1|}
\def\dotproduct[#1,#2]{
\langle #1 | #2 \rangle
}

\lstset{ %
  backgroundcolor=\color{white},   % choose the background color; you must add \usepackage{color} or \usepackage{xcolor}
  basicstyle=\small,        % the size of the fonts that are used for the code
  breakatwhitespace=false,         % sets if automatic breaks should only happen at whitespace
  breaklines=true,                 % sets automatic line breaking
  captionpos=b,                    % sets the caption-position to bottom
  commentstyle=\color{OliveGreen},    % comment style
  deletekeywords={...},            % if you want to delete keywords from the given language
  escapeinside={\%*}{*)},          % if you want to add LaTeX within your code
  extendedchars=true,              % lets you use non-ASCII characters; for 8-bits encodings only, does not work with UTF-8
%  frame=single,                    % adds a frame around the code
  keepspaces=true,                 % keeps spaces in text, useful for keeping indentation of code (possibly needs columns=flexible)
  keywordstyle=\color{blue},       % keyword style
  language=sh,                 % the language of the code
  morekeywords={pi,e,int, float, comp, rvect, cvect, mat, qub, true, false,if, elif, else,def, for, from, to, by, while, break,or, and, xor,not, re, im, norm, isunit, trans, det, adj, conj, sin, cos, tan, exp},                     % if you want to add more keywords to the set
  numbers=left,                    % where to put the line-numbers; possible values are (none, left, right)
  numbersep=5pt,                   % how far the line-numbers are from the code
  numberstyle=\tiny\color{Gray}, % the style that is used for the line-numbers
  rulecolor=\color{black},         % if not set, the frame-color may be changed on line-breaks within not-black text (e.g. comments (green here))
  showspaces=false,                % show spaces everywhere adding particular underscores; it overrides 'showstringspaces'
  showstringspaces=false,          % underline spaces within strings only
  showtabs=false,                  % show tabs within strings adding particular underscores
  stepnumber=2,                    % the step between two line-numbers. If it's 1, each line will be numbered
  stringstyle=\color{RedViolet},     % string literal style
  tabsize=2,                       % sets default tabsize to 2 spaces
  title=\lstname                   % show the filename of files included with \lstinputlisting; also try caption instead of title
}

%\def\twobytwomatrix[#1,#2,#3,#4]{
%\begin{bmatrix}
%	#1 & #2 \\
%	#3 & #4
%\end{bmatrix}
%}

\newcommand{\QL}{\textsf{QLang}\xspace}
\title{\QL: Qubit Language\\ \Large(Reference Manual)}
\author{
  Christopher Campbell
  \and Cl\'ement Canonne
  \and Sankalpa Khadka
  \and Winnie Narang
  \and Jonathan Wong
}

\begin{document}

\maketitle
\tableofcontents
\clearpage

\section{Introduction}
\section{Lexical conventions}
There are five kinds of tokens: identifiers, keywords, constants, expression operators, and other separators. There are six kinds of tokens: identifiers, keywords, constants, strings, expression operators, and other separators. If the input stream has been parsed into tokens up to a given character, the next token is taken to include the longest string of characters which could possibly constitute a token.\cmargin{Rephrase: that's plagiarism}

\section{Character set}
\QL supports a subset of ASCII; that is, allowed characters are
\fbox{\texttt{a-zA-Z0-9\#,.-\_;:()[]\{\}<>=+/|*@}}, as well as tabulations \texttt{\textbackslash{}t}, spaces, and line returns \texttt{\textbackslash{}n} and {\textbackslash{}r}.
\section{Comments}
Comments start with a \# sign, which then extends until the next carriage return. Multiline comments are not supported.

\section{Identifier (names)}
An identifier is an arbitrarily long sequence of alphabetic and numeric characters, where \texttt{\_} is included as ``alphabetic''. It must start with a lowercase or uppercase letter, i.e. one of $\texttt{a-zA-Z}$.

\noindent The language is case-sensitive: \texttt{hullabaloo} and \texttt{hullABaLoo} are considered as different.

%\section{Syntax notation}
An operation, or language elementary unit, starts from the end of the previous one, and ends whenever a semicolon (that is not part of a matrix declaration) is encountered.

%\section{What's in a Name?}
\section{Objects and types}
The language comprises 5 core types, namely \integ, \float, \complex, \qubit, \mat. (In particular, column and row vectors are represented respectively as $n\times1$ or $1\times n$ matrices).

\section{Expressions}
\subsection{Operator Precedence}
\begin{flushleft}
\begin{tabular}{|l|l|l|}
	\hline    
	                   
	Operator Type & 
	Operator & 
	Associativity\\
	
	\hline
	Primary Expressions &
	\textsf{() [] <| |>} &
	Left\\
	Unary & 
	\textsf{not re im norm unit trans det adj conj sin cos tan -} & 
	Right\\
	Binary & 
	\textsf{* / \% + - @ eq lt gt leq geq or and xor \string^} &
	Left (except \string^ which is Right)\\
	Assignment &
	\textsf{=} 
	& Right\\
	
	\hline  
\end{tabular}
\end{flushleft}

\subsection{Literals}
Literals are integers, floats, complex numbers, qubits, and matrices, as well as the built-in constants of the language (e.g. \textsf{pi}). Integers are of type \integ, floats are of type \float, complex numbers are of type \complex, qubits are of type \qubit, and matrices are of type \mat. The built-in constants have pre-determined types described above (e.g. \textsf{pi} is of type \float).

The remaining major subsections of this section describe the groups of \textit{expression} operators, while the minor subsections describe the individual operators within a group.
\subsection{Primary Expressions}
\subsubsection{identifier}
Identifiers are primary \textit{expressions}. All identifiers have an associated type that is given to them upon declaration (e.g. \float \textit{ident} declares an identifier named ident that is of type \float).
\subsubsection{literals}
Literals are primary \textit{expressions}. They are described above.
\subsubsection{(\textit{expression})}
Parenthesized \textit{expressions} are primary \textit{expressions}. The type and value of a parenthesized \textit{expression} is the same as the type and value of the \textit{expression} without parenthesis. Parentheses allow \textit{expressions} to be evaluated in a desired precedence. Parenthesized \textit{expressions} are evaluated relative to each other starting with the \textit{expression} that is nested the most deeply and ending with the \textit{expression} that is nested the least deeply (i.e. the shallowest).
\subsubsection{primary-\textit{expression}(\textit{expression}-list)}
Primary \textit{expressions} followed by a parenthesized \textit{expression} list are primary \textit{expressions}. Such primary \textit{expressions} can be used in the declaration of functions or function calls. The \textit{expression} list must consist of one or more \textit{expressions} separated by commas. If being used in function declarations, they must be preceded by the correct function declaration syntax and each \textit{expression} in the \textit{expression} list must evaluate to a type followed by an identifier. If being used in function calls each \textit{expression} in the \textit{expression} list must evaluate to an identifier.
\subsubsection{[\textit{expression}-elementlist]}
Expression element lists in brackets are primary \textit{expressions}. Such primary \textit{expressions} are used to define matrices and therefore are of type \mat. The \textit{expression} element list must consist of one or more \textit{expressions} separated by commas or semi-colons. Commas separate \textit{expressions} into matrix columns and \new{colons} separate \textit{expressions} into matrix rows. The \textit{expressions} must evaluate to the same type and can be of type \integ, \float, \complex, or \mat. Additionally, the number of \textit{expressions} in each row of the matrix must be the same. An example matrix is shown below.

\begin{lstlisting}
int a = 3;
int b = 12;
mat my_matrix = [ 0+1, 2, a: 5-1, 2*3-1, 12/2];
\end{lstlisting}

\subsubsection{<\textit{expression}|}
Expressions with a less than sign on the left and a bar on the right are primary \textit{expressions}. Such \textit{expressions} are used to define qubits and therefore are of type \qubit. The notation is meant to mimic the "bra-" of "bra-ket" notation and can therefore be thought of as a row vector representation of the given qubit. Following "bra-ket" notation, the \textit{expression} must evaluate  to an integer literal of only 0's and 1's, which represents the state of the qubit. An example "bra-" qubit is shown below.

\begin{lstlisting}
qub b_qubit = <0100|;
\end{lstlisting}

\subsubsection{|\textit{expression}>}
Expressions with a bar on the left and a greater than sign on the right are primary \textit{expressions}. All of the considerations are the same as for <\textit{expression}|, except that this notation mimics the "ket" of "bra-ket" notation and can therefore be though of as a column vector representation of the given qubit. An example "ket-" qubit is shown below.

\begin{lstlisting}
int a = 001;
qub k_qubit = |a>;
\end{lstlisting}

\subsection{Unary Operators}
\subsubsection{not \textit{expression}}
The result is a Boolean indicating the logical \textsf{not} of the \textit{expression}. The type of the \textit{expression} must be \integ or \float. In the \textit{expressions}, 0 is considered false and all other values are considered true.
\subsubsection{re \textit{expression}}
The result is the real component of the \textit{expression}. The type of the \textit{expression} must be  \complex. The result has the same type as the \textit{expression} (it is a complex number with  0 imaginary component).
\subsubsection{im \textit{expression}}
The result is the imaginary component of the \textit{expression}. The type of the \textit{expression} must be  \complex. The result has the same type as the \textit{expression} (it is a complex number with  0 real component).
\subsubsection{norm \textit{expression}}
The result is the norm of the \textit{expression}. The type of the \textit{expression} must be \mat, \new{\complex, \qubit or \float}. The result has type \float, and corresponds to the $2$-norm; in the case of \complex or \float, this coincides with respectively the module and absolute value.
% if the \textit{expression} is an integer matrix or \float matrix and type \complex if the \textit{expression} is a complex number matrix.
\subsubsection{isunit \textit{expression}}
The result is a Boolean indicating if it is true or false that the \textit{expression} is a unit matrix. The type of the \textit{expression} must be \mat.
\subsubsection{trans \textit{expression}}
The result is the transpose of the \textit{expression}. The type of the \textit{expression} must be \mat. The result has the same type as the \textit{expression}.
\subsubsection{det \textit{expression}}
The result is the determinant of the \textit{expression}. The type of the \textit{expression} must be \mat. The result has type \float if the \textit{expression} is an integer matrix or \float matrix and type \complex if the \textit{expression} is a complex number matrix.
\subsubsection{adj \textit{expression}}
The result is the \new{adjoint} of the \textit{expression}. The type of the \textit{expression} must be \mat. The result has the same type as the \textit{expression}.
\subsubsection{conj \textit{expression}}
The result is the complex conjugate of the \textit{expression}. The type of the \textit{expression} must be \complex or \mat. The result has the same type as the \textit{expression}.
\subsubsection{sin \textit{expression}}
The result is the evaluation of the trigonometric function sine on the \textit{expression}. The type of the \textit{expression} must be \integ, \float, or  \complex. The result has type \float if the \textit{expression} is of type \integ or \float and type \complex if the \textit{expression} is of type  \complex.
\subsubsection{cos \textit{expression}}
The result is the evaluation of the trigonometric function cosine on the \textit{expression}. The type of the \textit{expression} must be \integ, \float, or  \complex. The result has type \float if the \textit{expression} is of type \integ or \float and type \complex if the \textit{expression} is of type  \complex.
\subsubsection{tan \textit{expression}}
The result is the evaluation of the trigonometric function tangent on the \textit{expression}. The type of the \textit{expression} must be \integ, \float, or  \complex. The result has type \float if the \textit{expression} is of type \integ or \float and type \complex if the \textit{expression} is of type  \complex. \new{(If an error occured because of a division by zero, a runtime exception is raised.)}
\subsection{Binary Operators}
\subsubsection{\textit{expression} $\hat{}$ \textit{expression}}
The result is the exponentiation of the first \textit{expression} by the second \textit{expression}. The types of the \textit{expression} must be of type \integ, \float, or  \complex. If the \textit{expressions} are of the same type, the result has the same type as the \textit{expressions}. Otherwise, if at least one \textit{expression} is a \complex, the result is of type \complex; if neither \textit{expressions} are comp, but at least one is \float, the result is of type \float.
\subsubsection{\textit{expression} * \textit{expression}}
The result is the product of the \textit{expressions}. The type considerations are the same as they are for \textit{expression} $\hat{}$ \textit{expression}
\subsubsection{\textit{expression} / \textit{expression}}
The result is the quotient of the \textit{expressions}, where the first \textit{expression} is the dividend and the second is the divisor. The type considerations are the same as they are for \textit{expression} $\hat{}$ \textit{expression}. Integer division is rounded towards 0 and truncated. \new{(If an error occured because of a division by zero, a runtime exception is raised.)}
\subsubsection{\textit{expression} \% \textit{expression}}
The result is the remainder of the division of the \textit{expressions}, where the first \textit{expression} is the dividend and the second is the divisor. The sign of the dividend and the divisor are ignored, so the result returned is always the remainder of the absolute value (or module) of the dividend divided by the absolute value of the divisor. The type considerations are the same as they are for \textit{expression} $\hat{}$ \textit{expression}.
\subsubsection{\textit{expression} + \textit{expression}}
The result is the sum of the \textit{expressions}. The types of the \textit{expressions} must be of type \integ, \float, \complex, \new{\mat} or \qubit. If at least one \textit{expression} is a \complex, the result is of type \complex; if neither \textit{expressions} are comp, but at least one is \float, the result is of type \float. Qubits and matrices are special and can only be summed with within operands of the same type \new{(and, in the case of matrices, dimensions)}.
\subsubsection{\textit{expression} - \textit{expression}}
The result is the difference of the first and second \textit{expression}. The type considerations are the same as they are for \textit{expression} + \textit{expression}.
\subsubsection{\textit{expression} @ \textit{expression}}
The result is the tensor product of the first and second \textit{expressions}. The \textit{expressions} must be of type of \mat. The result has the same type as the \textit{expression}.
\subsubsection{\textit{expression} eq \textit{expression}}
The result is a Boolean indicating if it is true or false that the two \textit{expression} are structurally equivalent. The type of the \textit{expressions} must be the same.
\subsubsection{\textit{expression} lt \textit{expression}}
The result is a Boolean indicating if it is true or false that the first \textit{expression} is less than the second. The type of the \textit{expressions} must be \integ or \float and must be the same. \cmargin{no ordering for complex numbers}
\subsubsection{\textit{expression} gt \textit{expression}}
The result is a Boolean indicating if it is true or false that the first \textit{expression} is greater than the second. The type of the \textit{expressions} must be \integ or \float and must be the same. \cmargin{no ordering for complex numbers}
\subsubsection{\textit{expression} leq \textit{expression}}
The result is a Boolean indicating if it is true or false that the first \textit{expression} is less than  or equal to the second. The type of the \textit{expressions} must be \integ or \float and must be the same. \cmargin{no ordering for complex numbers}
\subsubsection{\textit{expression} geq \textit{expression}}
The result is a Boolean indicating if it is true or false that the first \textit{expression} is greater than or equal to the second. The type of the \textit{expressions} must be \integ or \float and must be the same. \cmargin{no ordering for complex numbers}
\subsubsection{\textit{expression} or \textit{expression}}
The result is a Boolean indicating the logical \textit{or} of the \textit{expressions}. The type of the \textit{expressions} must be \integ or \float and must be the same. In the \textit{expressions}, 0 is considered \textsf{false} and all other values are considered \textsf{true}.
\subsubsection{\textit{expression} and \textit{expression}}
The result is a Boolean indicating the logical \textit{and} of the \textit{expressions}. The type considerations are the same as they are for \textit{expression} or \textit{expression}.
\subsubsection{\textit{expression} xor \textit{expression}}
The result is a Boolean indicating the logical \textit{xor} of the \textit{expressions}. The type considerations are the same as they are for \textit{expression} or \textit{expression}.
\subsection{Assignment Operators}
Assignment operators have left associativity
\subsubsection{lvalue $=$ \textit{expression}}
The result is the assignment of the \textit{expression} to the lvalue. The lvalue must have been previously declared. The type of the \textit{expression} must be of the same that the lvalue was declared as. Recall, lvalues can be declared as \integ, \float, comp, mat, and qubit.

\section{Declarations}
Declarations are used within functions to specify how to interpret each identifier. Declarations have the form\\

	\textit{ declaration: }
		\\*\indent\indent\textit{ type-specifier declarator-list}

\subsection{Type Specifiers}

There are five main type specifiers:

	\textit{type-specifier: }
\begin{itemize}[~]
  \item \integ
  \item \float
  \item \complex
  \item \mat
\end{itemize}

\subsection{ Declarator List }
The declarator-list consist of either a single declarator, or a series of declarators separated by commas.\\

	\textit{declarator-list:}
		\\*\indent\indent\textit{declarator}
		\\*\indent\indent\textit{declarator , declarator-list}\\

A declarator refers to an object with a type determined by the type-specifier in the overall declaration. Declarators can have the following form\\

	\textit{declarator:}
		\\*\indent\indent\textit{identifier}
		\\*\indent\indent\textit{declarator ( )}
		\\*\indent\indent\textit{declarator [ constant-expression ]}
		\\*\indent\indent\textit{( declarator )}\\

\subsection{ Meaning of Declarators }
Each declarator that appears in an expression is a call to create an object of the specified type. Each declarator has one identifier, and it is this identifier that is now associated with the created object. 

If declarator D has the form\\
		\\*\indent\indent\textit{D ( )}\\\\
then the contained identifier has the type "function" that is returning an object. This object has the type which the identifier would have had if the declarator had just been D.\\\\
If a declarator has the form\\
		\\*\indent\indent\textit{D[constant-expression]}\\
or
		\\*\indent\indent\textit{D[ ]}\\\\
then it is a declarator whose identifier is of type "array". In the first case, the constant- expression is an expression whose value can be defined at compile time.  The type of that constant-expression is int. In the second case, the constant expression 1 is used. \\

An array may be constructed from one of the basic types, or from another array.\\

Parentheses in declarators do not change the the type of contained identifier, but can affect the relations between the individual components of the declarator.\\

Not all possible combinations of the above syntax are permitted. There are certain restrictions such as how array of functions cannot be declared.

\section{Statements}


\subsection{Expression statements}

Expression statements are the building blocks of an executable program. As the name suggests, expression statements are nothing but expressions, delimited by semicolons. 
Expressions can actually be declarations, assignments, operations or even function calls.
For example,
\begin{lstlisting}

x = a + 3;
\end{lstlisting}
is a valid expression statement, and so is 
\begin{lstlisting}

print(a);
\end{lstlisting}

\subsection{The if-elif-else statement}
The \texttt{if-elif-else} statement is used for selectively executing statements based on some condition.Essentially, if the condition following the \texttt{if} keyword is satisfied, the specified statements get executed.To specify what happens if the condition does not evaluate to true, we have the \texttt{else} keyword.
In case we want to evaluate more than one condition at a time, we also have the \texttt{elif} keyword. So an if can be followed by any number of \texttt{elif}s, and at most one else block which is the end of the construct.The statements following the \texttt{else} are executed only if neither of the conditions specified before that evaluate to true.


\begin{lstlisting}


	if ( condition) {
	} elif (condition) {
	} else {
	}


Example:
if ( x==5) {
       print("x is 5");
	} elif (x==3) {
	   print("x is 3");
	} else {
	   print("x is neither 5 nor 3")
	}
\end{lstlisting}

\subsection{The for loop}

The for statement is used for executing a set of statements a specified number of times.	The statements within the for loop are executed as long as the value of the variable is within the specified range. 
As soon as the value goes out of range, control comes out of the for loop. To ensure termination, each iteration of the for loop increments/decrements the value of the variable, bringing it one step closer to the final value that is to be achieved.

By default, increment or decrement is by 1. However, if the desired increment is something other than one, the by keyword lets you specify that explicitly.

An example of for loop, increment by 2 is as follows: 

\begin{lstlisting}

	for k from 1 to 10 by 2 {
	}
\end{lstlisting}




  
\section{Scope rules}


Within a program, variables may be declared and/or defined  in various places. The scope of each variable is different, depending on where it is declared.There are three primary scope rules.

If a variable is defined at the outset/outer block of a program, it is visible everywhere in the program.

If a variable is defined as a parameter to a function, or inside a function/block of code, it is visible only within that function.

Declarations made after a specific declaration are not visible to it, or to any declarations before it.

For instance, consider the following snippet.

\begin{lstlisting}

int x = 5;

int y = x + 10;  # this works

int z = a + 100;  # this does not

int a = 200; 
\end{lstlisting}
\section{Constant expressions}
In order to facilitate efficiency in writing expression, the language introduces various mathematical constants such as $\pi$ , $\mathrm{e}$ and matrices such  \emph{Pauli} matrices and \emph{Hadamard} matrices which are frequently used in quantum computation. The keywords \emph{I, X, Y, Z, and  H} are reserved for this expressions.

\[
I =
\begin{bmatrix}
 1 & 0  \\
 0 & 1  
\end{bmatrix}
\qquad
X =
\begin{bmatrix}
 0 & 1  \\
 1 & 0  
\end{bmatrix}
\qquad
Z =
\begin{bmatrix}
 1 & 0  \\
 0 & -1  
\end{bmatrix}
\qquad
Y =
\begin{bmatrix}
 0 & -i  \\
 i & 0  
\end{bmatrix}.
\]

The \emph{Hadamard gate} is defined by the matrix:
\[
H= \frac{1}{\sqrt{2}}\begin{bmatrix}
 1 & 1  \\
 1 & -1
\end{bmatrix}.
\]
\section{Examples}
We present some examples that illustrates the use of Qlang in solving quantum computing problems.

\subsection{Deutsch Jozsa Algorithm}
\begin{lstlisting}

	def outcome = deutschjozsa(qubit in, mat U){
		
		qubitInput = in @ |1>;
		input = (H @ H)*input;
		input = U * input;
		input = (H @ I)*input;
		input=(in*Adj(in)@ I)*input;
		
		if (M0 == 0){
			outcome = 0;
		}
		else{
			outcome = 1;
		}
	}
\end{lstlisting}
\end{document}
