\newcommand{\startsyn}{\begin{center}\begin{tabular}{l}}
\newcommand{\stopsyn}{\end{tabular}\end{center}}

\section{Introduction}
The \sys{} programming language is a linear algebra manipulation language
specifically targeted for image processing. It provides an efficient way to
express complex image manipulation algorithms through compact matrix
operations. \sys{} programs are first compiled into a ''C`` module which is
further compiled into a machine binary by an existing C compiler. This two-step
process is completely automated by the \sys{} compiler, and by default no C code
is output (this can be changed with compiler arguments - see section~\ref{sec:compiling}).

This language reference is inspired by the C reference manual~\cite{DBLP:KernighanR88}.
It details the syntax of the \sys{} language.

\section{Lexical Conventions}
\label{sec:lex}

\subsection{Tokens}
\label{ssec:tokens}
The tokens in \sys{} are broken down as follows: reserved
keywords, identifiers, constants, control characters, and operators.
The end of a token is defined by the presence of a newline, space,
tab character (whitespace), or other character that
cannot possibly be part of the current token.

\subsection{Comments}
\label{ssec:comments}
Comments are demarcated with an opening /* and closing */, as in C.
Any characters inside the comment boundaries are ignored. Comments
can be nested.

\clearpage % FORMAT HACK!
\subsection{Keywords}
\label{ssec:keywords}
The reserved keywords in \sys{} are:
\begin{center}\begin{tabular}{l l l l}
Image & imgread & Int8 & Uint8\\
Kernel & imgwrite & Int16 & Uint16\\
Calc & Angle & Int32 & Uint32
\end{tabular}\end{center}

\subsection{Identifiers}
\label{ssec:identifiers}
Identifiers are composed of an upper or lower-case letter immediately
followed by any number of additional letters and/or digits. Identifiers
are case sensitive, so ``foo'' and ``Foo'' are different identifiers.
Identifiers cannot be keywords, and cannot start with a digit.

\subsection{Constants}
\label{ssec:constants}
In \sys{} there are 3 types of constants: numeric constants,
calculation constants, and string literals.

\subsubsection{Numeric Constants}
\label{sssec:numericconstants}

\emph{Integers} are repesented by a series of number characters.
% the negative sign is dealt with by the unary '-' operator

Angles are represented by a series of number characters with an
optional period character.

\subsubsection{Calculation Constants}
\label{sssec:calcconstants}
Matrix calculation constants are represented by an opening curly brace, followed
by a series of \emph{numeric-expressions} separated by whitespace or
comma characters. The comma  characters represents the division between
the rows of the matrix. Each row must have the same number of
\emph{numeric-expressions}, but the matrix need not be square.

A calculation constant may also have an optional fraction preceding it,
which indicates that every value in the matrix should be multiplied
by that fraction. The fraction will be expressed as an opening
bracket character, a \emph{numeric-expression} representing the
numerator, a forward-slash character, a \emph{numeric-expression}
representing the denominator, and a closing bracket character.

\startsyn
\texttt{\{} \emph{numeric-expr} \emph{numeric-expr} \ldots \texttt{,} \emph{numeric-expr} \emph{numeric-expr}\ldots \texttt{\}} \\
\texttt{[}\emph{numeric-expr} \texttt{/} \emph{numeric-expr} \texttt{]}\texttt{\{} \emph{numeric-expr} \emph{numeric-expr} \ldots \texttt{,} \emph{numeric-expr} \emph{numeric-expr}\ldots \texttt{\}}
\stopsyn

The following is an example of a calculation constant.
\begin{lstlisting}[language=CLAM,escapechar=\%]
Calc sobelGy := [1 / 9]{1 3 1 , 2 -5 2 , 1 3 1 };
\end{lstlisting}

\subsubsection{String Literals}
\label{sssec:strings}

String constants are demarcated by double quote characters or single
quote characters. Consecutive string constants will be automatically
appended together into a single string constant.
String constants may contain escaped characters. Escaped characters begin with
a backslash, \texttt{\textbackslash}, and are followed by either an octal,
hexadecimal or base-10 integer value. The following escaped characters are also supported:
\startsyn
\texttt{\textbackslash{}n} (newline)\\
\texttt{\textbackslash{}t} (tab)\\
\texttt{\textbackslash{}b} (break)\\
\texttt{\textbackslash{}r} (carriage-return)\\
\stopsyn

\startsyn
\texttt{"}\emph{string-constant}\texttt{"} \\
\texttt{"}\emph{string-constant}\texttt{"} \texttt{"}\emph{string-constant}\texttt{"} \ldots \\
\stopsyn

\section{Meaning of Identifiers}
\label{sec:identmeaning}

\subsection{Basic Types}
\label{ssec:types}
There are three basic types defined by the \sys{} language.
Type identifiers always begin with an upper-case letter followed by a sequence
of zero or more legal identifier characters. The list of built-in types is as follows:
\startsyn
\texttt{Image} \\
\texttt{Calc} \\
\texttt{Kernel}
\stopsyn 

\subsubsection{Atom Types}
\label{sssec:atomtypes}
The \texttt{Calc} type may be further modified
to specify individual element, or ``atom'' types. This specifies the type
of the resulting calculation performed by a \texttt{Calc} object (either CString or Matrix --
see section~\ref{ssec:calc}). When calculation results exceed the bounds of the specified type,
values are clamped (set to the max or min value appropriately).
An \emph{atom-type} identifier is denoted using the \texttt{<} and \texttt{>}
characters immediately following the identifier of the object whose atom type
is being specified:
\startsyn
\texttt{Calc} \emph{identifier}\texttt{<}\emph{atom-type}\texttt{>}
\stopsyn
Legal \emph{atom-type}s are as follows:
\startsyn
Uint8 \\
Uint16 \\
Uint32 \\
Int8 \\
Int16 \\
Int32 \\
Angle
\stopsyn
In the absence of an atom-type specification, the atom-type defaults to Uint8,
which implies a range of integers from 0-255 inclusive. (This is also the atomic-type
for the default \emph{Red}, \emph{Green} and \emph{Blue} channels.)


\section{Objects and Definitions}
\label{sec:objdef}
An \emph{object} in \sys{} is either a named collection of Channels, called an
\texttt{Image}, or a named collection of calculation bases, called a
\texttt{Kernel}. A Channel is a mathematical matrix of numeric values
whose individual components are not directly accessible via \sys{} language
semantics -- Channel values are manipulated via the convolution
operator (see~\ref{ssec:convolutionop}). A calculation basis, known as a
\texttt{Calc}, is either a calculation constant
(see~\ref{sssec:calcconstants}) or a calculation expression (see~\ref{ssec:escapedC}).

\subsection{\texttt{Calc} objects}
\label{ssec:calc}
A  \texttt{Calc} object is an immutable object initialized with the \texttt{:=} assignment operator
(see section~\ref{sssec:colonequalop}). Its assigned value is either a CString (section~\ref{ssec:escapedC}),
or a calculation constant (\emph{matrix} - section~\ref{sssec:calcconstants}). The calculation
described by a \texttt{Calc} object will be performed for each pixel in an \texttt{Image} Channel.
The calculation is performed using either the convolution operator (section~\ref{sssec:convolutionop}),
or the channel composition operator (section~\ref{sssec:barequalop}).

\subsection{\texttt{Image} objects}
\label{ssec:images}
An \texttt{Image} is a collection of named Channels. Channels can
be dynamically added \comment{or removed} using the channel composition
operator (see section~\ref{sssec:barequalop}, or by assigning to a previously
undeclared Channel name. 

For example, to create a gray-scale image from a single, pre-existing
Channel:
\begin{lstlisting}[language=CLAM,escapechar=\%]
Image outImg;
outImg:Red = calcImg:G;
outImg:Green = calcImg:G;
outImg:Blue = calcImg:G;
\end{lstlisting}

\subsection{\texttt{Kernel} objects}
\label{ssec:kernels}
A \texttt{Kernel} is an ordered collection of calculation bases which is used by the convolution
operator (see section~\ref{ssec:convolutionop}). Each calculation must to be identified with a \texttt{Calc}
identifier, but the underlying basis can be either
a matrix calculation constant (see~\ref{sssec:calcconstants}) or an escaped C expression
(see~\ref{ssec:escapedC}). A \texttt{Kernel} is composed either using the composition
operator (see section~\ref{sssec:barop}), or the \texttt{|=} assignment operator (see section~\ref{sssec:barequalop}).

To see how a \texttt{Kernel} is used in calculation, see section~\ref{ssec:convolutionop}.

\section{Expressions}
\label{sec:expressions}

\subsection{Primary Expressions}
\label{ssec:primaryexpresions}
\sys{} primary expressions can be identifiers, constants, or strings.
The type of the expression depends on the identifier, constant or string.

\subsection{Unary Operators}
\label{ssec:unaryoperators}
There are two unary operators in \sys{}, and they are only used with a
numeric-valued operand such as a numeric constant
(see~\ref{sssec:numericconstants}).
These expressions are grouped right-to-left:
\startsyn
\texttt{+}\emph{unsigned-integer} \\
\texttt{-}\emph{unsigned-integer}
\stopsyn

\subsubsection{\texttt{+} operator}
This operator forces the value of its numeric operand to be positive.
The resulting expression is of numeric type with a value equal to the
value of the numeric operand.

\subsubsection{\texttt{-} operator}
This operator forces the value of its numeric operand to be negative.
The resulting expression is of numeric type with a value equal to the
negative of the numeric operand.

\subsection{Channel/Calc Expresions}
\label{ssec:channelexpressions}
Channel and \texttt{Calc} types are the basis of \texttt{Image} and
\texttt{Kernel} objects respectively. There are several operators that
manipulate Channels and \texttt{Calc}s.

\subsubsection{\texttt{:} operator}
\label{sssec:colonop}
Extract, reference or assign an individual Channel in an image.
\startsyn
\emph{image-identifier}\texttt{:}\emph{channel-identifier}
\stopsyn
The resulting expression has a type corresponding to the
extracted Channel.

\subsection{Composition Operators}
\label{ssec:compositionops}
These operators compose an \texttt{Image} or \texttt{Kernel} from one
or more \texttt{Calc} objects.
All channel composition operators are left-to-right associative.

\subsubsection{\texttt{|} operator}
\label{sssec:barop}
Define of list of (one or more) \texttt{Calc}s. The resulting expression is a
\emph{multi-calc} expression, and can be assigned
to a \texttt{Kernel} object.
\startsyn
\texttt{|} \emph{calc-expression} \\
\emph{multi-calc-expression} \texttt{|} \emph{calc-expression} \\
\stopsyn
Note that \texttt{Calc}s are appended in order, and
subsequent operations may rely on this order.
Also note that even single \texttt{Calc} identifiers must be preceded by \texttt{|}

\subsection{\texttt{**} operator}
\label{ssec:convolutionop}

The convlution operator, or \texttt{**}, performs the calculations of a
\texttt{Kernel} on a Channel, and evaluates to a new \texttt{Image}. The
resulting \texttt{Image} will have a Channel for the result of each
(non-transient) calculation.
When a \texttt{Kernel} is evaluated, as many operations as possible are
parallelized to improve performance.

Parallelized calculations may depend on each other as long as the
dependent calculation is listed after its dependency when the \texttt{Kernel} is
defined, and the dependent calculation does not depend on more than
the current pixel of the required calculation.

Any channels marked with an \texttt{@} symbol are transient and
are not added to the result \texttt{Image}.

\subsection{Escaped ``C'' Expression}
\label{ssec:escapedC}

Escaped ``C'' expressions, or \emph{CStrings}, are snippits of ``C'' code
which will be run once for every pixel in an \texttt{Image} Channel. The
string may reference other Channels in the \texttt{Image} provided the
Channels have been previously defined. To reference a Channel within a
CString, the name of the Channel is used just like a local variable in C.
CStrings are used on the right side
of the \texttt{:=} operator when defining a Channel.

The escaped code must be a single expression
in C that will evaluate to the type defined by the containing \texttt{Calc} object.
The expression cannot contain the following characters/tokens:
\texttt{; \# \{ \} " ' /* */}. Parentheses may be used to group expressions or
cast objects, but all parenthesis within the expression must be matched.

\texttt{Calc} types and their C-equivalent types:
\begin{center}\begin{tabular}{l | l}
\sys type & equivalent C type \\
\hline
Uint8  & uint8_t \\
Uint16 & uint16_t \\
Uint32 & uint32_t \\
Int8  & int8_t \\
Int16 & int16_t \\
Int32 & int32_t \\
Angle  & float
\end{tabular}\end{center}

When the channel described by a CString must be evaluated,
every pixel value in the channel is calculated by evaluating the C expression.
When the expression is evaluated, every identifier corresponding to
another channel in the image will be replaced by the value of the
pixel in the same location in the referenced channel. Thus, if the C expression
contains the identifier \texttt{Red}, then when the channel is calculated
it will replace \texttt{Red} in the expression with the appropriate value
from the \texttt{Red} channel.

All standard C operators are available for use, as well as any library functions
defined in \texttt{math.h}. Although bracket characters are not allowed within
CStrings, the \emph{ternary} conditional operator, \texttt{a ? b : c} is allowed.
This enables more complex pixel value calculations such as thresholding and hysteresis.

\subsection{I/O Expressions}

\subsubsection{\texttt{imgread} expression}
\label{sssec:imgread}
The \texttt{imgread} expression reads in an \texttt{Image} object from
a known image format located on the file system. The expression results
in an \texttt{Image} object which can be assigned using the \texttt{=}
operator (see section~\ref{sssec:equalop}). The resulting \texttt{Image}
object has 3 Channels named \emph{Red}, \emph{Green}, and
\emph{Blue}. Each of the channels correspond to the red, green, and blue
image data read into the \texttt{Image} object. This expression is invoked
as a ``C'' style function, and expects 1 parameter: either the path of the
image file to read (expressed as a string constant); or the number of the
command-line argument, an integer 1 or greater.
\startsyn
\texttt{imgread(} \emph{string-constant} \texttt{|} \emph{integer} \texttt{)}
\stopsyn

\subsubsection{\texttt{imgwrite} expression}
\label{sssec:imgwrite}
The \texttt{imgwrite} expression writes out an \texttt{Image} object to a known
image format. It requires that the \texttt{Image} object has at least 3 named
\texttt{Channels}: \emph{Red}, \emph{Green}, and \emph{Blue}.
This expression has no type (null type), and is invoked as a ``C'' style function.
It expects 3 parameters: the first parameter is an \texttt{Image} identifier, the
second is the image format, and the third is the path to which the image
should be written (or an integer which represents a command-line argument, as for
\texttt{imgread}).
\startsyn
\texttt{imgwrite(} \emph{image-identifier} \texttt{,} \emph{string-constant} \texttt{,} \emph{string-constant} \texttt{|} \emph{integer} \texttt{)}
\stopsyn

\subsection{Assignment Expressions}
\label{ssec:assignment}

\subsubsection{\texttt{=} assignment operator}
\label{sssec:equalop}
Assigns the value of the right operand to the left operand, copying data as necessary.
The types of both operands must match. Cannot be used with \texttt{Calc}s, which are
defined once only (see \texttt{:=} below).

The type of this expression is equal to the type of the left operand, and assignment
operations may be chained together. For example:
\begin{lstlisting}[language=CLAM]
Image a;
Image b;
imgwrite(a = b = imgread("foo.jpg"), "png", "foo.png");
\end{lstlisting}

\subsubsection{\texttt{:=} assignment operator}
\label{sssec:colonequalop}
Assigns a calculation constant (see section~\ref{sssec:calcconstants}), or
escaped ``C'' expression (see section~\ref{ssec:escapedC}) to a \texttt{Calc}
object. Only used once for each \texttt{Calc}, with declaration.

\subsubsection{\texttt{|=} assignment operator}
\label{sssec:barequalop}
Add a single Channel or a (possibly one-member) list of \texttt{Calc}s object to an \texttt{Image} or
\texttt{Kernel} object. Assignments using this operator are ordered by statement
order, and subsequent operations can rely on this order.

\section{Statements}
\label{sec:statements}

Statements in \sys{} always end in a semi-colon. No statement
can return a value. All statements should either declare a variable,
define or modify the definition of a variable, execute
some calculation based on previously declared variables with
the result stored in previously declared variables, or write an image to a file.
"Statements" consisting of a lone r-value (an identifier,
channel reference, escaped-C string, matrix, or \texttt{imgread()} call)
will be accepted but will perform no useful action - they are evaluated but
their return values are discarded. Additionally, the \sys{} compiler may
optionally remove these statements as an optimization.

\section{Program Definition}
A program in the \sys{} language is simply a sequence of statements which
are executed in order.

\section{Scope Rules}
All identifiers in the \sys{} language are global.

In a CString that defines a channel, the existing channels
for an image will be in scope when the block is executed. Because
this block will be executed on every pixel, the name of the channel
will bind to the current pixel value for that channel. These bindings
will be resolved when the channel is calculated, not when it is
defined, and will be removed after calculation.

\section{Declarations}
All variables must be declared before they can be used. However,
variable declarations can be made at any point in a program.
A variable becomes usable after the end of the semi-colon of the statement
in which it is contained.

\section{Grammar}

\subsection{Expressions}

\begin{center}\begin{tabular}{l l}
\emph{expression}: & \emph{identifier}\\
& \emph{integer}\\
& \emph{literal-string}\\
& \emph{c-string}\\
& \emph{matrix}\\
& \emph{matrix-scale matrix}\\
& \emph{kernel-calc-list}\\
& \emph{channel-ref}\\
& \emph{identifier} \texttt{=} \emph{expression}\\
& \emph{channel-ref} \texttt{=} \emph{expression}\\
& \emph{channel-ref} \texttt{**} \emph{identifier}\\
& \emph{identifier} \texttt{|=} \emph{expression}\\
& \emph{library-function} \texttt{(} \emph{argument-list} \texttt{)}\\
\\
\emph{matrix-scale}: & \texttt{[} \emph{integer} \texttt{/} \emph{integer} \texttt{]}\\
\\
\emph{matrix}: & \texttt{\{} \emph{row-list} \texttt{\}}\\
& \emph{matrix-scale} \texttt{\{} \emph{row-list} \texttt{\}}\\
\\
\emph{row-list}: & \emph{matrix-row}\\
& \emph{row-list} \texttt{,} \emph{matrix-row}\\
\\
\emph{matrix-row}: & \emph{integer}\\
& \emph{matrix-row} \emph{integer}\\
\\
\emph{kernel-calc-list}: & \texttt{@}\emph{identifier}\\
& \texttt{|} \emph{identifier}\\
& \texttt{| @}\emph{identifier}\\
& \emph{kernel-calc-list} \texttt{|} \emph{identifier}\\
& \emph{kernel-calc-list} \texttt{| @}\emph{identifier}\\
\\
\emph{channel-ref}: & \emph{identifier}\texttt{:}\emph{identifier}\\
\\
\emph{argument-list}: & \emph{literal-string}\\
& \emph{argument-list} \texttt{,} \emph{literal-string}\\
\\
\emph{library-function}: & \texttt{imgread}\\
& \texttt{imgwrite}\\
\end{tabular}\end{center}

\subsection{Declarations}

\begin{center}\begin{tabular}{l l}
\emph{declaration}: & \texttt{Image} \emph{identifier}\\
& \texttt{Kernel} \emph{identifier}\\
& \texttt{Calc} \emph{identifier}\\
& \texttt{Calc} \emph{identifier}\texttt{<}\emph{atomic-type}\texttt{>}\\
\\
\emph{atomic-type}: & \texttt{Uint8}\\
& \texttt{Uint16}\\
& \texttt{Uint32}\\
& \texttt{Int8}\\
& \texttt{Int16}\\
& \texttt{Int32}\\
& \texttt{Angle}\\
\end{tabular}\end{center}


\subsection{Statements and Programs}

\begin{center}\begin{tabular}{l l}
\emph{statement}: & \emph{expression} \texttt{;}\\
& \emph{declaration} \texttt{;}\\
& \emph{declaration} \texttt{=} \emph{expression} \texttt{;}\\
& \emph{declaration} \texttt{:=} \emph{expression} \texttt{;}\\
\\
\emph{program}: & \emph{statement}\\
& \emph{statement} \emph{program}\\
\end{tabular}\end{center}




\clearpage
\section{Examples}

\subsection{Gaussian Blur}
Figure~\ref{fig:clamblur} shows an example \sys{} program which performs a Gaussian
blur on an input image. The input and resulting output images are shown in
Figures~\ref{fig:clamblurin} and~\ref{fig:clamblurout}.

\begin{figure}[hb!]
  \begin{center}
  \lstinputlisting[language=CLAM]{src/blur.clam}
  \caption{Gaussian Blur Implemented in \sys{}}
  \label{fig:clamblur}
  \end{center}
\end{figure}
{
  \captionsetup{justification=centering}
  \begin{figure*}[h!]
    \centering
    \subfloat[Input Image]{
      \includegraphics*[width=7.5cm]{figures/lena.png}\label{fig:clamblurin}
    }\hfill
    \subfloat[Blurred Output Image]{
      \includegraphics*[width=7.5cm]{figures/lena-blur.png}\label{fig:clamblurout}
    }\hfill
    \captionsetup{font=bf}
    \caption{Gaussian Blur \sys{} Example}
  \end{figure*}
}

\clearpage
\subsection{Image Segmentation}
Figure~\ref{fig:clamseg} shows an example \sys{} program which performs
basic image segmentation. Pixels with a luminance value greater than 200 are
displayed as red, pixels with a luminance less than or equal to 80 are displayed as blue,
and pixels in between are displayed as Green. The input and resulting output images are shown in
Figures~\ref{fig:clamsegin} and~\ref{fig:clamsegout}.

\begin{figure}[hb!]
  \begin{center}
  \lstinputlisting[language=CLAM]{src/segment.clam}
  \caption{Simple Image Segmentation Implemented in \sys{}}
  \label{fig:clamseg}
  \end{center}
\end{figure}
{
  \captionsetup{justification=centering}
  \begin{figure*}[h!]
    \centering
    \subfloat[Input Image]{
      \includegraphics*[width=7.5cm]{figures/lena.png}\label{fig:clamsegin}
    }\hfill
    \subfloat[Segmented Output Image]{
      \includegraphics*[width=7.5cm]{figures/lena-seg.png}\label{fig:clamsegout}
    }\hfill
    \captionsetup{font=bf}
    \caption{Image Segmentation \sys{} Example}
  \end{figure*}
}
