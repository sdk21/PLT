\section{ Scanner }
scanner.mll
\begin{lstlisting}
(* Christopher Campbell, Winnie Narang*)
{ open Parser }

let whitespace = [' ' '\t' '\r' '\n'] 
let name = ['a'-'z' 'A'-'Z'] ['a'-'z' 'A'-'Z' '0'-'9' '_']*
let integers = ['0'-'9']+
let floats = ['0'-'9']+ '.' ['0'-'9']*

rule token = parse
  whitespace { token lexbuf }
| '#'        { comment lexbuf }

| "int"      { INT }    (* Integer type *)
| "float"    { FLOAT }  (* Float type *)
| "comp"     { COMP }   (* Complex type *)
| "mat"      { MAT }    (* Matrix *)

| "C"        { C }      (* Start of complex number *)
| "I"        { I }      (*  Imaginary component *)

| "def"      { DEF }    (* Define function *)

| '='        { ASSIGN } (* Assignment *)
| ','        { COMMA }  (* Separate list elements *)
| ':'        { COLON }  (* Separate matrix rows *)
| ';'        { SEMI }   (* Separate matrix columns *)
| '('        { LPAREN } (* Surround expression *)
| ')'        { RPAREN } 
| '['        { LBRACK } (* Surround vectors/matricies *)
| ']'        { RBRACK }
| '{'        { LBRACE } (* Surround blocks *)
| '}'        { RBRACE }  
| '<'        { LCAR }   (* Open bra- *)
| '>'        { RCAR }   (* Close -ket *)
| '|'        { BAR }    (* Close bra- and Open -ket *)

| '+'        { PLUS }   (* Addition *)
| '-'        { MINUS }  (* Subtraction *)
| '*'        { TIMES }  (* Multiplication *)
| '/'        { DIV }    (* Division *)
| '%'        { MOD }    (* Modulus *)
| '^'        { EXPN }   (* Exponentiation *)

| "eq"       { EQ }     (* Equal to (structural) *)
| "neq"      { NEQ }    (* Not equal to (structural) *) 
| "lt"       { LT }     (* Less than *)
| "gt"       { GT }     (* Greater than *)
| "leq"      { LEQ }    (* Less than or equal to *)
| "geq"      { GEQ }    (* Greater than or equal to *)

| "not"      { NOT }    (* Boolean not *)
| "and"      { AND }    (* Boolean and *)
| "or"       { OR }     (* Boolean or *)
| "xor"      { XOR }    (* Boolean xor *)

| "norm"     { NORM }   (* Get norm *)
| "trans"    { TRANS }  (* Get transpose *)
| "det"      { DET }    (* Get determinant *) 
| "adj"      { ADJ }    (* Get adjoint *)
| "conj"     { CONJ }   (* Get complex conjugate *)
| "unit"     { UNIT }   (* Is unit matrix? *)
| '@'        { TENS }   (* Tensor product *)
| "im"       { IM }     (* Is imaginary number? *)
| "re"       { RE }     (* Is real number *)
| "sin"      { SIN }    (* Sine *)
| "cos"      { COS }    (* Cosine *)
| "tan"      { TAN }    (* Tangent *)

| "if"       { IF }     (* If statement *)
| "else"	 { ELSE }	(* Else statement *)
| "for"      { FOR }    (* For loop - for(i from x to y by z) *)
| "from"     { FROM }
| "to"       { TO }
| "by"       { BY }
| "while"    { WHILE }  (* While loop *)
| "break"    { BREAK }  (* Break For or While loop *)
| "continue" { CONT }   (* Continue to For or While loop *)

| name     as lxm  { ID(lxm) } 
| integers as lxm  { INT_LIT(lxm) }
| floats   as lxm  { FLOAT_LIT(float_of_string lxm) }

| eof              { EOF }
| _ as char        { raise (Failure("illegal character " ^ Char.escaped char)) }

and comment = parse
  ['\r' '\n']       { token lexbuf }
| _                
\end{lstlisting}

\section{Parser}
parser.mly

\begin{lstlisting}
(* Christopher Campbell, Sankalpa Khadka*)
%{ open Ast %}

%token C I
%token INT FLOAT COMP MAT
%token DEF
%token ASSIGN
%token COMMA COLON SEMI LPAREN RPAREN LBRACK RBRACK LBRACE RBRACE LCAR RCAR BAR
%token PLUS MINUS TIMES DIV MOD EXPN
%token EQ NEQ LT GT LEQ GEQ
%token NOT AND OR XOR
%token TENS UNIT NORM TRANS DET ADJ CONJ IM RE SIN COS TAN
%token IF ELIF ELSE FOR FROM TO BY WHILE BREAK CONT
%token EOF

%token <string> ID
%token <string> INT_LIT
%token <float> FLOAT_LIT
%token <string> COMP_LIT

%nonassoc NOELSE
%nonassoc ELSE
%right ASSIGN
%left OR XOR
%left AND
%right NOT
%left EQ NEQ
%left LT GT LEQ GEQ
%left PLUS MINUS
%left TIMES DIV MOD TENS
%right EXPN
%nonassoc RE IM NORM TRANS DET ADJ CONJ UNIT SIN COS TAN

%start program
%type <Ast.program> program

%%

vtype:
  INT     { Int }
  | FLOAT { Float }
  | COMP  { Comp }
  | MAT   { Mat }

vdecl:
  vtype ID SEMI { { typ = $1;
                    name = $2 } } 
vdecl_list:
    /* nothing */    { [] }
  | vdecl_list vdecl { $2 :: $1 }

formal_params:
    /* nothing */      { [] }
  | formal_params_list { List.rev $1 }

formal_params_list:
  vtype ID                            { [{ typ = $1; 
                                           name = $2; }] }
  | formal_params_list COMMA vtype ID { {  typ = $3;
                                           name = $4; } :: $1 }
actual_params:
    /* nothing */      { [] }
  | actual_params_list { List.rev $1 }

actual_params_list:
    expr                          { [$1] }
  | actual_params_list COMMA expr { $3 :: $1 }

fdecl:
   DEF ID LPAREN formal_params RPAREN COLON vtype ID LBRACE vdecl_list stmt_list RBRACE
     { { func_name = $2;
         formal_params = $4;
         ret_typ = $7;
         ret_name = $8;
         locals = List.rev $10;
         body = List.rev $11; } }

mat_row:
    expr               { [$1] }
  | mat_row COMMA expr { $3 :: $1 }

mat_row_list:
    LPAREN mat_row RPAREN              { [List.rev($2)] }
  | mat_row_list LPAREN mat_row RPAREN { List.rev($3) :: $1 }

inner_comp:
    FLOAT_LIT                  { [$1; 0.] }
  | FLOAT_LIT I                { [0.; $1] }
  | FLOAT_LIT PLUS FLOAT_LIT I { [$1; $3] }

expr:
    ID                             { Id($1) }
  | INT_LIT                        { Lit_int(int_of_string $1) }
  | FLOAT_LIT                      { Lit_float($1) }
  | C LPAREN inner_comp RPAREN     { Lit_comp(List.hd $3, List.hd (List.rev $3)) } 
  | LCAR INT_LIT BAR               { Lit_qub($2, 0) }
  | BAR INT_LIT RCAR               { Lit_qub($2, 1) }
  | LBRACK mat_row_list RBRACK     { Mat(List.rev($2)) }
  | LPAREN expr RPAREN             { $2 }
  | ID ASSIGN expr                 { Assign($1, $3) }
  | ID LPAREN actual_params RPAREN { Call($1, $3) }
  | MINUS expr                     { Unop(Neg, $2)}
  | NOT LPAREN expr RPAREN         { Unop(Not, $3) }
  | RE LPAREN expr RPAREN          { Unop(Re, $3) }
  | IM LPAREN expr RPAREN          { Unop(Im, $3) }
  | NORM LPAREN expr RPAREN        { Unop(Norm, $3) }
  | TRANS LPAREN expr RPAREN       { Unop(Trans, $3) }
  | DET LPAREN expr RPAREN         { Unop(Det, $3) }
  | ADJ LPAREN expr RPAREN         { Unop(Adj, $3) }
  | CONJ LPAREN expr RPAREN        { Unop(Conj, $3) }
  | UNIT LPAREN expr RPAREN        { Unop(Unit, $3) }
  | SIN LPAREN expr RPAREN         { Unop(Sin, $3) }
  | COS LPAREN expr RPAREN         { Unop(Cos, $3) }
  | TAN LPAREN expr RPAREN         { Unop(Tan, $3) }
  | expr PLUS   expr               { Binop($1, Add,  $3) }
  | expr MINUS  expr               { Binop($1, Sub,  $3) }
  | expr TIMES  expr               { Binop($1, Mult, $3) }
  | expr DIV    expr               { Binop($1, Div,  $3) }
  | expr MOD    expr               { Binop($1, Mod,  $3) }
  | expr EXPN   expr               { Binop($1, Expn, $3) }
  | expr TENS   expr               { Binop($1, Tens, $3) }
  | expr EQ     expr               { Binop($1, Eq,   $3) }
  | expr NEQ    expr               { Binop($1, Neq,  $3) }
  | expr LT     expr               { Binop($1, Lt,   $3) }
  | expr GT     expr               { Binop($1, Gt,   $3) }
  | expr LEQ    expr               { Binop($1, Leq,  $3) }
  | expr GEQ    expr               { Binop($1, Geq,  $3) }
  | expr OR     expr               { Binop($1, Or,   $3) }
  | expr AND    expr               { Binop($1, And,  $3) }
  | expr XOR    expr               { Binop($1, Xor,  $3) }

 by:
   /* nothing */ { Noexpr }
 | BY expr       { $2 }

stmt:  
    expr SEMI                                        { Expr($1) }
  | LBRACE stmt_list RBRACE                          { Block(List.rev $2) }
  | FOR LPAREN expr FROM expr TO expr by RPAREN stmt { For($3, $5, $7, $8, $10) }
  | WHILE LPAREN expr RPAREN stmt                    { While($3, $5) }
  | IF LPAREN expr RPAREN stmt %prec NOELSE          { If($3, $5, Ast.Expr(Ast.Noexpr)) }
  | IF LPAREN expr RPAREN stmt ELSE stmt             { If($3, $5, $7) }
  | BREAK SEMI                                       { BreakCont(0) }
  | CONT  SEMI                                       { BreakCont(1) }

stmt_list:
    /* nothing  */ { [] }
  | stmt_list stmt { $2 :: $1 }

 rev_program:
   /* nothing */     { [] }
 | rev_program fdecl { $2 :: $1 }

program:
  rev_program { List.rev $1 }
\end{lstlisting}

\section{AST}
ast.ml
\begin{lstlisting}
(* Christopher Campbell, Winnie Narang *)
(* Elementary Data Types *)
type data_type =
    Int
  | Float
  | Comp
  | Mat

(* Unary Operators *) 
type un_op =
    Neg
  | Not
  | Re
  | Im
  | Norm
  | Trans
  | Det
  | Adj
  | Conj
  | Unit
  | Sin
  | Cos
  | Tan

(* Binary Operators *)
type bi_op =
    Add
  | Sub
  | Mult
  | Div
  | Mod
  | Expn
  | Tens
  | Eq
  | Neq
  | Lt
  | Gt
  | Leq
  | Geq
  | Or
  | And
  | Xor

(* Expressions *)
type expr =
    Lit_int of int
  | Lit_float of float
  | Lit_comp of float * float
  | Lit_qub of string * int
  | Mat of expr list list
  | Id of string
  | Unop of un_op * expr
  | Binop of expr * bi_op * expr
  | Assign of string * expr
  | Call of string * expr list
  | Noexpr 

(* Statements *)
type stmt =
    Expr of expr
  | Block of stmt list
  | If of expr * stmt * stmt
  | For of expr * expr * expr * expr * stmt
  | While of expr * stmt
  | BreakCont of int

(* Statement Lists *)
type stmt_list =
  stmt list
 
(* Variables Declaration *)
type var_decl = 
  { 
    typ : data_type;
    name : string;
  }

(* Function Declaration *)
type func_decl = 
  {
    ret_typ : data_type;
    ret_name : string;
    func_name : string;
    formal_params : var_decl list;
    locals : var_decl list;
    body : stmt list;
  }

(* Program *)
type program =
  func_decl list

(* Pretty Printer *)
let rec string_of_expr = function
    Lit_int(n) -> string_of_int n
  | Lit_float(n) -> string_of_float n
  | Lit_comp(f1,f2) -> string_of_float f1 ^ " + " ^ string_of_float f2 ^ "i"
  | Lit_qub(s,t) -> let typ = string_of_int t in (match typ with
                      "0" -> "Qub-bra of "^ s 
                    | _ -> "Qub-ket of "^ s)
  | Mat(l) ->  string_of_mat l
  | Id(s) -> s
  | Unop(un1,exp1) -> 
    (match un1 with
      Neg -> " -"
    | Not -> " ! "
    | Re -> " Re "
    | Im -> " Im "
    | Norm -> " Norm "
    | Trans -> " Trans "
    | Det -> " Det "
    | Adj -> " Adj "
    | Conj -> " Conj "
    | Unit -> " Unit "
    | Sin -> " Sin "
    | Cos -> " Cos "
    | Tan -> " Tan ") ^ string_of_expr exp1 

  | Binop(ex1,binop,ex2) -> string_of_expr ex1 ^ 
    (match binop with 
      Add -> " + "    | Sub -> " - "     | Mult -> " * " 
    | Div -> " / "    | Mod -> " % "     | Expn -> " ^ " | Tens -> " @ "
    | Eq-> " == "     | Neq -> " != "    | Lt -> " < "
    | Leq -> " <= "   | Gt -> " > "      | Geq -> " >= "
    | Xor -> " XOR "  | And -> " && "    | Or -> " || ") ^ string_of_expr ex2
  | Assign(str,expr) -> str ^ " = " ^ string_of_expr expr
  | Call(str,expr_list) -> "Calling " ^ str ^ " on " ^string_of_exprs expr_list
  | Noexpr -> ""

and string_of_mat l =
  let row_strs = 
    List.map string_of_row l
  in
    "[" ^ String.concat "" row_strs ^ "]"

and string_of_row r =
  let row_str = 
    String.concat "," (List.map string_of_expr r)
  in
    "(" ^ row_str ^ ")"
  
and string_of_exprs exprs = 
  String.concat "\n" (List.map string_of_expr exprs)

and string_of_stmt = function
    Expr(exp) -> string_of_expr exp ^ "\n"
  | Block(stmt_list) -> "{\n" ^ string_of_stmts stmt_list ^ "\n}"
  | If(e, s, Block([])) -> "if (" ^ string_of_expr e ^ ")\n" ^ string_of_stmt s
  | If(e, s1, s2) ->  "if (" ^ string_of_expr e ^ ")\n" ^string_of_stmt s1 ^ "else\n" ^ string_of_stmt s2
  | For(ex1,ex2,ex3,ex4,stmt) -> "For args : " ^ string_of_expr ex1 ^ " " ^ string_of_expr ex2 ^ " "^ string_of_expr ex3 ^ 
                                 " "^ string_of_expr ex4 ^ "\nstatement :\n" ^ string_of_stmt stmt 
  | While(expr,stmt) -> "While condition : " ^ string_of_expr expr ^ "\nstatement : " ^ string_of_stmt stmt
  | BreakCont(t) -> string_of_breakcont t

and string_of_breakcont t =
  if (t = 0) then
  "break"
  else
  "continue"

and string_of_stmts stmts = 
  String.concat "\n" (List.map string_of_stmt stmts) 

and string_of_var_decl var_decl = 
  "vdecl: typ: " ^ 
    (match var_decl.typ with
      Int -> "int," ^ " name: " ^ var_decl.name^ "  "
    | Float -> "float," ^ " name: " ^ var_decl.name^ "  "
    | Comp -> "comp," ^ " name: " ^ var_decl.name^ "  "
    | Mat -> "mat," ^ " name: " ^ var_decl.name^ "  ")
  
and string_of_fdecl fdecl =
  "\nfdecl:\nret_typ: " ^ 
    (match fdecl.ret_typ with
      Int -> " int "
    | Float -> " float "
    | Comp -> " comp "
    | Mat -> " mat ") ^
      "\nret_name: " ^ fdecl.ret_name ^ "\nfunc_name: "  ^ fdecl.func_name ^  "\n(" ^
        String.concat "" (List.map string_of_var_decl fdecl.formal_params) ^ ")\n{\n" ^
          String.concat "" (List.map string_of_var_decl fdecl.locals) ^ "\n" ^
            String.concat "" (List.map string_of_stmt fdecl.body) ^ "}"

and string_of_program (funcs) = 
  "program:\n" ^ String.concat "\n" (List.map string_of_fdecl funcs)
\end{lstlisting}

\section{Analyzer}
analyzer.ml
\begin{lstlisting}
(* Christopher Campbell *)
open Ast
open Sast

(***************
 * Environment *
***************)

type symbol_table =
  { ret_typ : Sast.sdata_type;
    ret_nam : string;
    func_nam : string;
    mutable formal_param : svar_decl list;
    mutable local : svar_decl list; 
    builtin : svar_decl list; }

type environment =
  { scope : symbol_table;
    mutable functions : Sast.sfunc_decl list; }

let builtin_vars =
  [ { styp = Sast.Float; sname = "e"; builtinv = true; };
    { styp = Sast.Float; sname = "pi"; builtinv = true; };
    { styp = Sast.Mat; sname = "X"; builtinv = true; };
    { styp = Sast.Mat; sname = "Y"; builtinv = true; };
    { styp = Sast.Mat; sname = "Z"; builtinv = true; };   
    { styp = Sast.Mat; sname = "H"; builtinv = true; };
    { styp = Sast.Mat; sname = "IDT"; builtinv = true; }; ]

let builtin_funcs = 
  [ { sret_typ = Sast.Void;
      sret_name = "null";
      sfunc_name = "print";
      sformal_params = [{ styp = Sast.Poly; sname = "print_val"; builtinv = true; };];
      slocals  = [];
      sbody = [Sast.Sexpr(Sast.Expr(Sast.Noexpr, Sast.Void))];
      builtinf = true; };

    { sret_typ = Sast.Void;
      sret_name = "null";
      sfunc_name = "printq";
      sformal_params = [{ styp = Sast.Mat; sname = "printq_val"; builtinv = true; };];
      slocals  = [];
      sbody = [Sast.Sexpr(Sast.Expr(Sast.Noexpr, Sast.Void))];
      builtinf = true; }; 

    { sret_typ = Sast.Int;
      sret_name = "null";
      sfunc_name = "rows";
      sformal_params = [{ styp = Sast.Mat; sname = "rows_val"; builtinv = true; };];
      slocals  = [];
      sbody = [Sast.Sexpr(Sast.Expr(Sast.Noexpr, Sast.Void))];
      builtinf = true; };

    { sret_typ = Sast.Int;
      sret_name = "null";
      sfunc_name = "cols";
      sformal_params = [{ styp = Sast.Mat; sname = "rows_val"; builtinv = true; };];
      slocals  = [];
      sbody = [Sast.Sexpr(Sast.Expr(Sast.Noexpr, Sast.Void))];
      builtinf = true; };

    { sret_typ = Sast.Comp;
      sret_name = "null";
      sfunc_name = "elem";
      sformal_params = [{ styp = Sast.Mat; sname = "elem_mat"; builtinv = true; };
                        { styp = Sast.Int; sname = "elem_row"; builtinv = true; };
                        { styp = Sast.Int; sname = "elem_col"; builtinv = true; };];
      slocals  = [];
      sbody = [Sast.Sexpr(Sast.Expr(Sast.Noexpr, Sast.Void))];
      builtinf = true; }; ]

let root_symbol_table =
  { ret_typ = Sast.Void;
    ret_nam = "";
    func_nam = "";
    formal_param = [];
    local = []; 
    builtin = builtin_vars; }

let root_environment = 
  { scope = root_symbol_table;
    functions = builtin_funcs; }

(**************
 * Exceptions *
**************)

exception Except of string

let matrix_error t = match t with
    0 -> raise (Except("Invalid matrix: incorrect type"))
  | _ -> raise (Except("Invalid matrix"))

let qub_error t = match t with
    0 -> raise (Except("Invalid qubit: incorrect use of |expr>"))
  | 1 -> raise (Except("Invalid qubit: incorrect use of <expr|"))
  | _ -> raise (Except("Invalid qubit"))

let assignment_error s =
  raise (Except("Invalid assignment to variable: " ^ s))

let var_error s =
  raise (Except("Invalid use of a variable: " ^ s ^ " was not declared" ))

let func_error s =
  raise (Except("Invalid function call: " ^ s ^ " was not declared" ))

let var_decl_error s =
  raise (Except("Invalid variable declaration: " ^ s ^ " was already declared" ))

let func_decl_error s =
  raise (Except("Invalid function declaration: " ^ s ^ " was already declared" ))

let unop_error t = match t with
    Ast.Neg -> raise (Except("Invalid use of unop: '-expr'"))
  | Ast.Not -> raise (Except("Invalid use of unop: 'Not(expr)'"))
  | Ast.Re -> raise (Except("Invalid use of unop: 'Re(expr)'"))
  | Ast.Im -> raise (Except("Invalid use of unop: 'Im(expr)'"))
  | Ast.Norm -> raise (Except("Invalid use of unop: 'Norm(expr)'"))
  | Ast.Trans -> raise (Except("Invalid use of unop: 'Trans(expr)'"))
  | Ast.Det -> raise (Except("Invalid use of unop: 'Det(expr)'"))
  | Ast.Adj -> raise (Except("Invalid use of unop: 'Adj(expr)'"))
  | Ast.Conj -> raise (Except("Invalid use of unop: 'Conjexpr)'"))
  | Ast.Unit -> raise (Except("Invalid use of unop: 'Unit(expr)'"))
  | Ast.Sin -> raise (Except("Invalid use of unop: 'Sin(expr)'"))
  | Ast.Cos -> raise (Except("Invalid use of unop: 'Cos(expr)'"))
  | Ast.Tan -> raise (Except("Invalid use of unop: 'Tan(expr)'"))

let binop_error t = match t with
    Ast.Add -> raise (Except("Invalid use of binop: 'expr + expr'"))
  | Ast.Sub -> raise (Except("Invalid use of binop: 'expr - expr'"))
  | Ast.Mult -> raise (Except("Invalid use of binop: expr * expr'"))
  | Ast.Div -> raise (Except("Invalid use of binop: 'expr / expr'"))
  | Ast.Mod -> raise (Except("Invalid use of binop: 'expr % expr'"))
  | Ast.Expn -> raise (Except("Invalid use of binop: 'expr ^ expr'"))
  | Ast.Or -> raise (Except("Invalid use of binop: 'expr or expr'"))
  | Ast.And -> raise (Except("Invalid use of binop: 'expr and expr'"))
  | Ast.Xor -> raise (Except("Invalid use of binop: 'expr xor expr'"))
  | Ast.Tens -> raise (Except("Invalid use of binop: 'expr @ expr'"))
  | Ast.Eq -> raise (Except("Invalid use of binop: 'expr eq expr'"))
  | Ast.Neq -> raise (Except("Invalid use of binop: 'expr neq expr'"))
  | Ast.Lt -> raise (Except("Invalid use of binop: 'expr lt expr'"))
  | Ast.Gt -> raise (Except("Invalid use of binop: 'expr gt expr'"))
  | Ast.Leq -> raise (Except("Invalid use of binop: 'expr leq expr'"))
  | Ast.Geq -> raise (Except("Invalid use of binop: 'expr geq expr'"))

let expr_error t = match t with
  _ -> raise (Except("Invalid expression"))

let call_error t = match t with
    0 -> raise (Except("Invalid function call: function undeclared"))
  | 1 -> raise (Except("Invalid function call: incorrect number of parameters"))
  | 2 -> raise (Except("Invalid function call: incorrect type for parameter"))
  | _ -> raise (Except("Invalid function call"))

let stmt_error t = match t with
    0 -> raise (Except("Invalid use of statment: 'if'"))
  | 1 -> raise (Except("Invalid use of statment: 'for'"))
  | 2 -> raise (Except("Invalid use of statment: 'while'"))
  | _ -> raise (Except("Invalid statement"))

let program_error t = match t with
    0 -> raise (Except("Missing 'compute' function"))
  | 1 -> raise (Except("'compute' function must be of type int"))
  | _ -> raise (Except("Invalid program"))

(*********************
 * Utility Functions *
*********************)

let var_exists name scope =
  if (List.exists (fun vdecl -> name = vdecl.sname) scope.formal_param) then true
  else if (List.exists (fun vdecl -> name = vdecl.sname) scope.formal_param) then true
  else List.exists (fun vdecl -> name = vdecl.sname) scope.builtin

let func_exists name env =
  List.exists (fun fdecl -> name = fdecl.sfunc_name) env.functions

let lookup_var name scope =
  let vdecl_found = 
  try List.find (fun vdecl -> name = vdecl.sname) scope.formal_param
  with Not_found ->
    try List.find (fun vdecl -> name = vdecl.sname) scope.local
    with Not_found ->
      try List.find (fun vdecl -> name = vdecl.sname) scope.builtin
      with Not_found -> var_error name in
  vdecl_found

let lookup_func name env =
  let fdecl_found = 
    try
      List.find (fun fdecl -> name = fdecl.sfunc_name) env.functions
    with Not_found -> func_error name
  in
    fdecl_found

(**********
 * Checks *
**********)

let rec check_qub_expr i =
  let r = i mod 10 in
   if (r = 0 || r = 1) then
     let i = i / 10 in
       if (i != 0)
        then
         check_qub_expr i
       else 1
    else 0

and check_qub i t =
  let int_expr =
    int_of_string i
  in
    if (check_qub_expr int_expr = 1) then
        (match t with
            0 -> Sast.Expr(Sast.Lit_qub(i, 1), Sast.Mat)
          | 1 -> Sast.Expr(Sast.Lit_qub(i, 0), Sast.Mat)
          | _ -> qub_error 2)
    else
      qub_error t

and check_mat l env =
  let mat = 
    List.map (fun row -> check_mat_rows row env) l
  in
    Sast.Expr(Sast.Mat(mat), Sast.Mat)

and check_mat_rows l env =
  let row =
    List.map (fun e -> check_mat_row e env) l
  in row

and check_mat_row e env =
  let se =
    check_expr env e
  in
    match se with
       Sast.Expr(_, t) ->
        match t with
            Sast.Int -> se
          | Sast.Float -> se
          | Sast.Comp -> se 
          | _ -> matrix_error 0

and check_id name env =
  let vdecl =
    lookup_var name env.scope
  in
    let typ = vdecl.styp in
      Sast.Expr(Sast.Id(name), typ)

and check_unop op e env =
  let e = check_expr env e in 
    match e with
      Sast.Expr(q, t) ->
        (match op with
            Ast.Neg ->
            (match t with
                Sast.Int -> Sast.Expr(Sast.Unop(op, e), Sast.Int)
              | Sast.Float -> Sast.Expr(Sast.Unop(op, e), Sast.Float)
              | Sast.Comp -> Sast.Expr(Sast.Unop(op, e), Sast.Comp)
              | _ -> unop_error op)
          | Ast.Not ->
            (match t with
                Sast.Int -> Sast.Expr(Sast.Unop(op, e), Sast.Int)
              | _ ->  unop_error op)
          | Ast.Re ->
            (match t with
                Sast.Comp -> Sast.Expr(Sast.Unop(op, e), Sast.Comp)
              | _ ->  unop_error op)
          | Ast.Im -> 
            (match t with
                Sast.Comp -> Sast.Expr(Sast.Unop(op, e), Sast.Comp)
              | _ ->  unop_error op)
          | Ast.Unit ->
            (match t with
                Sast.Mat -> Sast.Expr(Sast.Unop(op, e), Sast.Int)
              | _ ->  unop_error op)
          | Ast.Norm ->
            (match t with
                Sast.Mat -> Sast.Expr(Sast.Unop(op, e), Sast.Float)
              | _ ->  unop_error op)
          | Ast.Det ->
            (match t with
                Sast.Mat -> Sast.Expr(Sast.Unop(op, e), Sast.Comp)
              | _ ->  unop_error op)
          | Ast.Trans | Ast.Adj ->
            (match t with
                Sast.Mat -> Sast.Expr(Sast.Unop(op, e), Sast.Mat)
              | _ ->  unop_error op)
          | Ast.Conj ->
            (match t with
                Sast.Comp -> Sast.Expr(Sast.Unop(op, e), Sast.Comp)
              | Sast.Mat -> Sast.Expr(Sast.Unop(op, e), Sast.Mat)
              | _ ->  unop_error op)
          | Ast.Sin -> 
            (match t with
                Sast.Int -> Sast.Expr(Sast.Unop(op, e), Sast.Int)
              | Sast.Float -> Sast.Expr(Sast.Unop(op, e), Sast.Float)
              | Sast.Comp -> Sast.Expr(Sast.Unop(op, e), Sast.Comp)
              | _ ->  unop_error op)
          | Ast.Cos -> 
            (match t with
                Sast.Int -> Sast.Expr(Sast.Unop(op, e), Sast.Int)
              | Sast.Float -> Sast.Expr(Sast.Unop(op, e), Sast.Float)
              | Sast.Comp -> Sast.Expr(Sast.Unop(op, e), Sast.Comp)
              | _ ->  unop_error op)
          | Ast.Tan -> 
            (match t with
                Sast.Int -> Sast.Expr(Sast.Unop(op, e), Sast.Int)
              | Sast.Float -> Sast.Expr(Sast.Unop(op, e), Sast.Float)
              | Sast.Comp -> Sast.Expr(Sast.Unop(op, e), Sast.Comp)
              | _ ->  unop_error op))

and check_binop e1 op e2 env =
    let e1 = check_expr env e1 and e2 = check_expr env e2 in
      match e1 with
          Sast.Expr(_, t1) ->
          (match e2 with
              Sast.Expr(_, t2) ->
              (match op with
                  Ast.Add | Ast.Sub ->
                  (match t1 with
                      Sast.Int -> 
                      (match t2 with
                          Sast.Int -> Sast.Expr(Sast.Binop(e1, op, e2), Sast.Int)
                        | Sast.Float -> Sast.Expr(Sast.Binop(e1, op, e2), Sast.Float)
                        | Sast.Comp -> Sast.Expr(Sast.Binop(e1, op, e2), Sast.Comp)
                        | _ -> binop_error op)
                    | Sast.Float -> 
                      (match t2 with
                          Sast.Int -> Sast.Expr(Sast.Binop(e1, op, e2), Sast.Int)
                        | Sast.Float -> Sast.Expr(Sast.Binop(e1, op, e2), Sast.Float)
                        | Sast.Comp -> Sast.Expr(Sast.Binop(e1, op, e2), Sast.Comp)
                        | _ -> binop_error op)
                    | Sast.Comp -> 
                      (match t2 with
                          Sast.Int -> Sast.Expr(Sast.Binop(e1, op, e2), Sast.Int)
                        | Sast.Float -> Sast.Expr(Sast.Binop(e1, op, e2), Sast.Float)
                        | Sast.Comp -> Sast.Expr(Sast.Binop(e1, op, e2), Sast.Comp)
                        | _ -> binop_error op)
                    | Sast.Mat ->
                      (match t2 with
                          Sast.Mat -> Sast.Expr(Sast.Binop(e1, op, e2), Sast.Mat)
                        | _ -> binop_error op)
                    | _ -> binop_error op)
                | Ast.Mult | Ast.Div ->
                  (match t1 with
                    Sast.Int -> 
                      (match t2 with
                          Sast.Int -> Sast.Expr(Sast.Binop(e1, op, e2), Sast.Int)
                        | Sast.Float -> Sast.Expr(Sast.Binop(e1, op, e2), Sast.Float)
                        | Sast.Comp -> Sast.Expr(Sast.Binop(e1, op, e2), Sast.Comp)
                        | Sast.Mat -> Sast.Expr(Sast.Binop(e1, op, e2), Sast.Mat)
                        | _ -> binop_error op)
                    | Sast.Float -> 
                      (match t2 with
                          Sast.Int | Sast.Float -> Sast.Expr(Sast.Binop(e1, op, e2), Sast.Float)
                        | Sast.Comp -> Sast.Expr(Sast.Binop(e1, op, e2), Sast.Comp)
                        | Sast.Mat -> Sast.Expr(Sast.Binop(e1, op, e2), Sast.Mat)
                        | _ -> binop_error op)
                    | Sast.Comp -> 
                      (match t2 with
                          Sast.Int | Sast.Float | Sast.Comp -> Sast.Expr(Sast.Binop(e1, op, e2), Sast.Comp)
                        | Sast.Mat -> Sast.Expr(Sast.Binop(e1, op, e2), Sast.Mat)
                        | _ -> binop_error op)
                    | Sast.Mat ->
                      (match t2 with
                          Sast.Int | Sast.Float | Sast.Comp | Sast.Mat -> Sast.Expr(Sast.Binop(e1, op, e2), Sast.Mat)
                        | _ -> binop_error op)
                    | _ -> binop_error op)
                | Ast.Mod | Ast.Expn ->
                  (match t1 with
                    Sast.Int -> 
                      (match t2 with
                          Sast.Int -> Sast.Expr(Sast.Binop(e1, op, e2), Sast.Int)
                        | Sast.Float -> Sast.Expr(Sast.Binop(e1, op, e2), Sast.Float)
                        | Sast.Comp -> Sast.Expr(Sast.Binop(e1, op, e2), Sast.Comp)
                        | _ -> binop_error op)
                    | Sast.Float -> 
                      (match t2 with
                          Sast.Int | Sast.Float -> Sast.Expr(Sast.Binop(e1, op, e2), Sast.Float)
                        | Sast.Comp -> Sast.Expr(Sast.Binop(e1, op, e2), Sast.Comp)
                        | _ -> binop_error op)
                    | Sast.Comp -> 
                      (match t2 with
                          Sast.Int | Sast.Float | Sast.Comp -> Sast.Expr(Sast.Binop(e1, op, e2), Sast.Comp)
                        | _ -> binop_error op)
                    | _ -> binop_error op)
                | Ast.Tens ->
                  (match t1 with
                    Sast.Mat ->
                      (match t2 with
                          Sast.Mat -> Sast.Expr(Sast.Binop(e1, op, e2), Sast.Mat)
                        | _ -> binop_error op)
                    | _ -> binop_error op)
                | Ast.Eq | Ast.Neq ->
                  (match t1 with
                    Sast.Int ->
                      (match t2 with
                          Sast.Int -> Sast.Expr(Sast.Binop(e1, op, e2), Sast.Int)
                        | Sast.Float -> Sast.Expr(Sast.Binop(e1, op, e2), Sast.Int)
                        | Sast.Comp -> Sast.Expr(Sast.Binop(e1, op, e2), Sast.Int)
                        | _ -> binop_error op)
                    | Sast.Float ->
                        (match t2 with
                          Sast.Int -> Sast.Expr(Sast.Binop(e1, op, e2), Sast.Int)
                        | Sast.Float -> Sast.Expr(Sast.Binop(e1, op, e2), Sast.Int)
                        | Sast.Comp -> Sast.Expr(Sast.Binop(e1, op, e2), Sast.Int)
                          | _ -> binop_error op)
                    | Sast.Comp ->
                      (match t2 with 
                          Sast.Int -> Sast.Expr(Sast.Binop(e1, op, e2), Sast.Int)
                        | Sast.Float -> Sast.Expr(Sast.Binop(e1, op, e2), Sast.Int)
                        | Sast.Comp -> Sast.Expr(Sast.Binop(e1, op, e2), Sast.Int)
                       | _ -> binop_error op)  
                    | Sast.Mat ->
                        (match t2 with
                            Sast.Mat -> Sast.Expr(Sast.Binop(e1, op, e2), Sast.Int)
                          | _ -> binop_error op)
                    | _ -> binop_error op)
                | Ast.Lt | Ast.Gt | Ast.Leq | Ast.Geq ->
                  (match t1 with
                    Sast.Int ->
                      (match t2 with
                          Sast.Int -> Sast.Expr(Sast.Binop(e1, op, e2), Sast.Int)
                        | Sast.Float -> Sast.Expr(Sast.Binop(e1, op, e2), Sast.Int)
                        | _ -> binop_error op)
                    | Sast.Float ->
                      (match t2 with
                          Sast.Int -> Sast.Expr(Sast.Binop(e1, op, e2), Sast.Int)
                        | Sast.Float -> Sast.Expr(Sast.Binop(e1, op, e2), Sast.Int)
                        | _ -> binop_error op)
                    | _ -> binop_error op)
                | Ast.Or | Ast.And | Ast.Xor -> 
                  (match t1 with
                    Sast.Int ->
                      (match t2 with
                          Sast.Int -> Sast.Expr(Sast.Binop(e1, op, e2), Sast.Int)
                        | Sast.Float -> Sast.Expr(Sast.Binop(e1, op, e2), Sast.Int)
                        | Sast.Comp -> Sast.Expr(Sast.Binop(e1, op, e2), Sast.Int)
                        | _ -> binop_error op)
                    | Sast.Float ->
                        (match t2 with
                          Sast.Int -> Sast.Expr(Sast.Binop(e1, op, e2), Sast.Int)
                        | Sast.Float -> Sast.Expr(Sast.Binop(e1, op, e2), Sast.Int)
                        | Sast.Comp -> Sast.Expr(Sast.Binop(e1, op, e2), Sast.Int)
                          | _ -> binop_error op)
                    | Sast.Comp ->
                      (match t2 with 
                          Sast.Int -> Sast.Expr(Sast.Binop(e1, op, e2), Sast.Int)
                        | Sast.Float -> Sast.Expr(Sast.Binop(e1, op, e2), Sast.Int)
                        | Sast.Comp -> Sast.Expr(Sast.Binop(e1, op, e2), Sast.Int)
                        | _ -> binop_error op)
                    | _ -> binop_error op)))

and check_assign name e env =
  let vdecl = lookup_var name env.scope in
  let e = check_expr env e in
  match e with
    Sast.Expr(_, t1) -> 
      let t2 = vdecl.styp in
        if (t1 = t2) then
          Sast.Expr(Sast.Assign(name, e), t1)
        else
          assignment_error name

and check_call_params formal_params params =
  if ((List.length formal_params) = 0)
    then true
  else
    let fdecl_arg = List.hd formal_params in
    let param = match (List.hd params) with
      Sast.Expr(_, t) -> t in
      if (fdecl_arg.styp = Sast.Poly || (fdecl_arg.styp = param))
        then check_call_params (List.tl formal_params) (List.tl params)
      else false

and check_call name params env = 
  let fdecl =
    try
      lookup_func name env
    with Not_found -> call_error 0 in
    let params = List.map (check_expr env) params in
      if ((List.length fdecl.sformal_params) != (List.length params))
        then call_error 1
      else
        if ((check_call_params fdecl.sformal_params params) = true)
          then Sast.Expr(Sast.Call(name, params), fdecl.sret_typ)
        else
          call_error 2

and check_expr env = function
    Ast.Lit_int(i) -> Sast.Expr(Sast.Lit_int(i), Sast.Int)
  | Ast.Lit_float(f) -> Sast.Expr(Sast.Lit_float(f), Sast.Float)
  | Ast.Lit_comp(f1, f2) -> Sast.Expr(Sast.Lit_comp(f1, f2), Sast.Comp)
  | Ast.Lit_qub(i, t) -> check_qub i t
  | Ast.Mat(l) -> check_mat l env
  | Ast.Id(s) -> check_id s env
  | Ast.Unop(op, e) -> check_unop op e env
  | Ast.Binop(e1, op, e2) -> check_binop e1 op e2 env
  | Ast.Assign(s, e) -> check_assign s e env
  | Ast.Call(s, l) -> check_call s l env
  | Ast.Noexpr -> Sast.Expr(Sast.Noexpr, Sast.Void)

and check_block stmts env =
  let sstmts = List.map (fun stmt -> check_stmt env stmt) stmts in
  Sast.Block(sstmts)

and check_if e s1 s2 env =
    let se = check_expr env e in
      match se with
        Sast.Expr(_,t) ->
          (match t with
            Sast.Int ->
              let ss1 = check_stmt env s1 in
              let ss2 = check_stmt env s2 in
              Sast.If(se, ss1, ss2)
            | _ -> stmt_error 0)

and check_for e1 e2 e3 e4 s env =
  let se1 = check_expr env e1 in
  match se1 with
    Sast.Expr(Sast.Id(_), Sast.Int) ->
      let se2 = check_expr env e2 in
      (match se2 with
         Sast.Expr(_, Sast.Int) ->
           let se3 = check_expr env e3 in
           (match se3 with
              Sast.Expr(_, Sast.Int) ->
              let se4 = check_expr env e4 in
              (match se4 with
                 Sast.Expr(_, t) ->
                 (match t with
                    Sast.Int ->
                      let ss = check_stmt env s in
                      Sast.For(se1, se2, se3, se4, ss)
                  | Sast.Void ->
                      let ss = check_stmt env s in
                      Sast.For(se1, se2, se3, Sast.Expr(Sast.Lit_int(1), Sast.Int), ss)
                | _ -> stmt_error 1))
            | _ -> stmt_error 1)
        | _ -> stmt_error 1)
    | _ -> stmt_error 1

and check_while e s env =
  let se = check_expr env e in
    match se with
      Sast.Expr(Sast.Binop(_, op, _), Sast.Int) ->
        (match op with
          Ast.Eq | Ast.Neq | Ast.Lt | Ast.Gt | Ast.Leq | Ast.Geq ->
            let ss = check_stmt env s in
              Sast.While(se, ss)
          | _ -> stmt_error 2)
    | _ -> stmt_error 2

and check_stmt env = function
    Ast.Expr(e) -> Sast.Sexpr(check_expr env e)
  | Ast.Block(l) -> check_block l env
  | Ast.If(e, s1, s2) -> check_if e s1 s2 env
  | Ast.For(e1, e2, e3, e4, s) -> check_for e1 e2 e3 e4 s env
  | Ast.While(e, s) -> check_while e s env
  | Ast.BreakCont(t) -> Sast.BreakCont(t)

and vdecl_to_sdecl vdecl =
  match vdecl.typ with
      Ast.Int -> { styp = Sast.Int; sname = vdecl.name; builtinv = false; }
    | Ast.Float -> { styp = Sast.Float; sname = vdecl.name; builtinv = false; }
    | Ast.Comp -> { styp = Sast.Comp; sname = vdecl.name; builtinv = false; }
    | Ast.Mat -> { styp = Sast.Mat; sname = vdecl.name; builtinv = false; }

and formal_to_sformal scope formal_param  =
  let found = var_exists formal_param.name scope in
  if found then var_decl_error formal_param.name
  else let sdecl =  vdecl_to_sdecl formal_param in
  let new_formals = sdecl :: scope.formal_param in
  let new_scope =
    { ret_typ = scope.ret_typ;
      ret_nam = scope.ret_nam;
      func_nam = scope.func_nam;
      formal_param = new_formals;
      local = scope.local; 
      builtin = scope.builtin; } in
  new_scope

and formals_to_sformals scope formal_params =
  let new_scope =
    if (formal_params = []) then scope
    else List.fold_left formal_to_sformal scope (List.rev formal_params) in
    new_scope

and local_to_slocal scope local =
  let found = var_exists local.name scope in
  if found then var_decl_error local.name
  else let sdecl = vdecl_to_sdecl local in
  let new_locals = sdecl :: scope.local in
  let new_scope =
    { ret_typ = scope.ret_typ;
      ret_nam = scope.ret_nam;
      func_nam = scope.func_nam;
      formal_param = scope.formal_param;
      local = new_locals; 
      builtin = scope.builtin; } in
  new_scope

and locals_to_slocals scope locals =
  let new_scope =  List.fold_left local_to_slocal scope (List.rev locals) in
  new_scope

and ret_to_sret scope ret_typ =
  let sret_typ = 
    match ret_typ with
        Ast.Int -> Sast.Int
      | Ast.Float -> Sast.Float
      | Ast.Comp -> Sast.Comp
      | Ast.Mat -> Sast.Mat
    in
    let new_scope =
      { ret_typ = sret_typ;
        ret_nam = scope.ret_nam;
        func_nam = scope.func_nam;
        formal_param = scope.formal_param;
        local = scope.local;
        builtin = scope.builtin; } in
    new_scope

and rname_to_srname scope ret_name =
  let new_scope = { ret_typ = scope.ret_typ;
                    ret_nam = ret_name;
                    func_nam = scope.func_nam;
                    formal_param = scope.formal_param;
                    local = scope.local; builtin = scope.builtin; } in
  new_scope

and fname_to_sfname scope func_name =
  let new_scope = { ret_typ = scope.ret_typ;
                    ret_nam = scope.ret_nam;
                    func_nam = func_name;
                    formal_param = scope.formal_param;
                    local = scope.local;
                    builtin = scope.builtin; } in
  new_scope

and ret_to_slocal scope name typ =
  let vdecl = { typ = typ; name = name; } in
  let sdecl =  vdecl_to_sdecl vdecl in
  let new_locals =  sdecl :: scope.local in
  let new_scope ={ ret_typ = scope.ret_typ; 
                   ret_nam = scope.ret_nam;
                   func_nam = scope.func_nam;
                   formal_param = scope.formal_param;
                   local = new_locals;
                   builtin = scope.builtin; } in
  new_scope

and fdecl_to_sdecl fdecl env = 
  let new_scope = ret_to_slocal env.scope fdecl.ret_name fdecl.ret_typ in
  let new_scope = formals_to_sformals new_scope fdecl.formal_params in
  let new_scope = locals_to_slocals new_scope fdecl.locals in
  let new_scope = ret_to_sret new_scope fdecl.ret_typ in
  let new_scope = rname_to_srname new_scope fdecl.ret_name in
  let new_scope = fname_to_sfname new_scope fdecl.func_name in
  let new_env = { scope = new_scope; functions = env.functions; } in
  let stmts = List.map (fun stmt -> check_stmt new_env stmt) fdecl.body in
  { sret_typ = new_scope.ret_typ;
    sret_name = new_scope.ret_nam;
    sfunc_name = new_scope.func_nam;
    sformal_params = new_scope.formal_param;
    slocals = new_scope.local;
    sbody = stmts;
    builtinf = false; }

and check_function env fdecl =
  let found = func_exists fdecl.func_name env in
  if found then func_decl_error fdecl.func_name
  else let sfdecl = fdecl_to_sdecl fdecl env in
  let new_env = { scope = env.scope; functions = sfdecl :: env.functions; } in
  new_env

and check_compute_fdecl fdecls =
  let fdecl = List.hd (List.rev fdecls) in 
  let name = fdecl.func_name in
  if (name = "compute") then fdecls
  else program_error 0

and check_program fdecls =
  let fdecls = check_compute_fdecl fdecls in
  let env = List.fold_left check_function root_environment fdecls in
  let sfdecls = List.rev env.functions in
  sfdecls
\end{lstlisting}


\section{SAST}
sast.ml
\begin{lstlisting}
(* Sankalpa Khadka *)
open Ast

type sdata_type =
    Int
  | Float
  | Comp
  | Mat
  | Poly
  | Void

type expr_wrapper = 
    Expr of sexpr * sdata_type

and  sexpr =
    Lit_int of int
  | Lit_float of float
  | Lit_comp of float * float
  | Lit_qub of string * int
  | Mat of expr_wrapper list list
  | Id of string
  | Unop of Ast.un_op * expr_wrapper
  | Binop of expr_wrapper * Ast.bi_op * expr_wrapper
  | Assign of string * expr_wrapper
  | Call of string * expr_wrapper list
  | Noexpr

and sstmt =
    Sexpr of expr_wrapper
  | Block of sstmt list
  | If of expr_wrapper * sstmt * sstmt
  | For of expr_wrapper * expr_wrapper * expr_wrapper * expr_wrapper * sstmt
  | While of expr_wrapper * sstmt
  | BreakCont of int
 
and svar_decl = 
  { 
    styp : sdata_type;
    sname : string;
    builtinv : bool;
  }

and sfunc_decl = 
  {
    sret_typ : sdata_type;
    sret_name : string;
    sfunc_name : string;
    sformal_params : svar_decl list;
    slocals : svar_decl list;
    sbody : sstmt list;
    builtinf : bool;
  }

type sprogram =
  sfunc_decl list

(* Prety Printer *)
let rec string_of_unop op e =
  (match op with
  Neg -> " -"
  | Not -> " ! "
  | Re -> " Re "
  | Im -> " Im "
  | Norm -> " Norm "
  | Trans -> " Trans "
  | Det -> " Det "
  | Adj -> " Adj "
  | Conj -> " Conj "
  | Unit -> " Unit "
  | Sin -> " Sin "
  | Cos -> " Cos "
  | Tan -> " Tan ") ^ string_of_expr_wrapper e

and string_of_binop e1 op e2 =
  string_of_expr_wrapper e1 ^ 
    (match op with 
      Add -> " + "    | Sub -> " - "     | Mult -> " * " 
      | Div -> " / "    | Mod -> " % "     | Expn -> " ^ " | Tens -> " @ "
      | Eq-> " == "     | Neq -> " != "    | Lt -> " < "
      | Leq -> " <= "   | Gt -> " > "      | Geq -> " >= "
      | Xor -> " XOR "  | And -> " && "    | Or -> " || ") ^ string_of_expr_wrapper e2

and string_of_mat l =
  let row_strs = 
    List.map string_of_row l
  in
    "[" ^ String.concat "" row_strs ^ "]"

and string_of_row r =
  let row_str = 
    String.concat "," (List.map string_of_expr_wrapper r)
  in
    "(" ^ row_str ^ ")"

and string_of_sexpr = function
    Lit_int(i) -> string_of_int i
    | Lit_float(f) -> string_of_float f
    | Lit_comp(f1, f2) -> string_of_float f1 ^ " + " ^ string_of_float f2 ^ "i"
    | Lit_qub(i, t) -> i
    | Mat(l) ->  string_of_mat l
    | Id(s) -> s
    | Unop(op, e) -> string_of_unop op e
    | Binop(e1, op, e2) -> string_of_binop e1 op e2
    | Assign(name, e) -> name ^ " = " ^ string_of_expr_wrapper e
    | Call(name, params) -> "Calling " ^ name ^ " on " ^ string_of_sexprs params
    | Noexpr -> "noexpr"

and string_of_expr_wrapper w =
  let sexpr =
    match w with
        Expr(Lit_int(i), Int) -> Lit_int(i)
      | Expr(Lit_float(f), Float) -> Lit_float(f)
      | Expr(Lit_comp(f1, f2), Comp) -> Lit_comp(f1, f2)
      | Expr(Mat(l), Mat) -> Mat(l)
      | Expr(Id(name), typ) -> Id(name)
      | Expr(Unop(op, e), _) -> Unop(op, e) 
      | Expr(Binop(e1, op, e2), _) -> Binop(e1, op, e2)
      | Expr(Assign(name, e), t1) -> Assign(name, e)
      | Expr(Call(name, params), _) -> Call(name, params)
      | Expr(Lit_qub(i, t), _) -> Lit_qub(i, t)
      | _ -> Noexpr
    in
      string_of_sexpr sexpr

and string_of_svar_decl svar_decl = 
  "svdecl: styp: " ^ 
    (match svar_decl.styp with
      Int -> "int," ^ " name: " ^ svar_decl.sname ^ "  "
    | Float -> "float," ^ " name: " ^ svar_decl.sname ^ "  "
    | Comp -> "comp," ^ " name: " ^ svar_decl.sname ^ "  "
    | Mat -> "mat," ^ " name: " ^ svar_decl.sname ^ "  "
    | _ -> "")

and string_of_sexprs e = 
  String.concat "\n" (List.map string_of_expr_wrapper e)

and string_of_sstmt = function
    Sexpr(e) -> string_of_expr_wrapper e ^ "\n"
  | Block(l) -> "{\n" ^ string_of_sstmts l ^ "\n}"
  | If(e, s, Block([])) -> "if (" ^ string_of_expr_wrapper e ^ ")\n" ^ string_of_sstmt s
  | If(e, s1, s2) ->  "if (" ^ string_of_expr_wrapper e ^ ")\n" ^string_of_sstmt s1 ^ "else\n" ^ string_of_sstmt s2
  | For(e1, e2, e3, e4, s) -> "For args : " ^ string_of_expr_wrapper e1 ^ " " ^ string_of_expr_wrapper e2 ^ " "^ string_of_expr_wrapper e3 ^ 
                                 " "^ string_of_expr_wrapper e4 ^ "\nstatement :\n" ^ string_of_sstmt s 
  | While(e,s) -> "While condition : " ^ string_of_expr_wrapper e ^ "\nstatement : " ^ string_of_sstmt s
  | BreakCont(t) -> string_of_breakcont t

and string_of_breakcont t =
  if (t = 0) then
  "break"
  else
  "continue"

and string_of_sstmts sstmts = 
  String.concat "\n" (List.map string_of_sstmt sstmts) 

and string_of_sfdecl sfdecl =
  "\nsfdecl:\nsret_typ: " ^ 
    (match sfdecl.sret_typ with
      Int -> " int "
    | Float -> " float "
    | Comp -> " comp "
    | Mat -> " mat "
    | _ -> "") ^
      "\nsret_name: " ^ sfdecl.sret_name ^ "\nsfunc_name: "  ^ sfdecl.sfunc_name ^  "\n(" ^
        String.concat "" (List.map string_of_svar_decl sfdecl.sformal_params) ^ ")\n{\n" ^
          String.concat "" (List.map string_of_svar_decl sfdecl.slocals) ^ "\n" ^
            String.concat "" (List.map string_of_sstmt sfdecl.sbody) ^ "}"
 
and string_of_sprogram (l) = 
  "program:\n" ^ String.concat "\n" (List.map string_of_sfdecl l)
\end{lstlisting}

\section{Generator}
generator.ml
\begin{lstlisting}
(* Winnie Narang, Jonathan Wong, Sankalpa Khadka *)
open Sast
open Printf
open String

let builtin_funcs = ["print";"printq";"rows";"cols";"elem"]

let is_builtin_func name =
  List.exists (fun func_name -> func_name = name) builtin_funcs

(* get type *)
let type_of (a : Sast.expr_wrapper) : Sast.sdata_type =
    match a with
    | Expr(_,t)-> t

(* get expression from expression wrapper *)
let expr_of (a : Sast.expr_wrapper) : Sast.sexpr =
    match a with
    | Expr(e,_)-> e

(* generate type *)
let rec cpp_from_type (ty: Sast.sdata_type) : string =
    match ty with
    | Int -> "int"
    | Float -> "float"
    | Comp -> "complex<float>"
    | Mat -> "MatrixXcf"
    | Poly | Void -> " "

(* write program to .cpp file *)
and writeToFile fileName progString =
    let file = open_out (fileName ^ ".cpp") in
        fprintf file "%s" progString 

(* entry point for code generation*)        
and gen_program fileName prog =
    let cppString = writeCpp prog in
    let out = sprintf "
        #include <iostream>
        #include <complex>
        #include <cmath>
        #include <Eigen/Dense>
        #include <qlang>
        using namespace Eigen;
        using namespace std;
        %s" cppString in 
    writeToFile fileName out;

(* list of function declaration*)    
and writeCpp funcList =
    let outStr =
      List.fold_left (fun a b -> a ^ (cpp_funcList b)) "" funcList
    in
      sprintf "%s" outStr

(* generate functions *)    
and cpp_funcList func =
    if func.builtinf then
      ""
    else
      let cppFName = func.sfunc_name
      and cppRtnType = cpp_from_type func.sret_typ
      and cppRtnValue = func.sret_name
      and cppFParam = if (func.sformal_params = []) then "" else cppVarDecl func.sformal_params ","
      and cppFBody = cppStmtList func.sbody 
      and cppLocals = cppVarDecl func.slocals ";\n\t"
        in
      if cppFName = "compute" then
                  sprintf "\nint main ()\n{\n\t%s\n\t%s\n\tstd::cout << %s << endl;\n\n\treturn 0;\n}" cppLocals cppFBody cppRtnValue
      else 
        if (cppFParam = "") then
        sprintf "\n%s %s ()\n{\n\t%s\n%s\n\treturn %s;\n}" cppRtnType cppFName cppLocals cppFBody cppRtnValue
        else
        sprintf "\n%s %s (%s)\n{\n\t%s\n%s\n\treturn %s;\n}" cppRtnType cppFName cppFParam cppLocals cppFBody cppRtnValue

(* generate variable declarations *)
and cppVarDecl vardeclist delim =
   let varDecStr =
    List.fold_left (fun a b -> a ^ (cppVar b delim)) "" vardeclist
  in
    let varDectrun = String.sub varDecStr 0 ((String.length varDecStr)-1)
  in
    sprintf "%s " varDectrun

(* generate variable declaration *)   
and cppVar var delim =
    if not var.builtinv then
        let vartype =
          cpp_from_type var.styp
        in 
          sprintf "%s %s%s" vartype var.sname delim
    else ""

(* generate list of statements *)
and cppStmtList astmtlist =    
    let outStr =
      List.fold_left (fun a b -> a ^ (cppStmt b)) "" astmtlist
    in
      sprintf "%s" outStr

(* generate statement *)
and cppStmt stmts = match stmts with
    Sast.Sexpr(expr_wrap) -> "\t" ^ cppExpr (expr_of expr_wrap) ^ ";\n"  
  | Sast.Block(sstmt) -> cppStmtBlock sstmt
  | Sast.If(expr_wrap , sstmt1, sstmt2) -> writeIfStmt (expr_of expr_wrap) sstmt1 sstmt2
  | Sast.For(var,init, final, increment, stmt) -> writeForStmt var init final increment stmt
  | Sast.While(expr_wrap , sstmt) -> writeWhileStmt (expr_of expr_wrap) sstmt
  | Sast.BreakCont(t) -> writeBreakCont t

(* generate break/continue statement *)
and writeBreakCont t =
  if (t =0) then
  sprintf "break;"
  else
  sprintf "continue;"

(* generate expression *)
and cppExpr expr = match expr with
    Lit_int(lit) -> string_of_int lit
  | Lit_float(flit) -> string_of_float flit
  | Lit_comp(re,im) -> " complex<float>(" ^ string_of_float re ^ "," ^ string_of_float im  ^ ") " (* Not sure how to do this *)
  | Unop(op, expr) ->  writeUnop op expr
  | Binop(expr1, op, expr2) -> writeBinop expr1 op expr2
  | Lit_qub(vec, t) -> writeQubit vec t
  | Mat (expr_wrap) -> writeMatrix expr_wrap
  | Id(str) -> str 
  | Assign(name, expr) ->  name  ^ " = " ^ cppExpr (expr_of expr)
  | Call(name,l) ->
      if is_builtin_func name then
        writeBuiltinFuncCall name l
      else
        name ^ "(" ^ writeFunCall l ^ ")"    
  | Noexpr -> ""

(* generate built-in function call *)
and writeBuiltinFuncCall name l =
  match name with
    "print" -> writePrintStmt l
  | "printq" -> writePrintqStmt l
  | "rows" -> writeRowStmt l
  | "cols" -> writeColStmt l
  | "elem" -> writeElemStmt l
  | _ -> ""

(* generate row statement *)
and writeRowStmt l =
  let expr_wrap = List.hd l in
  let expr = cppExpr (expr_of expr_wrap) in 
  sprintf "%s.rows()" expr

(* generate col statement *)
and writeColStmt l =
  let expr_wrap = List.hd l in
  let expr = cppExpr (expr_of expr_wrap) in 
  sprintf "%s.cols()" expr

(* generate elem statement *)
and writeElemStmt l =
  let ew1 = List.hd l in
  let e1 = cppExpr (expr_of ew1)
  and ew2 = List.hd (List.tl l) in
  let e2 = cppExpr (expr_of ew2)
  and ew3 = List.hd (List.tl (List.tl l)) in
  let e3 = cppExpr (expr_of ew3) in
  sprintf "%s(%s,%s)" e1 e2 e3
 
(* generate print statement *)
and writePrintStmt l =
  let expr_wrap = List.hd l in
  let expr = cppExpr (expr_of expr_wrap) in 
    match expr_wrap with
      Sast.Expr(_,t) -> 
        (match t with 
            Sast.Mat -> sprintf "cout << %s << endl" expr
          | _ -> sprintf "cout << %s << endl" expr)

(* generate qubit print statement *)
and writePrintqStmt l =
  let expr_wrap = List.hd l in
  let expr = cppExpr (expr_of expr_wrap) in 
    match expr_wrap with
        Sast.Expr(_,t) -> 
          (match t with 
            Sast.Mat -> sprintf "cout << vectorToBraket(%s) << endl" expr
          | _ -> sprintf "cout << %s << endl" expr)

(* generate block *)  
and cppStmtBlock sstmtl = 
let slist = List.fold_left (fun output element ->
    let stmt = cppStmt  element in
    output ^ stmt ^ "\n") "" sstmtl in
    "\n\t{\n" ^ slist ^ "\t}\n"

(* generate if statement *)
and writeIfStmt expr stmt1 stmt2 = 
  let cond = cppExpr expr in
    let body = cppStmt stmt1 in
    let ebody = writeElseStmt stmt2 in
    sprintf " if(%s)%s%s" cond body ebody  

(* generate else statements *)
and writeElseStmt stmt =
    let body =
        cppStmt stmt
    in
        if ((String.compare body "\t;\n") = 0) then
            sprintf "\n"
        else
            sprintf "\telse%s" body

(* generate while statement *)
and writeWhileStmt expr stmt = 
let condString = cppExpr expr  
  and stmtString = cppStmt stmt in 
    sprintf "while (%s)\n%s\n" condString stmtString

(* generate for statements *)
and writeForStmt var init final increment stmt =
    let varname = cppExpr (expr_of var) 
    and initvalue = cppExpr (expr_of init)
    and finalvalue = cppExpr (expr_of final)
    and incrementval = cppExpr (expr_of increment)
    and stmtbody = cppStmt stmt
    in
    sprintf "
    for (int %s = %s; %s < %s; %s = %s + %s){
        %s
        }" varname initvalue varname finalvalue varname varname incrementval stmtbody

(* generate unary operators *)
and writeUnop op expr = 
    let exp = cppExpr (expr_of expr) in 
        let unopFunc op exp = match op with
          Ast.Neg   -> sprintf "  -%s" exp
        | Ast.Not   -> sprintf "  !(%s)" exp
        | Ast.Re    -> sprintf "  real(%s)" exp   (* assumes exp is matrix*)
        | Ast.Im    -> sprintf "  imag(%s)" exp
        | Ast.Norm  -> sprintf "  %s.norm()" exp
        | Ast.Trans -> sprintf "  %s.transpose()" exp
        | Ast.Det   -> sprintf "  %s.determinant()" exp
        | Ast.Adj   -> sprintf "  %s.adjoint()" exp
        | Ast.Conj  -> sprintf "  %s.conjugate()" exp
        | Ast.Unit  -> sprintf "  (%s.conjugate()*%s).isIdentity()" exp exp     (* till here *)
        | Ast.Sin   -> sprintf "  sin((double)%s)" exp
        | Ast.Cos   -> sprintf "  cos((double)%s)" exp
        | Ast.Tan   -> sprintf "  tan((double)%s)" exp
    in unopFunc op exp

(* generate binary operations *)
and writeBinop expr1 op expr2 = 
    let e1 = cppExpr (expr_of expr1) 
    and t1 = type_of expr1  
    and e2 = cppExpr (expr_of expr2) in 
      let binopFunc e1 t1 op e2 = match op with 
     Ast.Add  -> sprintf "%s + %s" e1 e2
    | Ast.Sub   -> sprintf "%s - %s" e1 e2
    | Ast.Mult  -> sprintf "%s * %s" e1 e2
    | Ast.Div   -> sprintf "%s / %s" e1 e2
    | Ast.Mod   -> sprintf "%s %% %s" e1 e2
    | Ast.Expn  -> sprintf "pow(%s,%s)" e1 e2
    | Ast.Tens  -> sprintf "tensor(%s, %s)" e1 e2
    | Ast.Eq  -> equalCaseWise e1 t1 e2
    | Ast.Neq   -> sprintf "%s != %s" e1 e2
    | Ast.Lt  -> sprintf "%s < %s" e1 e2
    | Ast.Gt  -> sprintf "%s > %s" e1 e2
    | Ast.Leq   -> sprintf "%s <= %s" e1 e2
    | Ast.Geq   -> sprintf "%s >= %s" e1 e2
    | Ast.Or  -> sprintf "%s || %s" e1 e2
    | Ast.And   -> sprintf "%s && %s" e1 e2
    | Ast.Xor   -> sprintf "%s ^ %s" e1 e2
  in binopFunc e1 t1 op e2 

(* generate equality expressions (structural equality is used) *)
and equalCaseWise e1 t1 e2 = match t1 with
       Sast.Mat -> sprintf "%s.isApprox(%s)" e1 e2
      | _ -> sprintf "%s == %s" e1 e2        

(* generate matrix *)
and writeMatrix expr_wrap = 
    let matrixStr = List.fold_left (fun a b -> a ^ (writeRow b)) "" expr_wrap in
    let submatrix = String.sub matrixStr 0 ((String.length matrixStr)-1) in
    sprintf "(Matrix<complex<float>, Dynamic, Dynamic>(%d,%d)<<%s).finished()" (rowMatrix expr_wrap) (colMatrix expr_wrap) submatrix

(* generate matrix row *)
and writeRow row_expr =
    let rowStr = List.fold_left (fun a b -> a ^ (cppExpr (expr_of b)) ^ "," ) "" row_expr in
    sprintf "%s" rowStr

(* generate column matrix *)
and colMatrix expr_wrap =
  List.length (List.hd expr_wrap)

(* generate row matrix *)
and rowMatrix expr_wrap =
  List.length expr_wrap

(* generate function call *)
and writeFunCall expr_wrap =
    if expr_wrap = [] then
    sprintf ""
    else
    let argvStr = List.fold_left (fun a b -> a ^ (cppExpr (expr_of b)) ^ ",") "" expr_wrap in
    let argvStrCom = String.sub argvStr 0 ((String.length argvStr)-1) in
    sprintf "%s" argvStrCom

(* generate qubits *)
and writeQubit expr bra=
   (* let exp = string_of_int expr in *)
    sprintf "genQubit(\"%s\",%d)" expr bra
\end{lstlisting}


\section{Scripts}


\subsection{Makefile}
Makefile    
\begin{lstlisting}
#Christopher Campbell, Jonathan Wong
#stuff for compiling cpp files
CXX = g++
CPPDIR = ./cpp
INC = $(CPPDIR) ./includes/headers
INCLUDES =$(INC:%=-I%)
CXXFLAGS = -g -Wall $(INCLUDES)

OBJS = ast.cmo sast.cmo parser.cmo scanner.cmo analyzer.cmo generator.cmo qlc.cmo

.PHONY: default

default: qlc cpp/qlang.o


qlc : $(OBJS) 
	ocamlc -g -o qlc $(OBJS)

scanner.ml : scanner.mll
	ocamllex scanner.mll

parser.ml parser.mli : parser.mly
	ocamlyacc parser.mly

%.cmo : %.ml
	ocamlc -g -c $<

%.cmi : %.mli
	ocamlc -g -c $<

cpp/qlang.o:
	$(MAKE) -C $(CPPDIR)


.PHONY : clean
clean :
	rm -f qlc parser.ml parser.mli scanner.ml *.cmo *.cmi
	$(MAKE) -C $(CPPDIR) clean

# Generated by ocamldep *.ml *.mli
analyzer.cmo: sast.cmo ast.cmo
analyzer.cmx: sast.cmx ast.cmx
generator.cmo: sast.cmo
generator.cmx: sast.cmx
parser.cmo: ast.cmo parser.cmi 
parser.cmx: ast.cm parser.cmi 
qlc.cmo: scanner.cmo sast.cmo parser.cmi ast.cmo analyzer.cmo
qlc.cmx: scanner.cmx sast.cmo parser.cmx ast.cmx analyzer.cmx
sast.cmo: ast.cmo
sast.cmx: ast.cmx
scanner.cmo: parser.cmi 
scanner.cmx: parser.cmx 
parser.cmi: ast.cmo
\end{lstlisting}

     \subsection{Compilation script}
qlc.ml

\begin{lstlisting}
(* Christopher Campbell, Winnie Narang *)
type action = Ast | Sast | Gen | Debug

let _ =
  let action =
    List.assoc Sys.argv.(1) [("-a", Ast); ("-s", Sast); ("-g", Gen); ("-d", Debug);]
  in
    let lexbuf = Lex
    ing.from_channel (open_in Sys.argv.(2)) (*stdin *) and
    output_file = String.sub Sys.argv.(2) 0 (String.length(Sys.argv.(2))-3) in
      let program = Parser.program Scanner.token  lexbuf in
        match action with 
          Ast ->  print_string (Ast.string_of_program program)
          | Sast -> 
            let sprogram =
              Analyzer.check_program program
            in
              print_string (Sast.string_of_sprogram sprogram)
          | Gen -> Generator.gen_program output_file (Analyzer.check_program program)
          | Debug -> print_string "debug"

\end{lstlisting}


\subsection{Testing script}
runTests.sh
\begin{lstlisting}
#Christopher Campbell, Winnie Narang
#!/bin/bash

AST=0
SAST=0
GEN=0
COMP=0
EXEC=0


if [ $1 == "clean" ]
then
rm -f ast_error_log sast_error_log gen_error_log comp_error_log ast_log sast_log ast_output sast_output exec_output
rm -f SemanticSuccess/*.cpp SemanticSuccess/*.o
rm -f SemanticFailure/*.cpp SemanticFailure/*.o
rm -f Analyzer/*.cpp Analyzer/*.o
else

if [ $1 == "a" ]
then
AST=1
fi
if [ $1 == "s" ]
then
SAST=1
fi
if [ $1 == "g" ] || [ $1 == "c" ] || [ $1 == "e" ]
then
GEN=1
fi
if [ $1 == "c" ]
then
COMP=1
fi
if [ $1 == "e" ]
then
EXEC=1
fi

if [ $2 == "ss" ]
then
files="SemanticSuccess/*.ql"
cfiles="SemanticSuccess/*.cpp"
elif [ $2 = "sf" ]
then
files="SemanticFailures/*.ql"
cfiles="SemanticFailures/*.cpp"
elif [ $2 = "al" ]
then
files="Algorithms/*.ql"
cfiles="Algorithms/*.cpp"
fi

ASTCheck()
{
    eval "../qlc -a $1" 1>> ast_output 2>> ast_error_log
    wc ast_error_log | awk '{print $1}'
}

SASTCheck()
{
    eval "../qlc -s $1" 1>> sast_output 2>> sast_error_log
    wc sast_error_log | awk '{print $1}'
}

GenerationCheck()
{
    eval "../qlc -g $1" 2>> gen_error_log
    wc gen_error_log | awk '{print $1}'
}

CompilationCheck()
{
    eval "g++ -w $1 -I../includes/headers -L../includes/libs -lqlang" 2>> comp_error_log
    wc comp_error_log | awk '{print $1}'
}

ExecutionCheck()
{
    output=$(eval "./a.out")
    echo " " >> exec_output
    echo "Output: " >> exec_output
    echo "$output" >> exec_output
    echo "$output"
}

#Check AST
if [ $AST == 1 ]
then
echo "* AST Generation *"
rm -f ast_error_log ast_output
errors=0
prev_errors=0
for file in $files
do
errors=0
errors=$(ASTCheck $file)
if [ "$errors" -le "$prev_errors" ]
then
count=1
echo "Pass " $file
else
echo "Fail " $file
fi
prev_errors=$errors
done
echo ""
fi

#Check SAST
if [ $SAST == 1 ]
then
echo "* SAST Generation *"
rm -f sast_error_log sast_output
errors=0
prev_errors=0
for file in $files
do
errors=$(SASTCheck $file)
if [ "$errors" -le "$prev_errors" ]
then
echo "Pass: " $file
else
echo "Fail: " $file
fi
prev_errors=$errors
done
echo ""
fi

#Check Generation
if [ $GEN == 1 ]
then
cd ../cpp
make
cd ../test
echo "* Code Generation *"
rm -f gen_error_log
errors=0
prev_errors=0
for file in $files
do
errors=$(GenerationCheck $file)
if [ "$errors" -le "$prev_errors" ]
then
echo "Pass: " $file
else
echo "Fail: " $file
fi
prev_errors=$errors
done
echo ""
fi

#Check Compilation
if [ $COMP == 1 ]
then
echo "* Compilation *"
rm -f comp_error_log
errors=0
prev_errors=0
for file in $cfiles
do
errors=$(CompilationCheck $file)
if [ "$errors" -le "$prev_errors" ]
then
echo "Pass: " $file
else
echo "Fail: " $file
fi
prev_errors=$errors
done
echo ""
fi

# Execution check
if [ $EXEC == 1 ]
then
echo "* Compilation and Execution *"
rm -f comp_error_log exec_output
errors=0
prev_errors=0
exec_output=0
for file in $cfiles
do
errors=$(CompilationCheck $file)
if [ "$errors" -le "$prev_errors" ]
then
echo "Pass (compilation): " $file
exec_output=$(ExecutionCheck)
if [ "$exec_output" != "0" ]
then
echo "Pass (execution): " $file
echo $exec_output
else
echo "Fail (execution): " $file
fi
else
echo "Fail (compilation): " $file
fi
prev_errors=$errors
done
fi

fi
\end{lstlisting}

\section{Programs}

\subsection{Demo}
demo1.ql

\begin{lstlisting}
# Sankalpa Khadka
def compute() : mat output{
        
        mat a;
        mat b;
        mat c;
        mat k;

        a = |11>;
        b = |0>;
        k = <0|;
        
        c = a @ b;
        printq(c);
        
        c = H*b;
        printq(c);

        output = b*k;
}
\end{lstlisting}

demo2.ql
\begin{lstlisting}
# Sankalpa Khadka
def measure(mat top): mat outcome{
        mat ad;
        
        ad = adj(top);
        outcome = top*ad;
}

def outcomezero(mat bottom) : float probability{
        
        mat top;
        mat input;
        mat had;
        mat cnot;
        mat ynot;
        mat output;
        mat meas;
        
        top = |0>;
        input = top @ bottom;
        
        had = H @ IDT;
        cnot = [(1,0,0,0)
                (0,1,0,0)
                (0,0,0,1)
                (0,0,1,0)];
      
        
        ynot = [(1,0,0,0)
                (0,0,0,C(1.0I))
                (0,0,1,0)
                (0,C(-1.I),0,0)];
    
        output = (ynot*(cnot*(had*input)));
        
        printq(output);
        
        probability = norm(output);
   
}

def compute() : float outcome{
        
        mat bottom;
        
        bottom = |1>;
        outcome = outcomezero(bottom);
        print(outcome);
        
        bottom = |0>;
        outcome = outcomezero(bottom);
}

\end{lstlisting}
demo3.ql
\begin{lstlisting}
# Sankalpa Khadka
# simulation of Deutsch's Algorithm
def measure (mat top) : mat outcome{
        # returns the measurement matrix for top qubit
        mat ad;

        ad = adj(top);
        outcome = top * ad;
}

def hadamard (int n) : mat gate{
        # returns Hadamard gate for n qubit system
        int i;
        gate = H;

        for (i from 0 to n-1 by 1){
            gate = gate @ H; 
        }
}

def topqubit (int n) : mat input{
	# return zero qubit for n qubit system
        int i;
        input = |0>;

        for (i from 0 to n-1 by 1){
                input = input @ |0>;
        }          
}

def deutsch (int n, mat U) : float outcomeZero{

        mat bottom; mat top; mat input;
        mat hadtop; mat meas;

        bottom = |1>;					
        top = topqubit(n);
        input = top @ bottom;			# input qubit, tensor of top and bottom
        
        hadtop = hadamard(n);		
        input = (hadtop @ H)*input;		# application of Hadamard gate
        input = U * input;				# application of the Oracle U
        input = (hadtop @ IDT)*input;		# application of Hadamard on top only, IDT-Identity
        meas = measure(top);

        input = (meas @ IDT)* input;		# measure zero on top register
        outcomeZero = norm(input);		# likelihood of getting zero on top register
}


def compute () : float outcome{

        int n; mat Ub; mat Uc;
	
	# test for n =1
        n = 1;
        # Ub - balanced, Uc - Constant Oracles
        Ub = [(1,0,0,0)(0,1,0,0)(0,0,0,1)(0,0,1,0)];
        Uc = [(1,0,0,0)(0,1,0,0)(0,0,1,0)(0,0,0,1)];

        outcome = deutsch(n, Ub);
        print(outcome);
        
        outcome = deutsch(n, Uc);
        print(outcome);

	#test for n=2
        n = 2;
        Ub = [(1,0,0,0,0,0,0,0) 
              (0,1,0,0,0,0,0,0)
              (0,0,1,0,0,0,0,0)
              (0,0,0,1,0,0,0,0) 
              (0,0,0,0,0,1,0,0) 
              (0,0,0,0,1,0,0,0)
              (0,0,0,0,0,0,0,1)
              (0,0,0,0,0,0,1,0)];

        outcome = deutsch(n, Ub);
}

\end{lstlisting}
demo4.ql
\begin{lstlisting}
# Sankalpa Khadka
# Simulation of Grover's Algorithm for f(0) =1
def measure (mat top) : mat outcome{
        # measurement matrix for top qubit
        mat ad;

        ad = adj(top);
        outcome = top * ad;
}

def ntensor (int n, mat k) : mat gate{
        # return k@k@...@k n times
        int i;
        gate = k;

        for (i from 0 to n-1 by 1){
            gate = gate @ k; 
        }
}

def prepareU (int n) : mat gate {
	# Prepare the Uw of grover oracle 
        mat i;
        mat u;

        i = [(1,0)
             (0,0)];

        u = ntensor(n+1, i);
        gate = ntensor(n+1,IDT)-2*u;
}

def prepareG (int n) : mat gate{
	# Prepare grover defusive operator
        mat s; mat sa; mat i; mat h;

        s = ntensor(n,|0>);
        sa = adj(s);
        i = ntensor(n,IDT);
        gate = 2*s*sa - i;
        h = ntensor(n, H);
        gate = h*gate*h;
        gate = gate @ IDT;         
}

def grover (int n) : float outcomeZero{

        mat bottom; mat top; mat input;
        mat hadtop; mat u; mat g; mat go; mat meas;
        int i;

        bottom = |1>;
        top = ntensor(n, |0>);
        input = top @ bottom;	# input is tensor of top and bottom registers
        
        hadtop = ntensor(n, H);
        input = (hadtop @ H)*input;	# apply Hadamard gate
        u = prepareU(n);
        g = prepareG(n);
        
        go = g*u;					# Grover Operator
        
        for (i from 0 to n by 1){		# Apply grover operator over iteration
                input = go*input; 
        }

        meas = measure(top);		
        input = (meas @ IDT)* input;	# measure on top register
        outcomeZero = norm(input);	# likelihood to find 0 on top register
}


def compute () : float outcome{
        #simulate the grover for f(0)=1
        
        int n; mat Ub; mat Uc;
        n = 1;
        
        outcome = grover(n);
        print(outcome);
        
        n = 2;
        outcome = grover(n);
}

\end{lstlisting}

\subsection{Successful Test cases}
binop\_comp\_matrix.ql
\begin{lstlisting}
#Winnie Narang
def test_func(comp a, comp b, comp c, comp d) : mat ret_val { 

  mat x;

  x = [(a,b)(c,d)];

  ret_val = [(a,c)(d,b)];

  ret_val = ret_val * x;
  ret_val = ret_val + x;
  ret_val = ret_val - x;
  ret_val = ret_val / 2;
}

def compute() : mat ret_val {


  comp a;
  comp b;
  comp c;
  comp d;
    mat  k;

  a = C(4.+5.I);
  b = C(6.+6.I);
  c = C(7.+8.I);
  d = C(9.+10.I);

  ret_val = test_func(a, b, c, d);
}
\end{lstlisting}
binop\_float\_matrix.ql
\begin{lstlisting}
#Winnie Narang
def test_func(float a, float b, float c, float d) : mat ret_val { 

  mat x;

  x = [(a,b)(c,d)];

  ret_val = [(a,c)(d,b)];

  ret_val = ret_val * x;
  ret_val = ret_val + x;
  ret_val = ret_val - x;
  ret_val = ret_val / 2;
  
}

def compute() : mat ret_val {


  float a;
  float b;
  float c;
  float d;

  a = 3.4;
  b = 6.;
  c = 5.6;
  d = 100.0;

  ret_val = test_func(a, b, c, d);
}
\end{lstlisting}
binop\_int\_arith.ql
\begin{lstlisting}
#Winnie Narang
def func_test(int z) : int ret_name { 
        int a;
        int b;
        int d;
        a = z;
        b = 10;
        d = a+b*a+b/a-b;
        ret_name=d;
}
def compute( int a ): int trial {
      trial = func_test(34);
}
\end{lstlisting}
binop\_tensor.ql
\begin{lstlisting}
#Jonathan Wong
def compute():mat out {
        
        mat a;
        mat b;
        mat c;

        a = [(1)(0)];
        b = [(0)(1)];
        c= a@b; 
        print(c); 
}
\end{lstlisting}
break\_continue.ql
\begin{lstlisting}
#Winnie Narang
def func_test(int a) : int ret_name { 
        
        int i;

        for(i from 0 to 2 by 1)
        a=a+5;

        for(i from 2 to 0 by -1)
        {
            a=a*10;
            print(a);
            break;
        }

        for(i from 1 to 5)
        {
            print(a);
            continue;
            a=a*10;

        }

    ret_name = a;
}

def compute(): int trial {

   trial = func_test(20);
}
\end{lstlisting}

builtin\_matrix\_ops.ql
\begin{lstlisting}
#Sankalpa Khadka
def compute(): comp trial {
      
  int num_rows;
  int num_cols;
  comp val;
  mat m;

  m = [(1,2,3)(4,5,6)(7,8,9)];
  num_rows = rows(m);
  num_cols = cols(m);
  val = elem(m, 1,2);

  print(num_rows);
  print(num_cols);

  trial = val;
}
\end{lstlisting}
comp\_type.ql
\begin{lstlisting}
#Sankalpa Khadka
def compute(): comp trial {
      
  int num_rows;
  int num_cols;
  comp val;
  mat m;

  m = [(1,2,3)(4,5,6)(7,8,9)];
  num_rows = rows(m);
  num_cols = cols(m);
  val = elem(m, 1,2);

  print(num_rows);
  print(num_cols);

  trial = val;
}
\end{lstlisting}
constants.ql
\begin{lstlisting}
#Jonathan Wong
def test_func(int a) : mat ret_val { 

  mat x;
  mat z;
  mat y;
  mat w;
  
  x = X;
  z = H;
  y = Y;
  w = IDT;

  print(x);
  print(z);
  print(y);
  print(w);

  ret_val = x * z * y * w;
}

def compute() : mat ret_val {

  ret_val = test_func(0);
}
\end{lstlisting}
empty.ql
\begin{lstlisting}
#Christopher Campbell
def test_func() : mat ret_val { 

        mat x;

  x = [(1,2)(3,4)];

  ret_val = x;
}

def compute() : mat ret_val {

  ret_val = test_func();
}

\end{lstlisting}
float\_type.ql
\begin{lstlisting}
#Christopher Campbell
def func_test(float b) : float ret_name { 
        
        float a; 
  float c;

        a = 5.0;
  c = a * b;

        ret_name = c;  
}

def compute() : float trial {

  trial = func_test(3.7);

}
\end{lstlisting}
for\_stmt.ql
\begin{lstlisting}
#Jonathan Wong
def func_test(int z) : int ret_name { 
        
        int i;
        int a;

        for(i from 0 to 2 by 1)
        a=a+5;

        for(i from 2 to 0 by -1)
        {
            a=a*10;
            print(a);
        }
        
        for(i from 1 to 10 by 1)
        {
            a=a-3;
        }

  for(i from 1 to 100){
    print (a*100);
  }

    ret_name = 5;
}

def compute(int a): int trial {

   trial = func_test(20);
}
\end{lstlisting}
if\_stmt.ql
\begin{lstlisting}
#Winnie Narang
def func_test(int z) : int ret_name { 
      
       int a;

      # comment before b; just checking for end of comment being correct

       int b;
       a = 10;
       
       if(z eq 5) a = 0;
       
       a = a - 2;
       if( z leq 5 )
       {
          a = 0;
       }
       else
       {
           a = 10;
           b = 24;
       }
       
       if( a gt 100 )
       {
           print(b); # a > 100
       }
       else 
       {
          print(a);
       }

       ret_name = 8;
}
\end{lstlisting}
mat\_add.ql
\begin{lstlisting}
#Sankalpa Khadka
def test_func(comp a, comp b, comp c, comp d) : mat ret_val { 

  mat x;

  x = [(a,b)(c,d)];

  ret_val = x;
}

def compute():mat trial {
  comp a;
  comp b;
  comp c;
  comp d;
        mat k;

  a = C(2.);
  b = C(2.);
  c = C(2.);
  d = C(2.);

  trial = test_func(a, b, c, d)+test_func(a,b,c,d);

}
\end{lstlisting}
mat\_mult.ql
\begin{lstlisting}
#Winnie Narang
def test_func(comp a, comp b, comp c, comp d) : mat ret_val { 

  mat x;

  x = [(a,b)(c,d)];

  ret_val = x;
}

def compute():mat trial {
  comp a;
  comp b;
  comp c;
  comp d;
        mat k;

  a = C(2.);
  b = C(2.);
  c = C(2.);
  d = C(2.);

  trial = test_func(a, b, c, d)*test_func(a,b,c,d);

}
\end{lstlisting}
mat\_qubit.ql
\begin{lstlisting}
#Winnie Narang
def func_test(mat a, mat b) : mat ret_name { 
        
        ret_name = a*b;

}


def compute(int a):mat trial {
  
   mat zero;
   mat one;

   zero = |0>;
   one = |1>;

     trial = func_test(H,zero);
     printq(trial);

     trial = func_test(H,one);
     printq(trial);

}
\end{lstlisting}

un\_op\_det.ql
\begin{lstlisting}
#Winnie Narang
def func_test(mat z) : mat ret_name { 
        mat a;
        comp b;
        a = [(1,9)(4,5)];
        b = det(a);
        ret_name = a;
}
def compute(int a):mat trial {
  mat x;
  x = [(1,2)(3,4)];
  trial = func_test(x);
}
\end{lstlisting}
un\_op\_trans.ql
\begin{lstlisting}
def func_test(mat z) : mat ret_name { 
        mat a;
        mat b;
        a=[(1,9,9)(4,5,5)];
        b = trans(a);
}
def compute(int a):int trial {
  trial = 8;
}
\end{lstlisting}
while\_stmt.ql
\begin{lstlisting}
#Winnie Narang
def func_test(int z) : int ret_name { 
        int a; 
        a = 5; 
        

        #now checking while with comment
        while(a leq 10)
        a=a+1;
        
        while(a neq 1)
        {
            # Comment, inside
            a = (a+1) % 42;
        }

        ret_name = a;


}
def compute():int trial{

trial = func_test(5);
}
\end{lstlisting}
\subsection{Execution output of successful cases}
exec\_output 
\begin{lstlisting}
Output: 
(-12,76) (-11.5,98)
(-10,87.5)   (-6,114)
 
Output: 
(21.46,0) (290.2,0)
(186.8,0)   (600,0)
 
Output: 
364
 
Output: 
2
 
Output: 
(0,0)
(1,0)
(0,0)
(0,0)
 
Output: 
30
30
30
30
30
 
Output: 
3
3
(6,0)
 
Output: 
(3.52,8.6)
 
Output: 
(0,0) (1,0)
(1,0) (0,0)
 (0.707107,0)  (0.707107,0)
 (0.707107,0) (-0.707107,0)
  (0,0) (-0,-1)
  (0,1)   (0,0)
(1,0) (0,0)
(0,0) (1,0)
(0,-0.707107) (0,-0.707107)
 (0,0.707107) (0,-0.707107)
 
Output: 
0
1
0
 
Output: 
(1,0) (2,0)
(3,0) (4,0)
 
Output: 
18.5
 
Output: 
3275000
3275000
3275000
3275000
3275000
3275000
3275000
3275000
3275000
3275000
3275000
3275000
3275000
3275000
3275000
3275000
3275000
3275000
3275000
3275000
3275000
3275000
3275000
3275000
3275000
3275000
3275000
3275000
3275000
3275000
3275000
3275000
3275000
3275000
3275000
3275000
3275000
3275000
3275000
3275000
3275000
3275000
3275000
3275000
3275000
3275000
3275000
3275000
3275000
3275000
3275000
3275000
3275000
3275000
3275000
3275000
3275000
3275000
3275000
3275000
3275000
3275000
3275000
3275000
3275000
3275000
3275000
3275000
3275000
3275000
3275000
3275000
3275000
3275000
3275000
3275000
3275000
3275000
3275000
3275000
3275000
3275000
3275000
3275000
3275000
3275000
3275000
3275000
3275000
3275000
3275000
3275000
3275000
3275000
3275000
3275000
3275000
3275000
3275000
5
 
Output: 
4
 
Output: 
10
8
 
Output: 
20
 
Output: 
(4,0) (4,0)
(4,0) (4,0)
 
Output: 
(8,0) (8,0)
(8,0) (8,0)
 
Output: 
(0.707107)|0> + (0.707107)|1>
(0.707107)|0> + (-0.707107)|1>
 (0.707107,0)
(-0.707107,0)
 
Output: 
(0,0) (0,0)
(0,0) (0,0)
 
Output: 
 (4,5)  (6,6)
 (7,8) (9,10)
 
Output: 
(-0,-4.5)
 
Output: 
5
8
 
Output: 
(0,0) (1,0) (1,0) (0,0)
 
Output: 
<01| + <10|
(0,0) (1,0) (1,0) (0,0)
 
Output: 
(1,-0) (4,-0)
(9,-0) (5,-0)
(9,-0) (5,-0)
 
Output: 

 
Output: 
1
 
Output: 
(1,0) (9,0)
(4,0) (5,0)
 
Output: 
(4.5,0)
(0,4.5)
 
Output: 
-5
 
Output: 
8
 
Output: 
1
 
Output: 
8
 
Output: 
8
 
Output: 
8
 
Output: 
1

\end{lstlisting}
\subsection{Failed cases}
comp\_wrong\_decl.ql
\begin{lstlisting}
# Winnie Narang
def func_test(comp val1, comp val2) : comp ret_name { 

  comp val3;
    val3 = 1;

  ret_name = val1 + val2 * val3;
}
def compute() : comp  ret_name {

  comp comp1;
  comp comp2;

  if (1) {1; 2+3;} else {3+6;}

  comp1 = C(7.5I);
  comp2 = C(3.2 + 1.I);

  ret_name = func_test(comp1, comp2);
}
\end{lstlisting}

func\_decl\_twice.ql
\begin{lstlisting}
# Winnie Narang
def func_test1(int z) : int ret_name { 
        int a;
        int b;
        int d;
        a = z;
        ret_name = z;

}
def func_test1(int z) : int ret_name2 { 

        ret_name2 = z;

}
def compute( int a):int trial {
      
      trial = func_test1(4);
}

\end{lstlisting}
if\_stmt.ql
\begin{lstlisting}
# Winnie Narang
def func_test(int z) : int ret_name { 
       
       int a;
       int b;
       a = 10;

       else
       {
       a = 10;
       b = 24;
       }
}


def compute(int a):int trial {

}

\end{lstlisting}

invalid\_use\_binop.ql
\begin{lstlisting}
# Winnie Narang
def compute() : int ret_name_test
      {
        int test_int;
        ret_name_test = test_int - + test_int;
        
      }
\end{lstlisting}
mat\_type.ql
\begin{lstlisting}
# Winnie Narang
def test_func(comp a, comp b, comp c, comp d) : mat ret_val { 

  mat x;
  mat f;
  x = [(a,b)(c,f)];
  ret_val = x;
}

def compute() : mat ret_val {


  comp a;
  comp b;
  comp c;
  comp d;

  a = C(4.+5.I);
  b = C(6.+6.I);
  c = C(7.+8.I);
  d = C(9.+10.I);

  ret_val = test_func(a, b, c, d);
}
\end{lstlisting}
mixed\_datatypes.ql
\begin{lstlisting}
# Winnie Narang
def func_test(int z) : int ret_name { 
        int a;
        comp b;
        int d;
        a = z;
        b = C(7.5I);
        
        d = a+b*a+b/a-b;

        ret_name=d;
}
def compute( int a ): int trial {
      
      trial = func_test(35);
}
\end{lstlisting}
no\_compute.ql
\begin{lstlisting}
# Winnie Narang
def func_test(float z) : float ret_name { 
        
        float a; 
        a = 5.8;
       
        ret_name = z;  
}
\end{lstlisting}
print\_stmt.ql
\begin{lstlisting}
# Winnie Narang
def func_test(int z) : int ret_name { 
        int a; 
        a = 5; 
        a = z;
        ret_name = a;        
}
def compute(int a):int trial {
   
        printq(a);
}
\end{lstlisting}

un\_op\_adj.ql
\begin{lstlisting}
# Winnie Narang
def func_test(mat z) : mat ret_name { 
        mat a;
        comp b;
        
        a =[(1,9,9)(4,5,5)];
        z = adj(b);

}


def compute(int a):int trial {

}
\end{lstlisting}
un\_op\_conj.ql
\begin{lstlisting}
# Winnie Narang
def func_test(mat z) : mat ret_name { 
        mat a;
        float b;

        a =[(1,9,9)(4,5,5)];
        
        b = conj(a);
}
def compute(int a):int trial {

}
\end{lstlisting}
un\_op\_cos.ql
\begin{lstlisting}
# Winnie Narang
def func_test(int z) : int ret_name { 
        int a;
        int b;
        a = 90;
        b = cos(a);

        comp d;
        d = C(7.5I);
        
        z = cos(d);
        ret_name=b;
}
def compute(int a):int trial {

}
\end{lstlisting}
undec\_func\_call.ql
\begin{lstlisting}
# Winnie Narang
def func_test1(int z) : int ret_name { 
        int a;
        int b;
        int d;
        a = z;
        
        ret_name = z;

}
def compute( int a):int trial {
      
        trial = func_test(4);
}
\end{lstlisting}
unmatched\_args.ql
\begin{lstlisting}
# Winnie Narang
def func_test1(int z, int c) : int ret_name { 
        int a;
        int b;
        int d;
        a = z;
        
        ret_name = z;

}

def compute( int a):int trial {
      
      trial = func_test1(4);
}
\end{lstlisting}
var\_undeclared.ql
\begin{lstlisting}
# Winnie Narang
def compute() : int ret_name_test
      {
        int test_int;
        ret_name_test = test_float;
        
      }
\end{lstlisting}
\subsection{Output for failed cases}
test.out
\begin{lstlisting}
#generated for test cases under SemanticFailures
Fatal error: exception Analyzer.Except("Invalid assignment to variable: val3")
Fatal error: exception Analyzer.Except("Invalid function declaration: func_test1 was already declared")
Fatal error: exception Parsing.Parse_error
Fatal error: exception Parsing.Parse_error
Fatal error: exception Analyzer.Except("Invalid matrix: incorrect type")
Fatal error: exception Analyzer.Except("Invalid assignment to variable: d")
Fatal error: exception Analyzer.Except("Missing 'compute' function")
Fatal error: exception Analyzer.Except("Invalid function call: incorrect type for parameter")
Fatal error: exception Analyzer.Except("Invalid use of unop: 'Adj(expr)'")
Fatal error: exception Analyzer.Except("Invalid assignment to variable: b")
Fatal error: exception Parsing.Parse_error
Fatal error: exception Analyzer.Except("Invalid function call: func_test was not declared")
Fatal error: exception Analyzer.Except("Invalid function call: incorrect number of parameters")
Fatal error: exception Analyzer.Except("Invalid use of a variable: test_float was not declared")
\end{lstlisting}
\section{C++ Helper files}
\subsection{qlang.cpp}
\begin{lstlisting}
//Jonathan Wong
#include <Eigen/Dense>
#include <iostream>
#include <complex>
#include <string>
#include <cmath>
#include "qlang.hpp"

using namespace Eigen;
using namespace std;


MatrixXcf tensor(MatrixXcf mat1, MatrixXcf mat2) {

	int mat1rows = mat1.rows();
	int mat1cols = mat1.cols();
	int mat2rows = mat2.rows();
	int mat2cols = mat2.cols();

	MatrixXcf output(mat1rows * mat2rows, mat1cols * mat2cols);

	//iterates through one matrix, multiplying each element with the whole
	//2nd matrix
	for(int m = 0; m < mat1rows; m++) {
		for(int n = 0; n < mat1cols; n++) {
			output.block(m*mat2rows,n*mat2cols,mat2rows,mat2cols) = 
				mat1(m,n) * mat2;
		}
	}

	return output;
	
}

Matrix4cf control(Matrix2cf mat) {
	Matrix4cf output;
	output.topLeftCorner(2,2) = IDT;
	output.topRightCorner(2,2) = Matrix<complex<float>,2,2>::Zero();
	output.bottomLeftCorner(2,2) = Matrix<complex<float>,2,2>::Zero();
	output.bottomRightCorner(2,2) = mat;

	return output;
}

MatrixXcf genQubit(string s, int bra) {
	
	int slen = s.length();
	int qlen = pow(2,slen); //length of vector

	int base10num = 0;

	//iterates through qstr. Whenever digit is a 1, it adds the associated 
	//power of 2 for that position to base10num
	const char * cq = s.c_str();
	char * c = new char();
	for(int i = 0; i < slen; i++) {
		strncpy(c,cq+i,1);
		base10num += strtol(c,NULL,10) * pow(2,(slen-1-i));  
	}
	delete c;

	//creates the vector and sets correct bit to 1
	MatrixXcf qub;
	if(bra) {
		qub = MatrixXcf::Zero(1,qlen);
		qub(0,qlen-1-base10num) = 1;
	} else if(!bra){
		qub = MatrixXcf::Zero(qlen,1);
		qub(base10num,0) = 1;
	}

	return qub;
}

string vectorToBraket(MatrixXcf qub) {
	int bra;
	int qlen;

	//determines whether bra or ket
	if(qub.rows() == 1) { qlen = qub.cols(); bra = 1; }
	else if(qub.cols() == 1) { qlen = qub.rows(); bra = 0;}
	else { //prints reg matrix if not row or column vector
		//cerr << "Incorrect matrix size for vectorToBraket" << endl;
		//exit(1);
		ostringstream  test;
		test << qub << endl;
		return test.str();
	}

	//gets position of 1 in the qubit
	complex<float> zero(0,0);
	int xi = 0;
	int yi = 0;
	int number;
	int index;
	string result;
	int count = 0;
	for(index = 0; index < qlen; index++) {
		if(bra) { xi = index; }
		else { yi = index; }

		if(qub(yi,xi) != zero) {
			//if(bra) { number = qlen-1-index; }
			//else { number = index; }
			number = index;

			//converts position to binary number reversed
			string bin = "";
			do {
				if ( (number & 1) == 0 )
					bin += "0";
				else
					bin += "1";

				number >>= 1;
			} while ( number );

			int outQubLen = sqrt(qlen);

			//adds necessary 0s
			for(int i = bin.length(); i < outQubLen; i++) {
				bin += "0";
			}

			reverse(bin.begin(), bin.end()); //reverses

			ostringstream convert;
			float re = qub(yi,xi).real();
			float im = qub(yi,xi).imag();
			string oper = "";
			string rstr = "";
			string istr = "";

			//adds constant expression
			convert << "(";
			if(re != 0) { convert << re; }
			if(re != 0 && im != 0) { convert << "+"; }
			if(im != 0) { convert << im << "i"; }
			convert << ")";

			//cleans up (1) and (1i) cases
			string constant = convert.str();
			if(constant.compare("(1)") == 0) { constant = ""; }
			else if(constant.compare("(1i)") == 0) { constant = "i"; }

			//generates appropriate bra or ket representation
			string qubstr;
			if(bra) { qubstr = constant + "<" + bin + "|"; }
			else { qubstr = constant + "|" + bin + ">"; }

			if(count > 0) {
				result += " + " + qubstr;
			} else { result = qubstr; }
			count++;
		}
	}
	return result;
}
\end{lstlisting}
\subsection{qlang.hpp}

\begin{lstlisting}
//Jonathan Wong
#ifndef QLANG_HPP_ 
#define QLANG_HPP_

using namespace Eigen;
using namespace std;

//CONSTANTS
const Matrix2cf H = (Matrix2cf() << 1/sqrt(2), 1/sqrt(2),
				1/sqrt(2), -1/sqrt(2)).finished();
const Matrix2cf IDT = Matrix2cf::Identity();
const Matrix2cf X = (Matrix2cf() << 0, 1, 1, 0).finished();
const Matrix2cf Y = (Matrix2cf() << 0, -std::complex<float>(0,1),
				std::complex<float>(0,1), 0).finished();
const Matrix2cf Z = (Matrix2cf() << 1, 0, 0, -1).finished();

//METHODS
MatrixXcf tensor(MatrixXcf mat1, MatrixXcf mat2);
Matrix4cf control(Matrix2cf mat);
MatrixXcf genQubit(string s, int bra);
MatrixXcf genQubits(string s);
string vectorToBraket(MatrixXcf qub);


#endif
\end{lstlisting}