\section{Expressions}
\label{sec:expressions}

\subsection{Primary Expressions}
\label{ssec:primaryexpresions}
\sys{} primary expressions can be identifiers, constants, or strings.
The type of the expression depends on the identifier, constant or string.

\subsection{Unary Operators}
\label{ssec:unaryoperators}
There are two unary operators in \sys{}, and they are only used with a
numeric-valued operand such as a numeric constant
(see~\ref{sssec:numericconstants}).
These expressions are grouped right-to-left:
\startsyn
\texttt{+}\emph{unsigned-integer} \\
\texttt{-}\emph{unsigned-integer}
\stopsyn

\subsubsection{\texttt{+} operator}
This operator forces the value of its numeric operand to be positive.
The resulting expression is of numeric type with a value equal to the
value of the numeric operand.

\subsubsection{\texttt{-} operator}
This operator forces the value of its numeric operand to be negative.
The resulting expression is of numeric type with a value equal to the
negative of the numeric operand.

\subsection{Channel/Calc Expresions}
\label{ssec:channelexpressions}
Channel and \texttt{Calc} types are the basis of \texttt{Image} and
\texttt{Kernel} objects respectively. There are several operators that
manipulate Channels and \texttt{Calc}s.

\subsubsection{\texttt{:} operator}
\label{sssec:colonop}
Extract, reference or assign an individual Channel in an image.
\startsyn
\emph{image-identifier}\texttt{:}\emph{channel-identifier}
\stopsyn
The resulting expression has a type corresponding to the
extracted Channel.

\subsection{Composition Operators}
\label{ssec:compositionops}
These operators compose an \texttt{Image} or \texttt{Kernel} from one
or more \texttt{Calc} objects.
All channel composition operators are left-to-right associative.

\subsubsection{\texttt{|} operator}
\label{sssec:barop}
Define of list of (one or more) \texttt{Calc}s. The resulting expression is a
\emph{multi-calc} expression, and can be assigned
to a \texttt{Kernel} object.
\startsyn
\texttt{|} \emph{calc-expression} \\
\emph{multi-calc-expression} \texttt{|} \emph{calc-expression} \\
\stopsyn
Note that \texttt{Calc}s are appended in order, and
subsequent operations may rely on this order.
Also note that even single \texttt{Calc} identifiers must be preceded by \texttt{|}

\subsection{\texttt{**} operator}
\label{ssec:convolutionop}

The convlution operator, or \texttt{**}, performs the calculations of a
\texttt{Kernel} on a Channel, and evaluates to a new \texttt{Image}. The
resulting \texttt{Image} will have a Channel for the result of each
(non-transient) calculation.
When a \texttt{Kernel} is evaluated, as many operations as possible are
parallelized to improve performance.

Parallelized calculations may depend on each other as long as the
dependent calculation is listed after its dependency when the \texttt{Kernel} is
defined, and the dependent calculation does not depend on more than
the current pixel of the required calculation.

Any channels marked with an \texttt{@} symbol are transient and
are not added to the result \texttt{Image}.

\subsection{Escaped ``C'' Expression}
\label{ssec:escapedC}

Escaped ``C'' expressions, or \emph{CStrings}, are snippits of ``C'' code
which will be run once for every pixel in an \texttt{Image} Channel. The
string may reference other Channels in the \texttt{Image} provided the
Channels have been previously defined. To reference a Channel within a
CString, the name of the Channel is used just like a local variable in C.
CStrings are used on the right side
of the \texttt{:=} operator when defining a Channel.

The escaped code must be a single expression
in C that will evaluate to the type defined by the containing \texttt{Calc} object.
The expression cannot contain the following characters/tokens:
\texttt{; \# \{ \} " ' /* */}. Parentheses may be used to group expressions or
cast objects, but all parenthesis within the expression must be matched.

\texttt{Calc} types and their C-equivalent types:
\begin{center}\begin{tabular}{l | l}
\sys type & equivalent C type \\
\hline
Uint8  & uint8_t \\
Uint16 & uint16_t \\
Uint32 & uint32_t \\
Int8  & int8_t \\
Int16 & int16_t \\
Int32 & int32_t \\
Angle  & float
\end{tabular}\end{center}

When the channel described by a CString must be evaluated,
every pixel value in the channel is calculated by evaluating the C expression.
When the expression is evaluated, every identifier corresponding to
another channel in the image will be replaced by the value of the
pixel in the same location in the referenced channel. Thus, if the C expression
contains the identifier \texttt{Red}, then when the channel is calculated
it will replace \texttt{Red} in the expression with the appropriate value
from the \texttt{Red} channel.

All standard C operators are available for use, as well as any library functions
defined in \texttt{math.h}. Although bracket characters are not allowed within
CStrings, the \emph{ternary} conditional operator, \texttt{a ? b : c} is allowed.
This enables more complex pixel value calculations such as thresholding and hysteresis.

\subsection{I/O Expressions}

\subsubsection{\texttt{imgread} expression}
\label{sssec:imgread}
The \texttt{imgread} expression reads in an \texttt{Image} object from
a known image format located on the file system. The expression results
in an \texttt{Image} object which can be assigned using the \texttt{=}
operator (see section~\ref{sssec:equalop}). The resulting \texttt{Image}
object has 3 Channels named \emph{Red}, \emph{Green}, and
\emph{Blue}. Each of the channels correspond to the red, green, and blue
image data read into the \texttt{Image} object. This expression is invoked
as a ``C'' style function, and expects 1 parameter: either the path of the
image file to read (expressed as a string constant); or the number of the
command-line argument, an integer 1 or greater.
\startsyn
\texttt{imgread(} \emph{string-constant} \texttt{|} \emph{integer} \texttt{)}
\stopsyn

\subsubsection{\texttt{imgwrite} expression}
\label{sssec:imgwrite}
The \texttt{imgwrite} expression writes out an \texttt{Image} object to a known
image format. It requires that the \texttt{Image} object has at least 3 named
\texttt{Channels}: \emph{Red}, \emph{Green}, and \emph{Blue}.
This expression has no type (null type), and is invoked as a ``C'' style function.
It expects 3 parameters: the first parameter is an \texttt{Image} identifier, the
second is the image format, and the third is the path to which the image
should be written (or an integer which represents a command-line argument, as for
\texttt{imgread}).
\startsyn
\texttt{imgwrite(} \emph{image-identifier} \texttt{,} \emph{string-constant} \texttt{,} \emph{string-constant} \texttt{|} \emph{integer} \texttt{)}
\stopsyn

\subsection{Assignment Expressions}
\label{ssec:assignment}

\subsubsection{\texttt{=} assignment operator}
\label{sssec:equalop}
Assigns the value of the right operand to the left operand, copying data as necessary.
The types of both operands must match. Cannot be used with \texttt{Calc}s, which are
defined once only (see \texttt{:=} below).

The type of this expression is equal to the type of the left operand, and assignment
operations may be chained together. For example:
\begin{lstlisting}[language=CLAM]
Image a;
Image b;
imgwrite(a = b = imgread("foo.jpg"), "png", "foo.png");
\end{lstlisting}

\subsubsection{\texttt{:=} assignment operator}
\label{sssec:colonequalop}
Assigns a calculation constant (see section~\ref{sssec:calcconstants}), or
escaped ``C'' expression (see section~\ref{ssec:escapedC}) to a \texttt{Calc}
object. Only used once for each \texttt{Calc}, with declaration.

\subsubsection{\texttt{|=} assignment operator}
\label{sssec:barequalop}
Add a single Channel or a (possibly one-member) list of \texttt{Calc}s object to an \texttt{Image} or
\texttt{Kernel} object. Assignments using this operator are ordered by statement
order, and subsequent operations can rely on this order.
